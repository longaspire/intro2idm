\begin{frame}
\frametitle{About This Work...}

\emph{Leveraging Spatio-Temporal Redundancy for RFID Data Cleansing}.~\cite{xie2015distance} \\
X.~Xie, H.~Lu, and T.~B. Pedersen.\\~\\

\begin{itemize}
  \item Published at \emph{TKDE' 2015}.
  \item Study efficient evaluation of distance-aware join operations on indoor moving objects, semi-range join and semi-neighborhood join.
  \item Design a composite index for indoor space as well as objects.
\end{itemize}

\end{frame}

%------------------------------------------------

\begin{frame}
\frametitle{Motivation}

\begin{itemize}
  \item People spend a large part of their lives in indoor spaces.

  \item the New Town Plaza in Hong Kong covers 200,000 square meters and consists of 34 interconnected buildings. The weekend traffic is as high as 320,000 people as reported in 2004.

  \item A large Danish hospital logistic system, requires tracking up to 164,000 objects, including around 10,000 persons, 10,000 pieces of equipment, 70,000 aids and 70,000 materials over 10 floors.
\end{itemize}

\end{frame}

%------------------------------------------------

\begin{frame}
\frametitle{Distance-Aware Joins}

\begin{example}[Indoor distance-based monitoring]
  \ssize{
  In a shopping plaza, a covey of mobile security guards are patrolling and monitoring the surrounding people for the suspicious, which may appear within a range. The range can be specified by a distance threshold $\epsilon$.
  }
\end{example}

\begin{example}[Indoor facility tracking]
  \ssize{
  In a large hospital logistic system, it is time-critical to monitor patients in special care or nurses on the ward with their nearest medical facilities, such as a medical staff. The number of nearby facilities can be specified by a parameter $k$.
  }
\end{example}

\begin{example}[Indoor data analysis]
  \ssize{
  Many algorithms related to similarity search and data mining can be constructed on top of a join query. For indoor spatial databases, the join operator is an important primitive that allows efficient distance-aware analysis.
  }
\end{example}

\end{frame}

%------------------------------------------------

\begin{frame}
\frametitle{Problem Definition}

\begin{block}{}
  Given two indoor objects $Q$ and $O$, let $|Q, O|_I$ denote the indoor distance from $Q$ to $O$.
\end{block}

\begin{definition}[Semi-range Join]
  \ssize{
  Given two sets of indoor objects $\mathbb{Q}$ and $\mathbb{O}$, and a distance theshold $\epsilon$, the semi-range join of the two sets returns all pairs $\{ \langle Q, O \rangle \}$ of objects, such that the distance from $Q$ to $O$ are within $\epsilon$. Formally:
  }
  \begin{equation*}
    \mathbb{Q} \underset{\epsilon}{\ltimes} \mathbb{O} = \{ (Q, O) \in \mathbb{Q} \times \mathbb{O} | |Q,O|_I \leq \epsilon \}
  \end{equation*}

\end{definition}

\begin{definition}[Semi-neighborhood Join]
  \ssize{
  Given two sets of indoor objects $\mathbb{Q}$ and $\mathbb{O}$, and an integer $k$, the semi-neighborhood join all object pairs as follows:
  }
  \begin{equation*}
    \mathbb{Q} \underset{k}{\ltimes} \mathbb{O} = \{ (Q, O) \in \mathbb{Q} \times \mathbb{O} | O \in kNN(Q) \}
  \end{equation*}

\end{definition}

\end{frame}

%------------------------------------------------

\begin{frame}
\frametitle{Problem Definition}

\begin{itemize}
  \setlength{\itemsep}{30pt}
  \item to study semi-joins instead of ful joins, $\underset{k}{\Join}$ and $\underset{\epsilon}{\Join}$
  \begin{fitemize}
    \item indoor space is a quasimetric space where distances are not symmetrics
    \item full joins can be easily implemented by semi-joins, e.g., $Q \underset{k}{\Join} O = Q \underset{k}{\ltimes} O \cap O \underset{k}{\ltimes} Q$
  \end{fitemize}
  \item call $\mathbb{Q}$ the \conceptbf{query objects}, and $\mathbb{O}$ the \conceptbf{target objects}.
\end{itemize}

\end{frame}

%------------------------------------------------

\begin{frame}
\frametitle{Challenges in Indoor Spaces}

\begin{itemize}
  \item the indoor space $\mathbb{I}$ is a quasimetric space, given two points $p, q \in \mathbb{I}$, the distance $|p, q|_I$ satisfies:
  \begin{fitemize}
    \item $|p, q|_I \geq 0$ (non-negativity);
    \item $|p, q|_I \neq |q, p|_I$ (non-symmetry);
    \item $|p, q|_I \leq |p, e|_I + |e, q|_I$ (triangle inequality)
  \end{fitemize}
  \item indoor entities can also be associated with temporal variations
  \begin{fitemize}
    \item a room may be only temporarily available due to its opening hours, or being blocked in a fire emergency
    \item a conference hall may be partitioned into several smaller rooms
  \end{fitemize}
  \item the accuracy of indoor positioning is limited, typically varying from a few to 100 meters
\end{itemize}

\textrm{To address these challenges, we need to support indoor distances taht take into account topological constraints, temporal variations and location uncertainties.}

\end{frame}

%------------------------------------------------

\begin{frame}
\frametitle{Notations}

\vspace{-15pt}
\begin{figure}[tb]
  \includegraphics[width=0.57\columnwidth]{figures/2-7/2-7-1.pdf}
\end{figure}

\end{frame}

%------------------------------------------------

\begin{frame}
\frametitle{Preliminaries: Indoor Space and Indoor Distance}

\conceptbf{Doors Graph} has been proposed to represent the connectivity of indoor partitions as well as door-to-door distances.~\cite{DBLP:conf/edbt/YangLJ10}\\~\\\pause

Given two indoor positions $p$ an $q$, we use $q \overset{\delta}{\rightsquigarrow} p$ to denote a path from $q$ to $p$ where $\delta$ is the sequence of doors on the path.\\~\\\pause

The length of the shortest path as \emph{indoor distance} from $q$ to $p$, and denote it formally as $|q, p|_{I} = min_{\delta}(|q \overset{\delta}{\rightsquigarrow} p|)$, also $q \overset{\delta}{\rightarrow} p$.\\~\\\pause

\emph{indoor distance} consists of \emph{door-door distance} and \emph{intra-partition object-door distance}:\pause
\begin{equation}
  min_{d_q \in D(q), d_p \in D(p)}(|q, d_q|_{E} + |d_q, d_p|_{I} + |d_p, p|_{E})
\end{equation}

\end{frame}

%------------------------------------------------

\begin{frame}
\frametitle{Indoor Moving Objects}

\begin{itemize}
  \item Existing proposals~\cite{pfoser1999capturing, DBLP:conf/edbt/YangLJ10} model a moving object by an \emph{uncertainty region}, where the exact location is considered as a random variable inside.
  \item The possibility of its appearance can be collected by object's velocities~\cite{DBLP:conf/edbt/YangLJ10}, parameters of positioning device~\cite{pfoser1999capturing}, or analysis of historical records (represented by \emph{pdf}).
  \item The \emph{pdf} can be either a close form equation~\cite{cheng2003evaluating,cheng2004querying} or a set of instance representation~\cite{kriegel2007probabilistic}, as it is general for arbitrary distribution.
  \item Thus, an indoor moving object $O$ is represented by a set ${(o, o.\rho)}$, where $o$ is an instance and $o.\rho$ is its \emph{existential probability}, satisfying $\sum_{o \in O}o.\rho = 1$.
\end{itemize}

\end{frame}

%------------------------------------------------

\begin{frame}
\frametitle{Expected Indoor Distance}

\begin{definition}[Expected Indoor Distance for Uncertain Object]
  Given two uncertain object $Q$ and $O$, the indoor distance between $Q$ to $O$ is
  \begin{equation}
    |Q, O|_{I} = E_{q \in Q, o \in O}(|q,o|_{I}) = \sum_{q \in Q}\sum_{o \in O}|q,o|_{I} \cdot q.\rho \cdot o.\rho
  \end{equation}
\end{definition}
\vspace{10pt}
an object $O$'s uncertainty region may overlap with multiple partitions. Accordingly, all the instances in $O$ are divided into subsets, i.e., $O = \cup_{1 \leq j \leq m}O[j](1 \leq m \leq |O|)$ where each $O[j]$ corresponds to a different partition, it is called $O$'s \emph{uncertainty subregion}.

\end{frame}

%------------------------------------------------

\begin{frame}
\frametitle{Case of Indoor Distance $|Q, O|_I$ (I)}

\conceptbf{Single-Partition Single-Path Distance} \quad $O$'s uncertainty region falls into one single partition $P$, so does $Q$. Let $P_Q$ ($P_O$) be the partition containing $Q$ ($O$). For an arbitrary pair $(q, o)_{q \in Q, o \in O}$, the shortest path $q \overset{d_Q*d_O}{\rightarrow} o$ shares the same door sequence starting with $d_Q$ and ending with $d_O$, through which the path reaches $o$ from $q$.

\begin{equation}
  \begin{split}
  |Q, O|_{I} & = \sum_{q \in Q}\sum_{o \in O} (|q, d_Q|_E + |d_Q, d_O|_I + |d_O, o|_E)\cdot q.\rho \cdot o.\rho \\
             & = \sum_{q \in Q} |q, d_Q|_E + |d_Q, d_O|_I + \sum_{o \in O} |d_O, o|_E
  \end{split}
\end{equation}

\end{frame}

%------------------------------------------------

\begin{frame}
\frametitle{Case of Indoor Distance $|Q, O|_I$ (II)}

\conceptbf{Single-Partition Multi-Path Distance} \quad $O$ and $Q$'s uncertainty region still falls into one single partition $P$. However, for different instances $o_i$ and $o_j$, the shortest path $q \overset{*}{\rightarrow} o_i$ and $q \overset{*}{\rightarrow} o_j$ do not share the same door sequence.

\begin{equation}
  |Q, O|_{I} = \sum_{o_i \in O}|q, o_i|_I \cdot q.\rho \cdot o_i.\rho
\end{equation}

\begin{columns}[c]

  \column{0.24\textwidth}
  \begin{figure}[tb]
    \includegraphics[width=\columnwidth]{figures/2-7/2-7-2.pdf}
  \end{figure}

  \column{0.76\textwidth}
  \begin{example}
    $O$ has two instance $o_1$ and $o_2$, the shortest path from $q$ to them are: $q \overset{d_3, d_1}{\rightsquigarrow} o_1$ and $q \overset{d_2}{\rightsquigarrow} o_2$.
  \end{example}

\end{columns}

\end{frame}

%------------------------------------------------

\begin{frame}
\frametitle{Case of Indoor Distance $|Q, O|_I$ (III)}

\conceptbf{Multi-Partition Multi-Path Distance} \quad either $Q$ or $O$'s uncertainty region overlaps with more than one partition, and thus $O = \cup_{1 \leq j \leq m}O[j](1 \leq m \leq |O|)$.

\begin{equation}
  |Q, O|_I = \sum_{i}\sum_{j}(|Q[i],O[j]|_I \cdot \sum_{q \in Q[i]}q.\rho \cdot \sum_{o \in O[j]}o.\rho)
\end{equation}

$|Q[i],O[j]|_I$ is calculated according to case I or case II.

\vspace{-5pt}
\begin{columns}[c]

  \column{0.2\textwidth}
  \begin{figure}[tb]
    \includegraphics[width=\columnwidth]{figures/2-7/2-7-3.pdf}
  \end{figure}

  \column{0.8\textwidth}
  \begin{example}
    $O$ has three uncertainty subregions $O[1]$, $O[2]$ and $O[3]$. Accordingly, $|Q,O|_I = E(\sum_{1 \leq j \leq 3}(|q, O[j]_I|))$.
  \end{example}

\end{columns}

\end{frame}

%------------------------------------------------

\begin{frame}
\frametitle{Bounds for Indoor Distances}

\conceptbf{Geometric Layer Lower Bounds}\\
\ssize{
Fot two indoor uncertain objects $Q$ and $O$, the (virtual) euclidean distance between them is the lower bound of their distance in the indoor space. Therefore, it has $|Q,O|_{minE} \leq |Q,O|_{minI}$, where $|Q, O|_{minE} = \min_{q \in Q, o \in O}|q,o|_E$.
}

\vspace{10pt}
\begin{lemma}[Geometric Lower Bounds]
  Given indoor object $Q$ denoted by $\odot(c_Q, r_Q)$, and $O$ denoted by $\odot(c_O, r_O)$, the geomtric lower bound property can be rewritten as:
  \begin{equation}
    |c_Q, c_O|_E - r_Q - r_O \geq |Q, O|_{minI}
  \end{equation}
\end{lemma}

\vspace{10pt}
\textrm{it is impossible to derive the indoor upper bounds by using Euclidean distances only.}

\end{frame}

%------------------------------------------------

\begin{frame}
\frametitle{Bounds for Indoor Distances}

\conceptbf{Indoor Topological ULBounds}

\vspace{30pt}

For two objects $Q = \bigcup_{i = 1}^{m} Q[i]$ and $O = \bigcup_{j = 1}^{n} O[j]$, suppose that $P(Q[i])$ is the partition containing subregion $Q[i]$ and $P(Q)$ are the partitions overlapping with $Q$.

\end{frame}


%------------------------------------------------

\begin{frame}
\frametitle{Bounds for Indoor Distances}

\begin{lemma}[Topological Lower Bounds]
  \ssize{
  Let $t_{min}(Q[i], O[j])$ be: $$\min_{d_q \in D(P(Q[i])), d_s \in D(P(O[j]))}|Q[i],d_q|_{minE} + |d_q \overset{*}{\rightarrow} d_s| + |d_s, O[j]|_{minE}$$. Then, $|Q,O|_I \geq min_{i,j}\{ t_{min}(Q[i], O[j]) \}$.
  }
\end{lemma}

\begin{lemma}[Topological Upper Bounds]
  \ssize{
  Let $t_{max}(Q[i], O[j])$ be: $$\min_{d_q \in D(P(Q[i])), d_s \in D(P(O[j]))}|Q[i],d_q|_{maxE} + |d_q \overset{*}{\rightarrow} d_s| + |d_s, O[j]|_{maxE}$$. Then, $|Q,O|_I \leq max_{i,j}\{ t_{max}(Q[i], O[j]) \}$.
  }
\end{lemma}

\ssize{\textrm{Suppose $Q$ and $O$ overlap with $m$ and $n$ partitions respectively. The above two lemmas involve $O(mn)$ shortest paths.}}

\end{frame}

%------------------------------------------------

\begin{frame}
\frametitle{Bounds for Indoor Distances}

\ssize{\textrm{If $Q$ and $O$'s uncertainty regions both overlap with one partition, the above two lemmas can be rewritten.}}

\begin{lemma}[]
  \ssize{
  Given two indoor object $Q$ and $O$, denoted by $\odot(c_Q, r_Q)$ and $\odot(c_O, r_O)$ respectively, the topological ULBounds can be rewritten as:
  }
  \begin{equation}
    |c_Q, c_O|_I - r_Q - r_O \leq |Q, O|_I \leq |c_Q, c_O|_I + r_Q + r_O
    \label{equation:simplified_1}
  \end{equation}

\end{lemma}

\begin{proof}{}
  \ssize{
  \begin{equation*}
    \begin{split}
    |Q, O|_{minI} & = \min_{\forall q \in Q}(|q, O|_I) \geq \min_{\forall q \in Q}(|q, c_O|_I - r_O) = \min_{\forall q \in Q}(|q, c_O|_I) - r_O \\
    & \geq |c_Q, c_O|_I - r_Q - r_O \\
    & \Rightarrow |Q, O|_{minI} \geq |c_Q, c_O|_I - r_Q - r_O \\
    & \Rightarrow |Q, O|_{I} \geq |c_Q, c_O|_I - r_Q - r_O \\
    \end{split}
  \end{equation*}
  $|Q, O|_I \leq |c_Q, c_O|_I + r_Q + r_O$ can be proved likewise.
  }
\end{proof}

\end{frame}

%------------------------------------------------

\begin{frame}
\frametitle{Bounds for Indoor Distances}

\textrm{The shortest path $|d_q \overset{*}{\rightarrow} d_s|$ computation is not economic, a \conceptbf{Topological Looser UBound} is proposed. }

\begin{lemma}[Topological Looser Upper Bounds]
  \ssize{
  Let $t_{max}(Q[i], O[j])$ be: $$\min_{d_q \in D(P(Q[i])), d_s \in D(P(O[j]))}|Q[i],d_q|_{maxE} + |d_q \overset{*}{\rightsquigarrow} d_s| + |d_s, O[j]|_{maxE}$$. Then, $|Q,O|_I \leq max_{i,j}\{ t_{max}(Q[i], O[j]) \}$.
  }
\end{lemma}

\textrm{In the case that both $Q$ and $O$ overlap with one partition, the lemma can be simplified as:}
\begin{equation}
  \min_{d_q \in D(P(Q[i])), d_s \in D(P(O[j]))} |d_q \overset{*}{\rightsquigarrow} d_s| + |d_q, c_Q|_E + |d_o, c_O|_E + r_Q + r_O
  \label{equation:simplified_2}
\end{equation}

\end{frame}

%------------------------------------------------

\begin{frame}
\frametitle{Bounds for Indoor Distances}

\fsize{
The simplified versions of ULBounds are more efficient since they only take one shortest path instead of $O(mn)$ paths. To generalize the single-parition case to multiple-partition scenarios, \conceptbf{star-connected region} is defined.
}

\begin{definition}[Star-connected regions]

  Let $O = \odot(c, r)$ be an indoor object overlapping with more than one partition, i.e., $O = \bigcup_{i=1}^{n}O[i]$. Let the subregion containing $c$ be the central region $C$. If all other subregions are connected to $C$ by doors, we call $O$'s region a star-connected region, formally:

  \begin{equation*}
    \forall O[i] \neq C, \exists~\text{door}~d , \text{such that}~d \in C~\text{and}~d \in O[i]
  \end{equation*}

\end{definition}


\end{frame}

%------------------------------------------------

\begin{frame}
\frametitle{Bounds for Indoor Distances}

\begin{columns}[c]

  \column{0.2\textwidth}
  \begin{figure}[tb]
    \includegraphics[width=\columnwidth]{figures/2-7/2-7-3.pdf}
  \end{figure}

  \column{0.8\textwidth}
  \begin{example}
    \ssize{
    $Q$ is a star-connected region, since $Q[1]$ and $Q[2]$ are connected by door $d_{14}$, $O$ is not a star-connected region, since $O[2]$ and $O[3]$ are separated into two partitions, and there is no door connecting the two partitions.
    }
  \end{example}

\end{columns}

\vspace{10pt}

\textrm{Then, we can define $O$ by $\odot(c, r_I)$, where $r_I$ is the maximum indoor distance from centroid $c$ to all subregions, $r_I = \max_{i} |c, O[i]|_{maxI}$. By defining star-connected regions, we can benefit from the simplifications in topological ULBounds by substituting $r_I$ into Equations.(\ref{equation:simplified_1}), (\ref{equation:simplified_2}).}

\end{frame}

%------------------------------------------------

\begin{frame}
\frametitle{Bounds for Indoor Distances}

\begin{columns}[c]

  \column{0.2\textwidth}
  \begin{figure}[tb]
    \includegraphics[width=\columnwidth]{figures/2-7/2-7-3.pdf}
  \end{figure}

  \column{0.8\textwidth}
  \begin{example}
    \ssize{
    the distance from $q$ to $O[1]$ is short, while the distance to $O[3]$ is long. If the gap between topological upper and lower bounds is large, the expected distance is only constrained by a loose range and thus not well approximated.
    }
  \end{example}

\end{columns}

\vspace{15pt}

\textrm{Geometric and topological ULBounds bound the distance by the minimum/maximum distance between sample pairs. The \conceptbf{Object Layer ULBounds} make a difference by considering the probability distributions among sample points.}

\end{frame}

%------------------------------------------------

\begin{frame}
\frametitle{Bounds for Indoor Distances}

\begin{definition}[$\beta$-region~\cite{chen2007efficient,lian2011similarity}]
  Given an indoor object $O$, the $\beta$-region is a closed region such that the probability of $O$ being located inside the region is greater than $\beta$.
\end{definition}

\end{frame}

%------------------------------------------------

\begin{frame}
\frametitle{Bounds for Indoor Distances}

\textbf{CASE I} \quad Object's uncertainty region is relatively big compared to its indoor distance.

\vspace{30pt}

\begin{columns}[c]

  \column{0.3\textwidth}
  \begin{figure}[tb]
    \includegraphics[width=\columnwidth]{figures/2-7/2-7-4.pdf}
  \end{figure}

  \column{0.7\textwidth}
  \begin{example}
    \ssize{
      Given a predefined $\beta$ balue, the $\beta$-region can be constructed b first sorting an object's samples according to their distances from the centroid. Count and summarize their probabilities until $\beta$ is reached. The distance between the last counted sample point and the centroid is $r^\beta$. Thus, $O$'s $\beta$-region $O^\beta$ is determined by a circle $\odot(c, r^\beta)$.
    }
  \end{example}

\end{columns}
\end{frame}

%------------------------------------------------

\begin{frame}
\frametitle{Bounds for Indoor Distances}

\textbf{CASE II} \quad Object overlaps with multiple partitions that are not interconnected. (not star-connected regions)

\vspace{30pt}

\begin{columns}[c]

  \column{0.3\textwidth}
  \begin{figure}[tb]
    \includegraphics[width=\columnwidth]{figures/2-7/2-7-5.pdf}
  \end{figure}

  \column{0.7\textwidth}
  \begin{example}
    \ssize{
      Randomly select a subregion $O[i]$ as the $\beta$-region. Here the value of $\beta$ equals to the summation of probabilities for samples inside $O[i]$, i.e., $\beta = \sum_{s \in O[i]}s.\rho$. The shape of the $\beta$-region is a rectangle, which is the intersection of $O$'s MBR and the partition containing $O[i]$.
    }
  \end{example}

\end{columns}

\end{frame}

%------------------------------------------------

\begin{frame}
\frametitle{Bounds for Indoor Distances}

\begin{lemma}[\small Simplified Case for Deriving Object Layer ULBounds]
  \tiny
  Given a point $q$ and an object $O$, we have:
  \begin{equation*}
    \begin{split}
    & (1 - \beta) \cdot |q, O|_{minI} + \beta \cdot |q, O^\beta|_{minI} \leq |q, O|_I  \\
    & \leq (1 - \beta) \cdot |q, O|_{maxI} + \beta \cdot |q, O^\beta|_{maxI}
    \end{split}
  \end{equation*}
\end{lemma}

\begin{proof}{\small}
  \tiny
  We first prove $|q, O|_I \leq (1 - \beta) \cdot |q, O|_{maxI} + \beta \cdot |q, O^\beta|_{maxI}$:
  \begin{equation*}
    \begin{split}
      & |q, O|_I = E(|q, O|_I) = E_{s \in O^\beta}(|q, s|_I) \cdot Pr\{s \in O^\beta \} + \\
      & E_{s \in O \setminus O^\beta} (|q, s|_I) \cdot Pr\{s \in O \setminus O^\beta \} \\
      & = E_{s \in O^\beta}(|q, s|_I) \cdot \beta + E_{s \in O \setminus O^\beta} (|q, s|_I) \cdot (1 - \beta) \} \\
      & \leq q, O|_{maxI} + \beta \cdot |q, O^\beta|_{maxI} (\forall s \in O^\beta, |q,s|_I \leq |q, O^\beta|_{maxI})
    \end{split}
  \end{equation*}
  Likewise, we can prove $(1 - \beta) \cdot |q, O|_{minI} + \beta \cdot |q, O^\beta|_{minI} \leq |q, O|_I$, thus the lemma is proved.
\end{proof}

\end{frame}

%------------------------------------------------

\begin{frame}
\frametitle{Bounds for Indoor Distances}

\begin{lemma}[\small Object Layer ULBounds]
  Suppose two objects $Q$ and $O$, $Q$'s $\beta_Q$-region is $Q^\beta$, $o$'s $\beta_O$-region is $O^\beta$ we have:
  \begin{equation*}
    \begin{split}
    & (1 - \beta_Q) \beta_O \cdot |Q, O^\beta|_{minI} + \beta_Q \beta_O \cdot |Q^\beta, O^\beta|_{minI} + \\
    & (1 - \beta_Q) (1 - \beta_O) \cdot |Q, O|_{minI} + \beta_Q (1 - \beta_O) \cdot |Q^\beta, O|_{minI} \\
    & \leq |Q, O|_I  \\
    & (1 - \beta_Q) \beta_O \cdot |Q, O^\beta|_{maxI} + \beta_Q \beta_O \cdot |Q^\beta, O^\beta|_{maxI} + \\
    & (1 - \beta_Q) (1 - \beta_O) \cdot |Q, O|_{maxI} + \beta_Q (1 - \beta_O) \cdot |Q^\beta, O|_{maxI}
    \end{split}
  \end{equation*}
\end{lemma}

\end{frame}

%------------------------------------------------

\begin{frame}
\frametitle{Bounds for Indoor Distances}

\begin{proof}{\small}
  Assume $q$ is a point inside $Q$, we have:
  \begin{equation*}
    \begin{split}
      & E(|Q, O|) = E(|Q,O|_I ~|~ q \in Q^\beta) \cdot Pr\{q \in Q^\beta\} + \\
      & E(|q,O|_I ~|~ q \in Q \setminus Q^\beta) \cdot Pr\{q \notin Q^\beta\} = \\
      & E(|Q,O|_I ~|~ q \in Q^\beta) \cdot \beta_Q + E(|q,O|_I ~|~ q \in Q \setminus Q^\beta) \cdot (1 - \beta_Q)
    \end{split}
  \end{equation*}
  for Case $q \in Q^\beta$, $E(|q, O|_I ~|~ q \in Q^\beta) \cdot \beta_Q \leq \beta_Q \beta_O \cdot |Q^\beta, O^\beta|_{maxI} + \beta_Q (1-\beta_O) \cdot |Q^\beta, O|_{maxI}$.\\
  for Case $q \in Q \setminus Q^\beta$, $E(|q, O|_I ~|~ q \in Q \setminus Q^\beta) \cdot (1 - \beta_Q) \leq (1 - \beta_Q) \beta_O \cdot |Q, O^\beta|_{maxI} + (1 - \beta_Q) (1-\beta_O) \cdot |Q, O|_{maxI}$.\\
  Summarize the two cases, we can get the conclusion. The lower bound part can be proved in a similar way.
\end{proof}

\end{frame}

%------------------------------------------------

\begin{frame}
\frametitle{Bounds for Indoor Distances}

\centering
\begin{tabular}{|c|c|}
\hline
\textbf{Cases} & Bounds \\
\hline
\tabincell{c}{single-partitioned region \\ star-connected region} & \tabincell{c}{Geometric Layer ULBounds \\ Topological Layer ULBounds} \\
\hline
\tabincell{c}{multi-partitioned region \\ big uncertainty region} & Object Layer ULBounds \\
\hline
\end{tabular}

\begin{itemize}
  \item to use geometric and topological ULBounds for the case that an object overlaps a single partition.
  \item to use probabilistic ULBounds for the case that an object overlaps with multiple partitions.
\end{itemize}

\end{frame}

%------------------------------------------------

\begin{frame}
\frametitle{Composite Index for Indoor Space}

\begin{columns}[c]

  \column{0.5\textwidth}
  \begin{figure}[tb]
    \includegraphics[width=\columnwidth]{figures/2-6/2-6-8.pdf}
  \end{figure}

  \column{0.5\textwidth}
  \begin{fitemize}
    \item \conceptbf{geometric layer} consists of a tree structure that adapts the R$^*$-tree to index all partitions, as well as a skeleton tier that maintains a small number of distances between staircases.
    \item \conceptbf{topological layer} maintains the connectivity information between indoor partitions.
    \item \conceptbf{object layer} stores all indoor moving objects and is associated with the tree through partitions at its leaf level.
  \end{fitemize}

\end{columns}

\end{frame}

%------------------------------------------------

\begin{frame}
\frametitle{Composite Index: Overview}

\begin{figure}[tb]
  \includegraphics[width=\columnwidth]{figures/2-6/2-6-9.pdf}
\end{figure}

\end{frame}

%------------------------------------------------

\begin{frame}
\frametitle{Composite Index: Tree Tier}

\begin{fitemize}
  \item instead of 3D $Minimum Bounding Rectangle$, when creating the tree, set the vertical length for one partition to 1 centimeter. Two advantage: 1) reduce the distance calculation workload; 2) makes the distance reflected in the tree more accurate without the disturbance from the vertical dimension.
  \item the imbalanced partition are decomposed to small but regular region, each is called an \emph{index unit}.
  \item A hash table is used to map such an index unit to its original indoor partition.
  \item in addition to the MBRs, a leaf node (index unit) also stores:
    \begin{sitemize}
      \item a linked bucket for all objects inside it (Object Layer)
      \item links to its connected partitions (Topological Layer)
    \end{sitemize}
\end{fitemize}

\end{frame}

%------------------------------------------------

\begin{frame}
\frametitle{Composite Index: Tree Tier - Augmented Tree Tier}

\textbf{Basic Idea} of performing a spatial join is \emph{to use the property that the MBR of an index node covers the MBRs of its subtree}.

\vspace{20pt}

To maintain the partial order property taht eases join processing, it augments each tree node $t$ with two attributes, $\{ t.r_{max}, t.r_{count} \}$.
\begin{fitemize}
  \item measure an object's size by the length of its MBR's longest dimension
  \item $t.r_{max}$ to represent the largest object size of $t$'s subtree
  \item $t.r_{count}$ to represent the number of objects associated with $t$'s subtree
\end{fitemize}

\vspace{20pt}

Consequently, the \emph{augmented area} of $t$ is the Minkovski sum of $t$'s sum of $t$'s MBR and its $r_{max}$, denoted by $t \oplus t.r_{max}$.

\end{frame}

%------------------------------------------------

\begin{frame}
\frametitle{Composite Index: Object Tier}

A hash table $o-table$

\begin{equation*}
  o-table : \{ O \} \rightarrow 2^{\{index~unit\}}
\end{equation*}

$o-table$ maps an object to all the index units it overlaps, and it is tightly tie up with the tree tier.\\~\\

When an object update occurs, $o-table$ needs to be updated accordingly.

\end{frame}

%------------------------------------------------

\begin{frame}
\frametitle{Composite Index: Topological Tier}

This layer maintains the connectivity between partitions. Each leaf node stores a (sub)partition.\\~\\

For accessibility, the doors belonging to the partitions are also stored, as well as the the links to accessible partitions through each door.

\end{frame}

%------------------------------------------------

\begin{frame}
\frametitle{Composite Index: Skeleton Tier}

Skeleton Tier is a graph, each staircase entrance is captured as a graph node, and an edge connects two nodes if their entrances are on the same floor or their entrances belong to the same staircase.\\~\\

The weight of an edge is the indoor distance between the two staircase entrances.

\vspace{-10pt}
\begin{columns}[c]

  \column{0.4\textwidth}
  \begin{figure}[tb]
    \includegraphics[width=\columnwidth]{figures/2-6/2-6-11.pdf}
  \end{figure}

  \column{0.6\textwidth}
  \ssize{
  \begin{definition}[staircase distance matrix $M_{s2s}$]
    \begin{sitemize}
      \item $M_{s2s}[s_i,s_i] = 0$;
      \item $M_{s2s}[s_i,s_j] = |s_i, s_j|_E$ if $s_i$ and $s_j$ are on the same floor;
      \item if $s_i$ and $s_j$ are of a same staircase, $M_{s2s}[s_i,s_j]$ is the shortest distance from $s_i$ to $s_j$ within that staircase;
      \item $M_{s2s}[s_i,s_j]$ is calculated as the shortest path distance from $s_i$ to $s_j$ in the skeleton layer for other cases.
    \end{sitemize}
  \end{definition}
  }

\end{columns}

\end{frame}

%------------------------------------------------

\begin{frame}
\frametitle{Skeleton Distance}

\textrm{Let $q$ be a fixed indoor point, $q.f$ the floor of $q$, and $S(q.f)$ all the staircases on floor $q.f$.}

\vspace{10pt}
\begin{definition}[Skeleton Distance]
  Given two points $p$ and $q$, their skeleton distance $|q,p|_K = |q,p|_E$ if they are on the same floor; otherwise, $|q,p|_K = \min_{s_q \in S(q.f), s_p \in S(p.f)}(|q,s_q|_E + M_{s2s}[s_q,s_p] + |s_p, p|_E)$.
\end{definition}

\vspace{10pt}
Define the skeleton distance as the alternative \emph{Geometric Distance}.

\end{frame}

%------------------------------------------------

\begin{frame}
\frametitle{Indoor Distance Bounds in the Geometric Layer}

\begin{lemma}[Geometric Lower Bound Property]
  Given two points $p$ and $q$, their skeleton distance lower bounds their indoor distance, i.e., $|q,p|_K \leq |q,p|_I$.\\~\\
  \textbf{Proof:}~If $q$ and $p$ are on the same floor, $|q,p|_K = |q,p|_E \leq |q,p|_I$. Otherwise, suppose $s_{q}^{*} \in S(q.f)$ and $s_{p}^{*} \in S(p.f)$ are on the shortest path from $q$ to $p$, denoted by $q \overset{*s_{q}^{*}*s_{p}^{*}}{\rightarrow} p$. Since $|q,p|_K = \min_{s_q \in S(q.f), s_p \in S(p.f)}(|q,s_q|_E + M_{s2s}[s_q,s_p] + |s_p, p|_E) \leq |q,s_{q}^{*}|_E + M_{s2s}[s_{q}^{*},s_{p}^{*}] + |s_{p}^{*}, p|_E = |q,p|_I$, the lemma is proved.
\end{lemma}

\end{frame}

%------------------------------------------------

\begin{frame}
\frametitle{Indoor Distance Bounds in the Geometric Layer}

Consider an entity $e$ that is either an object or an $ind$R-tree node. If $e$ spans multiple floors, we use interval $[e.lf,e.uf]$ to represent all those floors. Note those floors must be consecutive. We define the minimum skeleton distance $|q,e|_{minK}$:

\begin{figure}[tb]
  \includegraphics[width=0.7\columnwidth]{figures/2-6/2-6-12.pdf}
\end{figure}

With $|q,e|_{minK}$, one can constrain the search via the $ind$R-tree to a much smaller range compared to if use the Euclidean distance bounds.

\end{frame}
