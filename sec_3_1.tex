\begin{frame}
\frametitle{About This Work...}

\emph{Leveraging Spatio-Temporal Redundancy for RFID Data Cleansing}.~\cite{chen2010leveraging} \\
H.~Chen, W-S.~Ku, H.~W and M-T.~S.\\~\\

\begin{itemize}
  \item Published at \emph{SIGMOD' 2010}.
  \item Proposed a Bayesian inference based approach for cleaning RFID raw data.
  \item Desined an $n$-state detection model to capture the likelihood.
  \item Devised a Metropolis-Hastings sampler with Constraints (MH-C) to sample from the posterior.
\end{itemize}

\end{frame}

%------------------------------------------------

\begin{frame}
\frametitle{Motivation}

To take full advantage of:

\begin{itemize}
  \item duplicate readings (by multiple readers simultaneously or by a single reader over a period of time) of the same object are very common.
  \item prior knowledge about the readers and the environment (e.g., prior data distribution, false negative rates of readers) may help improve data quality and remove data anomalies.
  \item given constraints in target applications (e.g., the number of objects in a same location cannot exceed a given value).
\end{itemize}

\end{frame}

%------------------------------------------------

\begin{frame}
\frametitle{Data Redundancy: Spatial Redundancy}

\fsize{\textrm{The challenge is how to take advantage of redundancy while avoiding its undesirable effect in data cleansing.}}

\vspace{-10pt}
\begin{columns}[c]

  \column{0.5\textwidth}
  \begin{figure}[tb]
    \includegraphics[width=\columnwidth]{figures/3-1/3-1-2.pdf}
  \end{figure}

  \begin{example}
    \ssize{
    \textrm{
    the target area is divided into 6 zones, an RFID reader is located in the center of each zone. Spatial overlap of readers' detection regions leads to duplicate readings, i.e., an object is in the detection regions of multiple readers.
    }}
  \end{example}

  \column{0.5\textwidth}
  \begin{figure}[tb]
    \includegraphics[width=\columnwidth]{figures/3-1/3-1-1.pdf}
  \end{figure}

  \ssize{
  The above table shows two effects of redundancy:
  \begin{enumerate}
    \item Object 2 is detected by both readers in Zone 2 and 3, at least one of the readings belongs to spatial redundancy.
    \item Object 3 is detected in Zone 4 only, however, it does not necessarily mean that the Object 3 is in Zone 4 for sure.
  \end{enumerate}
  }

\end{columns}

\end{frame}

%------------------------------------------------

\begin{frame}
\frametitle{Data Redundancy: Temporal Redundancy}

Many applications monitor the target area using a \emph{mobile reader} instead of employing multiple \emph{stationary readers}.\\~\\

Because the exact location of the mobile reader is always changing, the detection regions at different time points may overlap.\\~\\

\textbf{The temporal redundancy problem can be reduced to the spatial redundancy problem:} by treating the same reader at different time points as different readers.

\end{frame}

%------------------------------------------------

\begin{frame}
\frametitle{Prior Knowledge}

As false negatives and false positives abound in raw RFID readings, in order to revoer the true information, the data cleasing system should take prior knowledge into account.\\~\\

For example, the detection areas of readers in Zone 2 and 3 have significant overlapping, the positioning of the reader in Zone 4 makes it more likely to detect objects in Zone 3 than objects in Zone 5, or the reader in Zone 3 has high false negative rate.

\end{frame}

%------------------------------------------------

\begin{frame}
\frametitle{Constraints}

\emph{Environmental constraints} can be utilized to improved data cleansing.\\~\\

For example, the maximal capacity of each zone (the number of objects that can reside the same zone) is a constraint.\\~\\

In addition to these pysical constraints, information obtained from other channels can be translated into constraints. E.g., if an extra source indicates that two certain objects are in the same zone, it may help cleanse the data of there two.

\end{frame}

%------------------------------------------------

\begin{frame}
\frametitle{Overview of the Approach}

\begin{enumerate}
  \item By using Bayesian inference, it derives a universal framework of computing the posterior probabilities (of the location of each object).
  \item Based on the physical characteristic of RFID readers, it proposes an $n$-state detection model to capture likelihoods.
  \item It devised MH-C, an improved \emph{Metropolis-Hastings} sampler, to sample from the posterior while taking the environmental constraints into consideration.
\end{enumerate}

\end{frame}

%------------------------------------------------

\begin{frame}
\frametitle{Notations}

\begin{figure}[tb]
  \includegraphics[width=\columnwidth]{figures/3-1/3-1-3.pdf}
\end{figure}

\end{frame}

%------------------------------------------------

\begin{frame}
\frametitle{A Bayesian Interface Base Approach}

\conceptbf{Bayesian Interface} estimates the probalility of a hypothesis $(x)$ based on observations $(y)$, showing that posterior is proportional to the multiplication of likelihood and prior, i.e., $p(x|y) \propto p(y|x)p(x)$.\\~\\\pause

Suppose there're $m$ zones (each with a reader mounted in the center) and $n$ objects. For each object $o_i$, its location is represented by a random variable $h_i$. A possible distribution of $n$ objects in $m$ zones can be denoted as an instance of the random vector:\pause
\begin{equation}
  \hat{H} = \{ h_1, h_2, ..., h_n \}
\end{equation}
\pause

where $h_i$ represents the zone ID where object $o_i$ is in. E.g., $h_1 = 2$ denotes that object $o_1$ is in zone 2 in the current instance.

\end{frame}

%------------------------------------------------

\begin{frame}
\frametitle{A Bayesian Interface Base Approach}

For the reader in zone $j$, the raw data (0 or 1) it receives from the RFID tag of objects $o_i$ is denoted as $z_{ij}$. Thr \emph{raw data matrix} for each complete scan from $m$ readers can then be represented as an $n \times m$ matrix $\mathbb{Z} = [z_{ij}]$. \\~\\ \pause

Using the \conceptbf{Bayesian theorem}, where $post(\hat{H}|\mathbb{Z})$ denotes the posterior probability of location vector $\hat{H}$ given the raw data matrix $\mathbb{Z}$. The hypothesis should satisfy all constraints: \pause
\begin{equation}
  \begin{split}
  post(\hat{H}|\mathbb{Z}) = 0 & :\hat{H} \textrm{ is not valid} \\
  post(\hat{H}|\mathbb{Z}) > 0 & :\hat{H} \textrm{ is valid} \\
  post(\hat{H_1}|\mathbb{Z}) > post(\hat{H_2}|\mathbb{Z}) & :\hat{H_1} \textrm{ is more likely than } \hat{H_2}
  \end{split}
\end{equation}

\end{frame}
