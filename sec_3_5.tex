\begin{frame}
\frametitle{About This Work...}

\emph{Learning-Based Cleansing for Indoor RFID Data}~\cite{baba2016learning}\\
A.I.~Baba, M.~Jaeger, H.~Lu, T.B.~Pedersen, W.-S.~Ku, X.~Xie.\\~\\

\begin{itemize}
  \item Published at \emph{SIGMOD' 2016}.
  \item Proposed a learning-based data cleansing approach that, requires no detailed prior knowledge about the spatio-temporal properties of the indoor space and the RFID reader deployment.
  \item Proposed the Indoor RFID Multi-variate Hidden Markov Model (IR-MHMM) to capture the uncertainties of indoor RFID data as well as the correlation of moving object locations and object readings.
\end{itemize}

\end{frame}

%------------------------------------------------

\begin{frame}
\frametitle{Motivation}

\begin{itemize}
  \item Recently there has been a remarkable proliferation of RFID in indoor tracking and monitoring systems
  \begin{fitemize}
    \item airport baggage monitoring. ~\cite{baba2013spatiotemporal,baba2013graph}
  \end{fitemize}
  \item The dirtiness of RFID data poses challenges to high-level RFID data querying and analysis.~\cite{chen2010leveraging}
  \begin{fitemize}
    \item \conceptbf{false negatives} (missing readings) occur when a reader fails to read a tag in its detection range.
    \item \conceptbf{false positives} (cross readings) occur when a tagged object is unexpectedly read by multiple readers simultaneously.
  \end{fitemize}
  \item This work focuses on cleansing \emph{false negatives} and \emph{false positives}.
\end{itemize}

\end{frame}

%------------------------------------------------

\begin{frame}
\frametitle{False Negatives \& False Positives}

\begin{columns}

  \column{0.4\textwidth}
  \begin{figure}[tb]
    \includegraphics[width=\columnwidth]{figures/3-5/3-5-1.pdf}
  \end{figure}
  \ssize{\textit{an object with tag $tag_1$ moved into the building and was first detected by reader $R_1$ from time point $t_0$ to time point $t_3$, yielding four observations by reader $R_1$.}}

  \column{0.6\textwidth}
  \begin{figure}[tb]
    \includegraphics[width=\columnwidth]{figures/3-5/3-5-2.pdf}
  \end{figure}
  \ssize{
    1. at $t_{12}$ and $t_{13}$, $tag_1$ was detected by readers $R_4$ and $R_9$, it seems to be present in both locations, thus giving rise to a \emph{false positives}.\\~\\
    2. after $t_{16}$, $tag_1$ was detected by reader $R_{17}$ that kept detecting $tag_1$ until $t_{29}$. However $tag_1$ is not supposed to be detected by $R_{17}$ before it's detected by $R_{10}$. $tag_1$ passed through $R_{10}$ but failed to generate any information, giving rise to \emph{false negatives}.
  }

\end{columns}

\end{frame}

%------------------------------------------------

\begin{frame}
\frametitle{Motivation}

To support high-level RFID business logic processing, it is necessary to perform data cleansing to remove false negatives and false positives.\\~\\

Existing RFID data cleansing techniques require considerable specific prior knowledge for cleansing operations.

\begin{sitemize}
  \item to cleanse historical indoor RFID data, the graph model based cleansing approaches~\cite{baba2013spatiotemporal,baba2013graph,DBLP:conf/edbt/FazzingaFFP14} rely on graphs that capture the indoor topology, the deployment of readers, and multiple pertinet spatial-temporal properties.
  \item in the context of streaming RFID data, a probabilistic approach~\cite{tran2009probabilistic} demands to build four domain-specific probabilistic models before any data cleansing.
  \item also streaming RFID data, ~\cite{nie2009probabilistic} assumes that locations of neighboring objects are correlated to each other, lifting such assumptions and needs only minimal prior knowledge about indoor settings.
\end{sitemize}

\end{frame}

%------------------------------------------------

\begin{frame}
\frametitle{Preliminaries}

\ssize{\conceptbf{Hidden Markov Models} (HMMs) models two connected (discrete time) stochastic processes: an un-observed state transition process, and an observation process consisting of observable signals generated at each state. The underlying (hidden) state transition process is assumed to be Markovian and stationary.}

\begin{definition}[Hidden Markov Models]
  \ssize{
  An HMM is a tuple $\lambda = (\mathcal{S}, \mathcal{O}, A, B, \pi)$.
  \begin{enumerate}
    \item $\mathcal{S} = \{ s_1, ..., s_N \}$ is a set of (hidden) states.
    \item $\mathcal{O} = \{ 0_1, ..., o_K \}$ is a set of possible observations.
    \item $A$ is an $N \times N$ transition probability matrix:$A = (a_{ij})_{i,j=1,...,N}$, where $a_{ij}$ represents the transition probability form hidden state $s_i$ to hidden $s_j$.
    \item $B$ is an $N \times K$ observation probability matrix: $B = (b_{ih})_{i=1,...,N;h=1,...K}$, where $b_{ih}$ is the probability of ovserving $o_h$ when the hidden process is in state $s_i$.
    \item $\pi$ is a $N$-dimensional initial state probability vector: $\pi = (\pi_i)_{i=1,...,N}$, where $\pi_i$ is the probability that the hidden state process starts in state $s_i$.
  \end{enumerate}
  }
\end{definition}

\end{frame}

%------------------------------------------------

\begin{frame}
\frametitle{Preliminaries}

An HMM defines a discrete time stochastic process over the combined state and observation space $\mathcal{S} \times \mathcal{O}$. \\~\\

The state of the process at time $t$ is described by \emph{random variables} $S^{(t)}$ with values in $\mathcal{S}$, and $\mathcal{O}^{(t)}$ is defined by

\pause
\begin{equation}
  P(S^{(0)} = s_i) = \pi_i
\end{equation}

\pause
\begin{equation}
  P(S^{(t+1)} = s_j | S^{(t)} = s_i) = a_{ij}
\end{equation}

\pause
\begin{equation}
  P(o^{(t)} = o_h | S^{(t)} = s_i) = b_{ih}
\end{equation}

\end{frame}

%------------------------------------------------

\begin{frame}
\frametitle{Preliminaries}

For modeling processes where at each point in time observations of multiple variables are made, the basic HMM model have been generalized to \conceptbf{Multi-variate Hidden Markov Models}(MH-MMs).~\cite{kirshner2005modeling}.\\~\\

The simple observation space $\mathcal{O}$ is replaced by a multivariate observation space $\mathcal{O}_1 \times \hdots \times \mathcal{O}_M$, and at each time $t$ one observes variables $O^{(t)}_1,...,O^{(t)}_M$.\\~\\

One can furthermore introduce the assumption that the different observations are \emph{independent} given the hidden state, i.e.,

\pause

\begin{equation}
  P(O^{(t)}_1,...,O^{(t)}_M | S^{(t)}) = \prod_{i=1}^M P(O^{(t)}_i | S^{(t)})
\end{equation}

\end{frame}

%------------------------------------------------

\begin{frame}
\frametitle{Data Transformation}

\textbf{Discretize the contunuous timestamp data} \quad for the duration $(t_e - t_s)$, a time granularity $\alpha$ is chosen to spans the discrete time points $0,...,(T-1)$, where $T = \frac{t_e - t_s}{\alpha}$.\\~\\

A continuous time record $\langle readerID, t \rangle$ is replaced by $\langle readerID, i \rangle$ where $i$ is such that $t \in [t_s + i * \alpha, t_s +(i+1) * \alpha]$.


\end{frame}

%------------------------------------------------

\begin{frame}
\frametitle{Data Transformation}

\textbf{Choice of $\alpha$} \quad
\begin{itemize}
  \item it should be chosen small enough so that discretization does not introduce any spurious appearances of cross readings, this can be ensured when $\alpha < \Delta_{min} / 2$, where $\Delta_{min}$ is the minimum absolute difference between continuous time-stamps of records with different $readID$s
  \item the time granularity must be small enough so that clean data of this granularity is sufficient for subsequent tracking and analysis purposes.
  \item under the above constraints, it should be as large as possible, to minimize the computational complexity.
\end{itemize}

\end{frame}

%------------------------------------------------

\begin{frame}
\frametitle{Data Transformation}

\begin{columns}

  \column{0.55\textwidth}
  \begin{figure}[tb]
    \includegraphics[width=0.7\columnwidth]{figures/3-5/3-5-3.pdf}
  \end{figure}
  \fsize{this figure shows an alternative representation of discretized data, time points and possible reader IDs are on the $x$ and $y$-axis respectively.}

  \column{0.45\textwidth}
  \begin{figure}[tb]
    \includegraphics[width=0.7\columnwidth]{figures/3-5/3-5-4.pdf}
  \end{figure}
  \fsize{this figure equivalently encodes left figure's data as a sequence of $T$ binary vectors of length $M$, where $M$ is the number of readers.}

\end{columns}

\end{frame}

%------------------------------------------------

\begin{frame}
\frametitle{Data Transformation}

After data transformation, the data consists of a sequence $\mathbf{V}_r^{(0)},...,\mathbf{V}_r^{(T-1)}$, where each $\mathbf{V}_r^{(t)} \in \{ 0,1 \}^{M}$. The subscript $r$ stands for \emph{raw}, the $k$th component of $\mathbf{V}_r^{(t)}$ is denoted as $\mathbf{V}_{r,k}^{(t)}$. \\~\\

Consider the $\mathbf{V}_r^{(t)}$ as the observations in a MHMM process, the \conceptbf{task} is to use standard HMM learning and inference techniques to transform the $\mathbf{V}_r^{(t)}$ into cleaned version $\mathbf{V}_c^{(t)}$. \\~\\

The cleaned data still takes values in  $\{ 0,1 \}^{M}$, but with the condition that each $\mathbf{V}_c^{(t)}$ contains at most one non-zero entry. \\~\\

\end{frame}
