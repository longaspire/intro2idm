%%%%%%%%%%%%%%%%%%%%%%%%%%%%%%%%%%%%%%%%%
% Beamer Presentation
% LaTeX Template
% Version 1.0 (10/11/12)
%
% This template has been downloaded from:
% http://www.LaTeXTemplates.com
%
% License:
% CC BY-NC-SA 3.0 (http://creativecommons.org/licenses/by-nc-sa/3.0/)
%
%%%%%%%%%%%%%%%%%%%%%%%%%%%%%%%%%%%%%%%%%

%----------------------------------------------------------------------------------------
%	PACKAGES AND THEMES
%----------------------------------------------------------------------------------------

\documentclass{beamer}

\mode<presentation> {

% The Beamer class comes with a number of default slide themes
% which change the colors and layouts of slides. Below this is a list
% of all the themes, uncomment each in turn to see what they look like.

%\usetheme{default}
%\usetheme{AnnArbor}
%\usetheme{Antibes}
%\usetheme{Bergen}
%\usetheme{Berkeley}
%\usetheme{Berlin}
%\usetheme{Boadilla}
%\usetheme{CambridgeUS}
%\usetheme{Copenhagen}
\usetheme{Darmstadt}
%\usetheme{Dresden}
%\usetheme{Frankfurt}
%\usetheme{Goettingen}
%\usetheme{Hannover}
%\usetheme{Ilmenau}
%\usetheme{JuanLesPins}
%\usetheme{Luebeck}
%\usetheme{Madrid}
%\usetheme{Malmoe}
%\usetheme{Marburg}
%\usetheme{Montpellier}
%\usetheme{PaloAlto}
%\usetheme{Pittsburgh}
%\usetheme{Rochester}
%\usetheme{Singapore}
%\usetheme{Szeged}
%\usetheme{Warsaw}

% As well as themes, the Beamer class has a number of color themes
% for any slide theme. Uncomment each of these in turn to see how it
% changes the colors of your current slide theme.

%\usecolortheme{albatross}
\usecolortheme{beaver} %great
%\usecolortheme{beetle}
%\usecolortheme{crane}
%\usecolortheme{dolphin}
%\usecolortheme{dove} %great
%\usecolortheme{fly}
%\usecolortheme{lily}
%\usecolortheme{orchid}
%\usecolortheme{rose}
%\usecolortheme{seagull}
%\usecolortheme{seahorse}
%\usecolortheme{whale}
%\usecolortheme{wolverine}

%\setbeamertemplate{footline} % To remove the footer line in all slides uncomment this line
\setbeamertemplate{footline}[page number] % To replace the footer line in all slides with a simple slide count uncomment this line
\setbeamertemplate{headline}{%
\leavevmode%
  \hbox{%
    \begin{beamercolorbox}[wd=\paperwidth,ht=2.5ex,dp=1.125ex]{palette secondary}%
    \insertsectionnavigationhorizontal{\paperwidth}{}{\hskip0pt plus1filll}
    \end{beamercolorbox}%
  }
  \hbox{%
    \begin{beamercolorbox}[wd=\paperwidth,ht=2.5ex,dp=1.125ex]{palette quaternary}%
      \hskip10pt \insertsubsection
    % \insertsubsectionnavigationhorizontal{\paperwidth}{}{\hskip0pt plus1filll}
    \end{beamercolorbox}%
  }
}
\setbeamertemplate{mini frames}{}
%\setbeamertemplate{navigation symbols}{} % To remove the navigation symbols from the bottom of all slides uncomment this line
%\beamertemplatenavigationsymbolsempty

}

\usepackage{graphicx} % Allows including images
\usepackage{booktabs} % Allows the use of \toprule, \midrule and \bottomrule in tables
\usepackage{bbm}
\usepackage{amssymb,amsmath}
\usepackage{soul}
\usepackage{sansmathaccent}
\usepackage{latexsym}
\pdfmapfile{+sansmathaccent.map}
\usepackage{hyperref}
\usepackage[export]{adjustbox}
\usepackage{bookmark}
\usefonttheme[onlymath]{serif}
\hypersetup{
bookmarksdepth=3,
bookmarksnumbered=true,
bookmarksopen=true,
colorlinks,
citecolor=blue,
linkcolor=red
}
\setbeamertemplate{bibliography item}{\insertbiblabel}
\setbeamertemplate{blocks}[rounded][shadow=false]

\newcommand{\conceptbf}[1]{{\bf{\textrm{\textcolor[rgb]{0,0,0.9}{#1}}}}}
\newcommand{\fsize}[1]{\footnotesize{#1}}
\newcommand{\ssize}[1]{\scriptsize{#1}}
\renewcommand{\emph}[1]{\textit{\textrm{#1}}}
\newenvironment{smitemize}{\begin{itemize}\small}{\end{itemize}}
\newenvironment{fitemize}{\begin{itemize}\footnotesize}{\end{itemize}}
\newenvironment{sitemize}{\begin{itemize}\scriptsize}{\end{itemize}}
\newenvironment{titemize}{\begin{itemize}\tiny}{\end{itemize}}

%----------------------------------------------------------------------------------------
%	TITLE PAGE
%----------------------------------------------------------------------------------------

\title[Manage the Data from Indoor Spaces]{Manage the Data from Indoor Spaces: \\Models, Indexes \& Query Processing} % The short title appears at the bottom of every slide, the full title is only on the title page

\author{Huan Li} % Your name
\institute[Zhejiang University] % Your institution as it will appear on the bottom of every slide, may be shorthand to save space
{
Database Laboratory, Zhejiang University \\ % Your institution for the title page
\medskip
\textit{lihuancs@zju.edu.cn} % Your email address
}
\date{\today} % Date, can be changed to a custom date

\begin{document}

\begin{frame}
\titlepage % Print the title page as the first slide
\end{frame}

\begin{frame}
\frametitle{Overview} % Table of contents slide, comment this block out to remove it
\setcounter{tocdepth}{1}
\tableofcontents % Throughout your presentation, if you choose to use \section{} and \subsection{} commands, these will automatically be printed on this slide as an overview of your presentation
\end{frame}

\AtBeginSection[]
{
    \begin{frame}[shrink]
        \tableofcontents[sectionstyle=show/shaded,subsectionstyle=show/shaded/hide]
    \end{frame}
}

%----------------------------------------------------------------------------------------
%	PRESENTATION SLIDES
%----------------------------------------------------------------------------------------

%------------------------------------------------
\section{1. Outlines} % Sections can be created in order to organize your presentation into discrete blocks, all sections and subsections are automatically printed in the table of contents as an overview of the talk
%------------------------------------------------

% \begin{frame}
\frametitle{Aims}
\begin{itemize}
\item To give a brief review introduction to \emph{indoor data management techniques}.
\\~\\
\item To review a series of works in this field, including their proposed \emph{models}, \emph{indexes} and \emph{algorithms}.
\\~\\
\item To discuss how to bring those advanced theoretical contents into practice.
\end{itemize}
\end{frame}

% \subsection{1.3 References} % A subsection can be created just before a set of slides with a common theme to further break down your presentation into chunks

%------------------------------------------------
%------------------------------------------------

\begin{frame}
\frametitle{References}

\begin{thebibliography}{99} % Beamer does not support BibTeX so references must be inserted manually as below
\fsize{

\bibitem{hightower2001location}
J.~Hightower, G.~Borriello.
\newblock Location systems for ubiquitous computing.
\newblock In {\em Journal of Computer}, pp. 7--66, 2001.

\bibitem{jensen2010indoor}
C.S.~Jensen, H.~Lu, B.~Yang.
\newblock Indoor-A New Data Management Frontier.
\newblock In {\em IEEE Data Eng. Bull.}, pp. 12--17, 2010.

\bibitem{becker2005location}
C.~Becker,  F.~D{\"u}rr.
\newblock On location models for ubiquitous computing.
\newblock In {\em Personal and Ubiquitous Computing}, pp. 20--31, 2005.

}
\end{thebibliography}

\end{frame}


%------------------------------------------------
\section{2. Indoor Space Models \& Applications} % Sections can be created in order to organize your presentation into discrete blocks, all sections and subsections are automatically printed in the table of contents as an overview of the talk
%------------------------------------------------

\subsection{2.1 Graph Model Based Indoor Tracking} % A subsection can be created just before a set of slides with a common theme to further break down your presentation into chunks

% \include{sec_2_1}
% \begin{frame}[allowframebreaks]
\frametitle{References}

\begin{thebibliography}{99} % Beamer does not support BibTeX so references must be inserted manually as below
\fsize{

\bibitem{DBLP:conf/mdm/JensenLY09}
C.S.~Jensen, H.~Lu, B.~Yang.
\newblock Graph model based indoor tracking.
\newblock In {\em {MDM}}, pp. 122--131, 2009.

\bibitem{civilis2005techniques}
A.~Civilis, C.S.~Jensen, S.~Pakalnis.
\newblock Techniques for efficient road-network-based tracking of moving objects.
\newblock In {\em {TKDE}}, pp. 698--712, 2005.

\bibitem{pfoser1999capturing}
D.~Pfoser, C.S.~Jensen.
\newblock Capturing the uncertainty of moving-object representations.
\newblock In {\em {Advances in Spatial Databases}}, pp. 111--131, 1999.

}
\end{thebibliography}

\end{frame}


\subsection{2.2 Scalable Continuous Range Monitoring of Moving Objects in Symbolic Indoor Space} % A subsection can be created just before a set of slides with a common theme to further break down your presentation into chunks

% \begin{frame}
\frametitle{About This Work...}

\emph{Scalable Continuous Range Monitoring of Moving Objects in Symbolic Indoor Space}.~\cite{DBLP:conf/cikm/YangLJ09} \\
B.~Yang, H.~Lu, and C.~S. Jensen.\\~\\

\begin{itemize}
  \item Published in \emph{CIKM' 2009}.
  \item Application: continuously monitor indoor moving objects for space use analysis or security purposes.
  \item An incremental, query-aware continuous range query processing technique for objects moving in indoor space.
  \item Use maximum speed constraint on object movement to refine the uncertain results.
\end{itemize}

\end{frame}
%------------------------------------------------

\begin{frame}
\frametitle{Motivation}

\begin{itemize}
  \item People spend much time in indoor spaces.

  \item Indoor spaces are becoming increasingly larger and complex.
    \begin{itemize}
      \item E.g., London Underground, 268 stations, 408 kilometers of network, +4 million daily passengers.
    \end{itemize}

  \item Indoor monitoring of people can help support.
    \begin{itemize}
      \item space use analysis
      \item security purposes
    \end{itemize}
\end{itemize}

\end{frame}

%------------------------------------------------

\begin{frame}
\frametitle{Preliminaries: Indoors vs. Outdoors}

\begin{itemize}
  \item Modeling of indoor spaces do not assume~\cite{jensen2010indoor}
    \begin{itemize}
      \item Euclidean space. (since obstacles render movement more constrained)
      \item Spatial network. (since indoor movement is less constrained than movements in polylines)
    \end{itemize}

  \item Instead indoor spaces are characterized by entities~\cite{DBLP:conf/mdm/JensenLY09}.
    \begin{itemize}
      \item Doors, rooms, hallways, staircase, etc.
    \end{itemize}

  \item \conceptbf{Symbolic models} are more suitable~\cite{becker2005location}.

  \item \emph{GPS} and \emph{cellular tracking} do not work indoors.

  \item Sensing devices are used to detect objects within their activation range, e.g., RFID readers or Bluetooth hotspots.
\end{itemize}

\end{frame}

%------------------------------------------------

\begin{frame}
\frametitle{Positioning Devices Deployment Graph}

\begin{columns}[c]

  \column{.57\textwidth}
  \begin{itemize}
    \fsize{
    %\item More advanced compared to \emph{RFID Deployment Graph}.
    \item Two types of positioning devices
      \begin{itemize}
        \ssize{
        \item Partitioning Device -- \emph{undirected} (\conceptbf{UP}), e.g., $d_{21}$ -- \emph{directed} (\conceptbf{DP}), e.g., $d_{11}$ and $d_{11'}$
        \item Presence Device -- (\conceptbf{PR})
        }
      \end{itemize}
    \item Note an indoor space is partitioned into \emph{activation ranges} and \emph{cells}
    }
  \end{itemize}
  \begin{block}{Deployment Graph}
    \textrm{
    \begin{itemize}
      \ssize{
      \item $G = \{C, E, \Sigma_{devices}, l_E\}$
      \item $C$: the set of cells
      \item $E$: the set of edges, $\{ c_i, c_j \}$ where $c_i, c_j \in C$
      \item $\Sigma_{devices}$: a mapping from $deviceID$ to activation range and type
      \item $l_E$ maps an edge to a set of positioning devices, i.e., $E \rightarrow 2^{\Sigma_{devices}}$
      }
    \end{itemize}
    }
  \end{block}

  \column{.43\textwidth}
    \vspace{-30pt}
    \begin{figure}[tb]
      \includegraphics[width=\columnwidth]{figures/2-2/2-2-1.pdf}
    \end{figure}
    \vspace{-20pt}
    \pause
    \begin{figure}[tb]
      \includegraphics[width=\columnwidth]{figures/2-2/2-2-2.pdf}
    \end{figure}

\end{columns}

\end{frame}

%------------------------------------------------

\begin{frame}
\frametitle{States of Indoor Moving Objects}

%\begin{columns}[c]

  %\column{.57\textwidth}
  \begin{figure}[tb]
    \includegraphics[width=0.6\columnwidth]{figures/2-2/2-2-3.pdf}
  \end{figure}
  \vspace{-10pt}
  %\column{.43\textwidth}
  \begin{itemize}
    \item An object is in an \conceptbf{active state} when it is inside the activation range of a positioning device.
    \item Otherwise the object is in an \conceptbf{inactive state}
    \item When an object is in the inactive state it is
      \begin{itemize}
        \item \conceptbf{nondeterministic} if it can be in more than one cell
        \item \conceptbf{deterministic} if it is in one specific cell
      \end{itemize}
  \end{itemize}

%\end{columns}

\end{frame}

%------------------------------------------------

\begin{frame}
\frametitle{Indexing Indoor Moving Objects}

\conceptbf{The proposed indexing scheme uses 4 hash tables}
\\~\\
\pause

\fsize{
\emph{Device Hash Table(DHT)} maps each device to a set of active objects:
\pause
$$DHT[deviceID] = O_A;~deviceID \in \Sigma_{devices}, O_A \subseteq O_{indoor}$$
\\~\\
\pause

\emph{Cell Deterministic Hash Table(CDHT)} maps each cell to a set of deterministic objects:
\pause
$$CDHT[cellID] = O_D;~cellID \in C, O_D \subseteq O_{indoor}$$
\\~\\
\pause

\emph{Cell Nondeterministic Hash Table(CNHT)} maps each cell to a set of nondeterministic objects:
\pause
$$CNHT[cellID] = O_N;~cellID \in C, O_N \subseteq O_{indoor}$$
\\~\\
\pause

\emph{Object Hash Table(OHT)} maps objects to their current data(state, time, cell(s) the object can be in)
\pause
$$OHT[objectID] = (STATE, t, IDSet);~objectID \in O_{indoor}$$
\\~\\
\pause
}

\end{frame}

%------------------------------------------------

\begin{frame}
\frametitle{RFID Deployment Graph Construction}

\begin{columns}[c]

  \column{.47\textwidth}
    \begin{figure}[tb]
      \includegraphics[width=\columnwidth]{figures/2-2/2-2-4.pdf}
    \end{figure}

  \column{.53\textwidth}
  \ssize{
    \begin{enumerate}
      \pause
      \item Line 1: \textrm{reset $IDSet$} \pause
      \item Lines 2--12: \textrm{$O.flag$ is ENTER so check the object's previous state. Remove $O$ from the corresponding table according its previous state} \pause
      \item Lines 13--14: \textrm{add $O$ to table of active objects (DHT), and update $O$'s in the objects' table (OHT)} \pause
      \item Lines 16--17: \textrm{$O.flag$ is LEAVE so remove the object from DHT. Get the possible cells that $O$ can move to} \pause
      \item Lines 18--25: \textrm{if the device is undirected, set $O$ in OHT and add $O$ to CNHT for the cells in sSet, else apply the same to CDHT}
    \end{enumerate}
  }
  \end{columns}

\end{frame}

%------------------------------------------------

\begin{frame}
\frametitle{Continuous Range Monitoring: Query Definition}

\begin{itemize}
  \item A \emph{Continuous Range Monitoring Query} (CRMQ)
    \begin{itemize}
      \item takes an \conceptbf{indoor spatial range} $\bf R$ as parameter
      \item keeps reporting the objects when it is registered for a certain time frame $[t_s, t_e]$
    \end{itemize}
  \item The \conceptbf{query result} $\bf \mathcal{M}$ -- the set of moving objects in $\bf R$ - is maintained as follows:
    \begin{equation*}
      \forall t \in [t_s, t_e]: o \in CRMQ[R](\mathcal{M})  \Leftrightarrow o \in \mathcal{M} \wedge pos_{\mathcal{M}}(o, t) \in R
    \end{equation*}
    where $pos_\mathcal{M}$ is a function that can determine the position of object $o$ at time $t$
  \item Multiple monitoring queries may coexist
\end{itemize}

\end{frame}

%------------------------------------------------

\begin{frame}
\frametitle{Critical Devices}

\ssize{
For a \textrm{CRMQ} query, a \emph{critical device} is one from which a new observation can potentially change the query result (either certain or uncertain). Use a \emph{Device Query Hash Table} (DQHT) to record the relationships:
\vspace{-5pt}
\begin{equation*}
  DQHT[deviceID] = \{ (queryID, CLASS) \}
\end{equation*}
}

\begin{columns}[c]

  \column{.3\textwidth}
    \vspace{-20pt}
    \begin{figure}[tb]
      \includegraphics[width=\columnwidth]{figures/2-2/2-2-5.pdf}
    \end{figure}

  \column{.8\textwidth}
    \vspace{-15pt}
    \begin{itemize}
      \ssize{
      \pause
      \item $\mathrm{CLASS1}$ -- \textrm{Device is fully covered in $R$ along with cells, e.g., $(device_{16}, query_2)$}
      \pause
      \item $\mathrm{CLASS2}$ -- \textrm{Device is fully covered but corresponding cells are not,  e.g., $(device_{13}, query_1)$}
      \pause
      \item $\mathrm{CLASS3}$ -- \textrm{Device intersects with the query range $R$,  e.g., $(device_{16}, query_1)$}
      \pause
      \item $\mathrm{CLASS4}$ -- \textrm{Device is disjoint from $R$ and at least one of its corresponding cells in $C_{ic} = \{ c|c \sqcap R \neq \varnothing \}$,  e.g., $(device_{1}, query_1)$}
      \pause
      \item $\mathrm{CLASS5}$ -- \textrm{Device is disjoint from $R$ and at least one of its corresponding cells in $C_{ex} = \{ c| \{ c, c'\} \in G.E, c' \in C_{ic} \}$, but none of them are in $C_{ic}$, e.g., $(device_{10}, query_2)$}
      }
    \end{itemize}

\end{columns}

\end{frame}

%------------------------------------------------

\begin{frame}
\frametitle{Query Registration}

\begin{itemize}
  \item To handle concurrent \textrm{CRMQ}s, a \emph{Query Hash Table} is created hold the results
    \begin{itemize}
      \item $QHT[queryID] = (CR, UR);~ CR \subseteq O_{indoor}, UR \subseteq O_{indoor}$
      \item where $CR$ is the certain result and $UR$ is the uncertain result
    \end{itemize}
  \item Overview
    \begin{figure}[tb]
      \includegraphics[width=0.68\columnwidth]{figures/2-2/2-2-6.pdf}
    \end{figure}
\end{itemize}

\end{frame}

%------------------------------------------------

\begin{frame}
\frametitle{Query Registration Algorithm (I)}

\begin{columns}[c]

\column{.48\textwidth}
  \begin{figure}[tb]
    \includegraphics[width=\columnwidth]{figures/2-2/2-2-7.pdf}
  \end{figure}

\column{.6\textwidth}
\ssize{
  \begin{enumerate}
    \pause
    \item Lines 1--9: \textrm{Initialization} \pause
    \item Lines 10--14: \textrm{Add possible devices to CriticalDeviceList $cd$ ($\mathrm{CLASS1}$ and $\mathrm{CLASS2}$)} \pause
    \item Lines 15--16: \textrm{Add possible $\mathrm{CLASS3}$ devices} \pause
    \item Lines 17--20: \textrm{Add possible $\mathrm{CLASS4}$ devices} \pause
    \item Line 21: \textrm{Determine extended cell set $C_{ex}$} \pause
    \item Lines 22--25: \textrm{Add possible $\mathrm{CLASS5}$ devices}
  \end{enumerate}
}
\end{columns}

\end{frame}

%------------------------------------------------

\begin{frame}
\frametitle{Query Registration Algorithm (II)}

\begin{columns}[c]

\column{.48\textwidth}
  \begin{figure}[tb]
    \includegraphics[width=\columnwidth]{figures/2-2/2-2-8.pdf}
  \end{figure}

\column{.6\textwidth}
\vspace{-10pt}
\ssize{
  \begin{enumerate}
    \pause
    \item Lines 26--27: \textrm{Add active objects from DHT to the certain result} \pause
    \item Lines 28--29: \textrm{Intersected device set, add active objects from DHT to the uncertain result} \pause
    \item Lines 30--31: \textrm{From covered cells, add deterministic objects to the certain result} \pause
    \item Lines 32-37: \textrm{If more than one cell, check nondeterministic objects. If all its possible cells are in $C_c$ add the object to the certain result, else uncertain result} \pause
    \item Lines 38--39: \textrm{Only one cell. Nondeterministic objects are added to the uncertain result} \pause
    \item Lines 40--41: \textrm{Intersected set. Add all objects to the uncertain result} \pause
    \item Line 42: \textrm{Results added to QHT} \pause
    \item Lines 43--44: \textrm{DQHT entry is created for each critical device}
  \end{enumerate}
}
\end{columns}

\end{frame}

%------------------------------------------------

\begin{frame}
\frametitle{Query Result Updates}

\begin{itemize}
  \fsize{
  \item When an object enters the activation range of a critical device:
    \begin{itemize}
      \ssize{
      \item For $\mathrm{CLASS1}$ or $\mathrm{CLASS2}$ devices the object is the certain result
      \item For $\mathrm{CLASS3}$ devices the object is possibly in the query range
      \item For $\mathrm{CLASS4}$ or $\mathrm{CLASS5}$ devices the object is not in the query range
      }
    \end{itemize}
  \item When an object leaves:
    \begin{itemize}
      \ssize{
      \item For $\mathrm{CLASS1,3,5}$ devices there are no changes
      \item For $\mathrm{CLASS2}$ devices the object may still be in the query range, thus it is moved to the uncertain result
      \item For $\mathrm{CLASS4}$ devices the object may be in a cell that intersects with the query range and it added to the uncertain result
      }
    \end{itemize}
  }
\end{itemize}

\begin{figure}[tb]
  \includegraphics[width=0.75\columnwidth]{figures/2-2/2-2-9.pdf}
\end{figure}

\end{frame}

%------------------------------------------------

\begin{frame}
\frametitle{Deferred Query Updates}

\begin{itemize}
	\item Deferred query updates is the concept of postponing updates where we already know the result
  \item The time a query result is still valid is calculated from \emph{minimum indoor walking distance} divided by the \emph{maximum speed} an object can travel
\end{itemize}

\begin{columns}[c]

\column{.3\textwidth}
  \begin{figure}[tb]
    \includegraphics[width=\columnwidth]{figures/2-2/2-2-5.pdf}
  \end{figure}

\column{.7\textwidth}
  \fsize{
  Consider $query_1$, after the object $o$ leaves a $\mathrm{CLASS2}$ critical device $devoce_{13}$, it should be moved from certain to uncertain result.
  \\~\\
  Let $V_{max}$ be the maximum speed, if $R_1 = V_{max} * \Delta t$, the certain result can be maintained without updating for an extra period of time $\Delta t$.
  }
\end{columns}

\end{frame}

%------------------------------------------------

\begin{frame}
\frametitle{Deferred Query Updates}

\begin{itemize}
	\item Deferred query updates is the concept of postponing updates where we already know the result
  \item The time a query result is still valid is calculated from \emph{minimum indoor walking distance} divided by the \emph{maximum speed} an object can travel
\end{itemize}

\begin{columns}[c]

\column{.3\textwidth}
  \begin{figure}[tb]
    \includegraphics[width=\columnwidth]{figures/2-2/2-2-5.pdf}
  \end{figure}

\column{.7\textwidth}
  \fsize{
  Consider $query_1$, after the object $o$ leaves a $\mathrm{CLASS2}$ critical device $devoce_{13}$, it should be moved from certain to uncertain result.
  \\~\\
  Let $V_{max}$ be the maximum speed, if $R_1 = V_{max} * \Delta t$, the certain result can be maintained without updating for an extra period of time $\Delta t$.
  }
\end{columns}

\end{frame}

%------------------------------------------------

\begin{frame}
\frametitle{Probabilistic Analysis of Uncertain Results}

\fsize{\textrm{
To analyze probability that $o$ is in the query range $R$. Assume that the possible locations in a given indoor space conform a uniform distribution within all reachable regions constrained by its maximum speed.
}}
\\~\\
\textit{1. Probabilities for Active Objects}
\\~\\
Formally, the probability that an active object $o$ is in the range $R$ is defined as:

\begin{equation}
  prob(o \Theta R) = \frac{Area(Devices(d).ActRange \sqcap R)}{Area(Devices(d).ActRange)}
\end{equation}

\begin{columns}[c]

\column{.3\textwidth}
  \vspace{-15pt}
  \begin{figure}[tb]
    \includegraphics[width=\columnwidth]{figures/2-2/2-2-5.pdf}
  \end{figure}

\column{.7\textwidth}
  Consider $device_{16}$, a $\mathrm{CLASS3}$ device for $query_1$, the probability for an active object in $device_{16}$ to be in the query range is calculated as ...

\end{columns}

\end{frame}

%------------------------------------------------

\begin{frame}
\frametitle{Probabilistic Analysis of Uncertain Results}

\textit{2. Probabilities for Inactive Objects}
\\~\\
For the case that after leaving $\mathrm{CLASS2,3,4}$ devices, the probabilities for inactive objects can be defined based on the maximum speed constraint.
\\~\\
An example for an inactive object that just leaves $device_{12}$...
\vspace{-5pt}
\begin{figure}[tb]
  \includegraphics[width=0.7\columnwidth]{figures/2-2/2-2-10.pdf}
\end{figure}

\end{frame}

%------------------------------------------------

\begin{frame}
\frametitle{Conclusion}

\begin{itemize}
	\item A solution with a symbolic model of the floor plan, device locations, and activation ranges
  \item Data is stored in several hash tables which make it possible to efficiently locate a specific object (result is a signle room/cell, or a set of rooms/cells)
  \item Future work
    \begin{itemize}
      \item sharing of query processing among concurrent queries
      \item common critical devices exploitation
      \item other types of queries: range and $k$NN
      \item further investigate the probability analysis
    \end{itemize}
\end{itemize}

\end{frame}

% \begin{frame}[allowframebreaks]
\frametitle{References}

\begin{thebibliography}{99} % Beamer does not support BibTeX so references must be inserted manually as below
\fsize{

\bibitem{DBLP:conf/mdm/JensenLY09}
C.~S. Jensen, H.~Lu, and B.~Yang.
\newblock Graph model based indoor tracking.
\newblock In {\em {MDM}}, pp. 122--131, 2009.

\bibitem{DBLP:conf/cikm/YangLJ09}
B.~Yang, H.~Lu, and C.~S. Jensen.
\newblock Scalable continuous range monitoring of moving objects in symbolic
  indoor space.
\newblock In {\em CIKM}, pp. 671--680, 2009.

\bibitem{becker2005location}
C.~Becker and F.~D{\"u}rr.
\newblock On location models for ubiquitous computing.
\newblock In {\em Personal and Ubiquitous Computing}, pp. 20--31, 2005.

\bibitem{jensen2010indoor}
C.~S. Jensen, H.~Lu, B.~Yang.
\newblock Indoor-A New Data Management Frontier.
\newblock In {\em IEEE Data Eng. Bull.}, pp. 12--17, 2010.

}
\end{thebibliography}

\end{frame}


\subsection{2.3 Probabilistic Threshold k Nearest Neighbor Queries over Moving Objects in Symbolic Indoor Space} % A subsection can be created just before a set of slides with a common theme to further break down your presentation into chunks

% \begin{frame}
\frametitle{About This Work...}

\emph{Probabilistic Threshold $k$ Nearest Neighbor Queries over Moving Objects in Symbolic Indoor Space}.~\cite{DBLP:conf/edbt/YangLJ10} \\
B.~Yang, H.~Lu, and C.~S. Jensen.\\~\\

\begin{itemize}
  \item Published in year 2010 at the \emph{EDBT} conference.
  \item \emph{Minimal Indoor Walking Distance}(MIWD) along with algorithms and data structures are proposed for distance computing and storage.
  \item Effective object indexing structures, also capture the uncertainty of object locations.
  \item On this foundation, Probabilistic threshold $k$NN (PT$k$NN) query is studied.
\end{itemize}

\end{frame}
%------------------------------------------------

\begin{frame}
\frametitle{Motivation}

\begin{itemize}
  \item Indoor positioning makes it possible to support interesting queries over large populations of moving objects.
    \begin{itemize}
      \item shopping mall, airports, office buildings
      \item $k$NN queries over indoor moving objects enables the detection of approaching potential threats at sensitive locations in a subway system
    \end{itemize}
  \\~\\
  \item Existing $k$NN techniques in spatial and spatialtemporal databases are inapplicable in indoor spaces.
    \begin{itemize}
      \item complex entities and topologies
      \item indoor positioning techniques differ fundamentally from outdoor GPS, low sampling frequency and accuracy
    \end{itemize}
\end{itemize}

\end{frame}
%------------------------------------------------

% \begin{frame}[allowframebreaks]
\frametitle{References}

\begin{thebibliography}{99} % Beamer does not support BibTeX so references must be inserted manually as below
\fsize{

\bibitem{DBLP:conf/mdm/JensenLY09}
C.S.~Jensen, H.~Lu, B.~Yang.
\newblock Graph model based indoor tracking.
\newblock In {\em {MDM}}, pp. 122--131, 2009.

\bibitem{DBLP:conf/cikm/YangLJ09}
B.~Yang, H.~Lu, C.S.~Jensen.
\newblock Scalable continuous range monitoring of moving objects in symbolic
  indoor space.
\newblock In {\em CIKM}, pp. 671--680, 2009.

\bibitem{DBLP:conf/edbt/YangLJ10}
B.~Yang, H.~Lu, C.S.~Jensen.
\newblock Probabilistic threshold k nearest neighbor queries over moving
  objects in symbolic indoor space.
\newblock In {\em EDBT}, pp. 335--346, 2010.

\bibitem{jensen2010indoor}
C.S.~Jensen, H.~Lu and B.~Yang.
\newblock Indoor-A New Data Management Frontier.
\newblock In {\em IEEE Data Eng. Bull.}, pp. 12--17, 2010.

\bibitem{cheng2004querying}
R.~Cheng, D.V.~Kalashnikov, S.~Prabhakar.
\newblock Querying imprecise data in moving object environments.
\newblock In {\em TKDE}, pp. 1112--1127, 2004.

\bibitem{cheng2009evaluating}
R.~Cheng, L.~Chen, J.~Chen, X.~Xie.
\newblock Evaluating probability threshold k-nearest-neighbor queries over uncertain data.
\newblock In {\em EDBT}, pp. 672--683, 2009.

}
\end{thebibliography}

\end{frame}


\subsection{2.4 Spatio-temporal Joins on Symbolic Indoor Tracking Data} % A subsection can be created just before a set of slides with a common theme to further break down your presentation into chunks

% \begin{frame}
\frametitle{About This Work...}

\emph{Spatio-temporal Joins on Symbolic Indoor Tracking Data}.~\cite{DBLP:conf/icde/LuYCJ11} \\
H.~Lu, B.~Yang, and C.~S. Jensen.\\~\\

\begin{itemize}
  \item Published at \emph{ICDE' 2011}.
  \item Studies the probabilistic, spatio-temporal joins on hisorical indoor tracking data.
  \item Two-phase hash-based algorithms are proposed for the point and interval joins.
  \item A filter-and-refine framework, along with spatial indexes and pruning rules.
\end{itemize}

\end{frame}
%------------------------------------------------

\begin{frame}
\frametitle{Motivation}

\begin{itemize}
  \item Huge amount of tracking data serves as a foundation for a wide variety of indoor applications and services.\cite{jensen2010indoor}
    \begin{fitemize}
      \item shopping mall, airports, office buildings, akin to those enabled by outdoor GPS
      \item hot area detection, space planning, security control, movement pattern discovery
    \end{fitemize}

  \item Spatio-temporal joins fall short in indoor setting.
    \begin{fitemize}
      \item indoor space consists of semantic entities enable or constrain movement
      \item semantics of indoor space call for novel spatio-temporal join predicates
      \item indoor positioning technologies differ fundamentally from outdoor setting, low accuracy and sampling frequency
    \end{fitemize}

  \item Joins on indoor tracking data call for new definition and new implementation techniques that take into account:
  \begin{fitemize}
    \item specifics of indoor space
    \item limitations of indoor positioning
  \end{fitemize}
\end{itemize}

\end{frame}
%------------------------------------------------

%------------------------------------------------

\begin{frame}
\frametitle{Preliminaries: Symbolic Indoor Tracking}

\begin{figure}[tb]
  \includegraphics[width=\columnwidth]{figures/2-4/2-4-1.pdf}
\end{figure}

\begin{enumerate}
  \ssize{
  \item $\mn{C2P: C \rightarrow 2^P}$ maps a cell to a set of indoor partitions
  \item $\mn{D2C: D \rightarrow 2^C}$ maps a device to a set of corresponding cells
  \item According to Deployment Graph, for partitioning device, $\mn{D2C(device_{13})} = \{ C_{10},C_{13} \} \cup \{ C_{12},C_{13} \} = \{ C_{10},C_{12},C_{13} \}$
  \item For presence device, $\mn{D2C(device_{25})} = \{ C_{21},C_{22} \}$ as the cells intersect its detection range.
  \item $\mn{D2C: D \rightarrow 2^C}$ is useful as it captures the possible movements of objects.
  }
\end{enumerate}

\end{frame}

%------------------------------------------------

\begin{frame}
\frametitle{Preliminaries: Symbolic Indoor Tracking}

\begin{columns}[c]
  \column{.52\textwidth}
  \begin{figure}[tb]
    \includegraphics[width=\columnwidth]{figures/2-4/2-4-2.pdf}
  \end{figure}

  \column{.48\textwidth}
  \begin{fitemize}
    \item \conceptbf{Object Tracking Table} $\mn{OTT}$ records the converted trajectories with schema $\mn{(ID, objectID, deviceID, t_s, t_e)}$
    ~\\
    \item a record states that the object $\mn{objectID}$ is observed by the device $\mn{deviceID}$ in the closed interval from time $\mn{t_s}$ to $\mn{t_e}$.
  \end{fitemize}

\end{columns}

\end{frame}

%------------------------------------------------

\begin{frame}
\frametitle{Problem Definitions}

Given an $\mn{OTT}$, it is of interesting to identify object pairs that join w.r.t some specific spatio-temporal join predicate.
\begin{fitemize}
  \item to know all pair of individuals that were probably at the same gate when a particular event (terrorist attack) occurred in a large airport.
\end{fitemize}
~\\
Due to tracking uncertainty, only interested in those objects that satisfy the join predicate with some given probability (specified threshold).
\\~\\
The joins are effectively \emph{self-joins} because all tracking data is contained in a single $\mn{OTT}$.

\end{frame}

%------------------------------------------------

\begin{frame}
\frametitle{Problem Definition I}

\textrm{One can apply a join predicate to a time point to find pairs that join at that particular time point...}
\\~\\
\begin{definition}[\ssize{Probabilistic Threshold Indoor Spatio-temporal Join--PTISSJ}]
  \textrm{
  \ssize{
  Given an $\mn{OTT}$, a join predicate $\mn{P}$, a time point $\mn{t}$, and a threshold value $\mn{M \in (0,1]}$, a probabilistic threshold indoor spatio-temporal join  $\mn{\Join_{P,t,M}(OTT) = \{ (o_i, o_j) | o_i, o_j \in O  \wedge o_i \neq o_j \wedge pr(P(o_i, o_j, t)) >M \}}$, where $\mn{pr(P(o_i,o_j,t))}$ is the \textbf{Timeslice Join Probability} of $\mn{o_i, o_j}$ at time $\mn{t}$, i.e., the probability that predicate $\mn{P(o_i,o_j,t)}$ is true.
  }}
\end{definition}

\end{frame}

%------------------------------------------------

\begin{frame}
\frametitle{Problem Definition II}

\textrm{It's also interesting to know object pairs satisfy the predicate for some consecutive timestamp...}
\\~\\
\begin{definition}[\ssize{Probabilistic Threshold $k$ Indoor Spatio-temporal Join--PT$k$ISSJ}]
  \textrm{
  \ssize{
  Given an $\mn{OTT}$, a join predicate $\mn{P}$, a time interval $\mn{I = [t_m, t_n] (m < n)}$, an integer $\mn{k(0 < k \leq n - m)}$, and a threshold value $\mn{M \in (0,1]}$, a probabilistic $\mn{k}$ threshold indoor spatio-temporal join
  \begin{equation*}
    \begin{split}
    &\mn{\Join_{P,I,k,M}(OTT) = \{ (o_i, o_j) | o_i, o_j \in O \wedge o_i \neq o_j \wedge } \\
    &\mn{\exists s \in m...n -k + 1 (\forall\delta \in 0...k-1 (pr(P(o_i, o_j, t_{s+\delta})) > M)) \} }
    \end{split}
  \end{equation*}
  }}
\end{definition}

\end{frame}

%------------------------------------------------

\begin{frame}
\frametitle{Uncertainty Model for Indoor Tracking}

\textrm{For outdoor moving objects~\cite{cheng2004querying}, \conceptbf{Uncertainty Region}, denoted by $\mn{UR(o_i,t)}$, is a region such that $\mn{o_i}$ must be in this region at time $\mn{t}$.}\\~

In general terms, an object $\mn{o_i}$'s location can be modeled as a random variable $\mn{l}$ associated with a probability density function $\mn{f_{o_i}(l,t)}$ that has non-zero values only in $\mn{o_i}$'s suncertainty region $\mn{UR(o_i, t)}$.~\cite{DBLP:conf/edbt/YangLJ10}

\begin{equation}
  \mn{\int_{l \in UR(o_i,t)} f_{o_i}(l,t) dl = 1}
\end{equation}

\end{frame}

%------------------------------------------------

\begin{frame}
\frametitle{Object State in OTT}

\begin{columns}[c]

\column{.45\textwidth}
\begin{figure}[tb]
  \includegraphics[width=\columnwidth]{figures/2-4/2-4-3.pdf}
\end{figure}

\column{.55\textwidth}
\begin{definition}[Active State]
  \textrm{
  \ssize{
  Given an object $\mn{o_i}$ and a time point $\mn{t}$, if a tracking record $\mn{rd_{cov}}$ is found in $\mn{OTT}$ such that $\mn{rd_{cov}.objectID = o_i}$ and $\mn{t \in [rd_{cov}.t_s, rd_{cov}.t_e]}$, $\mn{o_i}$ is in the \conceptbf{active state} at time $\mn{t}$.
  }}
\end{definition}

\end{columns}

\begin{definition}[Inactive State]
  \textrm{
  \ssize{
  Given an object $\mn{o_i}$ and a time point $\mn{t}$, if no record $\mn{rd_{cov}}$ is found in $\mn{OTT}$, $\mn{o_i}$ is in the \conceptbf{inactive state} at time $\mn{t}$. Instead, two tracking records of $\mn{o_i}$ called $\mn{rd_{pre}}$ and $\mn{rd_{suc}}$, can be found in $\mn{OTT}$, such that they are consecutive in the sense that $\mn{rd_{pre}.t_e < t < rd_{suc}.t_s}$ and there is no record for $\mn{o_i}$ between times $\mn{rd_{pre}.t_e}$ and $\mn{rd_{suc}.t_s}$.
  }}
\end{definition}

\end{frame}

%------------------------------------------------

\begin{frame}
\frametitle{Uncertainty Region in the Active State}

\begin{columns}[c]
  \column{.52\textwidth}
  \vspace{-10pt}
  \begin{figure}[tb]
    \includegraphics[width=\columnwidth]{figures/2-4/2-4-2.pdf}
  \end{figure}
  \vspace{-15pt}
  \begin{example}
    \textrm{
    \ssize{
    $\mn{t = t_5}$, $\mn{rd_{cov} = rd_3}$ and $\mn{rd_{pre} = rd_1}$, which tells $\mn{o_i}$ left $\mn{dev_4}$'s detection range at time $\mn{t_2}$, and is currently detected by $\mn{dev_2}$.
    }}
  \end{example}

  \column{.48\textwidth}
  \vspace{-10pt}
  \begin{figure}[tb]
    \includegraphics[width=\columnwidth]{figures/2-4/2-4-4.pdf}
  \end{figure}
  \ssize{
  \textbf{Step 1:}
  UR is the detection range of device $\mn{rd_{cov}.deviceID}$, denote as:
  \begin{equation*}
  \begin{split}
    \mn{ C_{cov} = } &\mn { Cir(Loc(rd_{cov}.deviceID), }\\
    &\mn{ Rad(rd_{cov}.deviceID)) }
  \end{split}
  \end{equation*}
  ~\\
  \textbf{Step 2:}
  UR should consider the $\mn{rd_{pre}}$'s \emph{maximum speed bounding ring}(MSBR):
  \tiny{
  \begin{equation*}
  \begin{split}
    &\mn{ UR(o_i, t) = C_{cov} \cap Ring(Loc(rd_{pre}.deviceID), } \\
    &\mn{ Rad(rd_{pre}.deviceID), V_i \cdot (t - rd_{pre}.t_e) ) }
  \end{split}
  \end{equation*}
  }
  }
\end{columns}

\end{frame}

%------------------------------------------------

\begin{frame}
\frametitle{Uncertainty Region in the Inactive State}

\begin{columns}[c]

  \column{.52\textwidth}
  \vspace{-10pt}
  \begin{figure}[tb]
    \includegraphics[width=\columnwidth]{figures/2-4/2-4-2.pdf}
  \end{figure}
  \vspace{-15pt}
  \begin{example}
    \textrm{
    \ssize{
    $\mn{t = t_{19}}$, $\mn{rd_{pre} = rd_6}$ and $\mn{rd_{suc} = rd_8}$, since $\mn{rd_6.t_e = t_{16} < t_{19} < rd_8.t_s = t_{21}}$. we have $\mn{dev_p = dev_{12}}$ and $\mn{dev_s = dev_{13}}$
    }}
  \end{example}

  \column{.48\textwidth}
  \vspace{-10pt}
  \begin{figure}[tb]
    \includegraphics[width=\columnwidth]{figures/2-4/2-4-5.pdf}
  \end{figure}
  \ssize{
  \textbf{Step 1:}
  Determine the possible cells in which the object can be in the inactive period:
  \begin{equation*}
    \mn{ Cells_{mid} = D2C(dev_p) \cup D2C(dev_s)}
  \end{equation*}
  ~\\
  \textbf{Step 2:}
  UR is constrained by two \emph{maximum speed bounding ring}(MSBR)s of $\mn{rd_{pre}}$ and $\mn{rd_{suc}}$:
  \tiny{
  \begin{equation*}
  \begin{split}
    \mn{ UR(o_i, t) = \bigcup_{c \in Cells_{mid}} c \cap R_{pre} \cap R_{suc} }
  \end{split}
  \end{equation*}
  }
  }
\end{columns}

\end{frame}

%------------------------------------------------

\begin{frame}
\frametitle{Join Probability Evaluation}

\begin{definition}[the \emph{same X} predicate]
  \textrm{
  \ssize{
    termed as $\mn{P_X}$, where $\mn{X}$ represents an indoor region type. $\mn{IR_X}$ represents all $\mn{X}$ type regions ($\mn{X}$-regions).
  }
  }
\end{definition}
~\\
\begin{example}[the \emph{same room} predicate]
  \textrm{
  \ssize{
    Given two objects $\mn{o_i, o_j}$ at a time point $\mn{t}$, the \emph{same room} predicate $\mn{P_X(o_i, o_j, t)}$ evaluates to true if both $\mn{o_i, o_j}$ were located in a same room $\mn{rm \in IR_X}$. Other predicates can be \emph{same floor, same reserach group (maps to several rooms)}.
  }
  }
\end{example}
~\\
The \emph{same X} predicates are more practical than Euclidean distance based join predicates in indoor space.

\end{frame}

%------------------------------------------------

\begin{frame}
\frametitle{Join Probability Evaluation}
\vspace{-10pt}
\begin{definition}[``be located at'' predicate probability]
  \textrm{
  \ssize{
    Given an object $\mn{o_i}$, an $\mn{X}$-region $\mn{x_l \in IR_X}$, and a time $\mn{t}$, predicate $\Theta(o_i, x_l, t)$ indicate that $\mn{o_i}$ was located in $\mn{x_l}$ at $\mn{t}$. The probability that $\mn{\Theta}$ is satisfied is defined as:
    \begin{equation*}
      \mn{pr(\Theta(o_i, x_l, t)) = \frac{Area(UR(o_i, t) \cap x_l)}{Area(UR(o_i, t))}}
    \end{equation*}
  }
  }
\end{definition}

\begin{definition}[the \emph{same X} predicate probability]
  \textrm{
  \ssize{
    The probability that $\mn{o_i}$ and $\mn{o_j}$ were located in the same $\mn{x_l}$ at $\mn{t}$, indicated by $\mn{pr(P_{x_l}(o_i, o_j, t))}$ is defined as:
    \begin{equation*}
      \mn{pr(P_{x_l}(o_i, o_j, t)) = pr(\Theta(o_i, x_l, t)) \cdot pr(\Theta(o_j, x_l, t)) }
    \end{equation*}
    Therefore, the porbability that $\mn{o_i}$ and $\mn{o_j}$ satisfy a \emph{same X} predicate at time $\mn{t}$ can be defined as:
    \begin{equation*}
      \mn{pr(P_X(o_i, o_j, t)) = \max_{x_l \in IR_X} pr(P_{x_l}(o_i, o_j, t)) }
    \end{equation*}
  }
  }
\end{definition}

\end{frame}

%------------------------------------------------

\begin{frame}
\frametitle{Indexing the Indoor Tracking Data}

\textrm{to determine the \emph{Uncertainty Region} during join processing, it needs to retrieve the records $\mn{rd_{cov}}$ and $\mn{rd_{pre}}$ for active objects or $\mn{rd_{pre}}$ and $\mn{rd_{suc}}$ for inactive state.}
\\~\\
to index $\mn{OTT}$ with an augmented 1D R-tree, where each leaf entry has the form $\mn{(t^{\vdash},t^{\dashv},Ptr_{p},Ptr_{c})}$. $\mn{t^{\vdash} = rd_p.t_e}$, $\mn{t^{\dashv} = rd_c.t_e}$, $\mn{Ptr_p}$ and $\mn{Ptr_c}$ points to $\mn{rd_p}$ and $\mn{rd_c}$ respectively.

\begin{fitemize}
  \item if $\mn{t \geq rd_c.t_s}$, $\mn{o_i}$ is active, $\mn{rd_p \rightarrow rd_{pre}}$ and $\mn{rd_c \rightarrow rd_{cov}}$;
  \item if $\mn{t < rd_c.t_s}$, $\mn{o_i}$ is inactive, $\mn{rd_p \rightarrow rd_{pre}}$ and $\mn{rd_c \rightarrow rd_{suc}}$;
\end{fitemize}

\end{frame}

%------------------------------------------------

\begin{frame}
\frametitle{Accessing $\mn{X}$-Regions}

\textrm{object locations are bounded by either device detection ranges or cells.}

\begin{columns}[c]

  \column{.54\textwidth}
  \begin{figure}[tb]
    \includegraphics[width=\columnwidth]{figures/2-4/2-4-6.pdf}
  \end{figure}

  \column{0.48\textwidth}
  \begin{sitemize}
    \item $\mn{CovD2X: D \rightarrow IR_X}$ maps a device to an $\mn{X}$-Region that fully covers the device's detection range;
    \item $\mn{IntD2X: D \rightarrow IR_X}$ maps a device to an $\mn{X}$-Region that only intersects the device's detection range;
    \item $\mn{CovC2X: C \rightarrow IR_X}$ maps a cell to an $\mn{X}$-Region that fully covers this cell;
    \item $\mn{IntC2X: C \rightarrow IR_X}$ maps a cell to an $\mn{X}$-Region that only intersects with;
  \end{sitemize}

\end{columns}

\end{frame}

%------------------------------------------------

\begin{frame}
\frametitle{Processing PTISSJ Queries: Partitioning Phase}

\begin{columns}[c]

  \column{.47\textwidth}
  \vspace{-10pt}
  \begin{figure}[tb]
    \includegraphics[width=\columnwidth]{figures/2-4/2-4-7.pdf}
  \end{figure}

  \column{0.53\textwidth}
  \begin{fitemize}
    \item all indoor objects are partitioned into buckets that each refers to a distinct $\mn{X}$-region
    \item first, A1R-tree is searched to get all leaf entries whose interval $\mn{(t^{\vdash},t^{\dashv}]}$ contains the join time $\mn{t}$
    \item second, the spatial examination obtains all relevant $\mn{X}$-region in which $\mn{o_i}$ may be at time $\mn{t}$
    \item the relevant probabilities are evaluated for each object, and the necessary records are generated and added to relevant buckets, for each $\mn{p_l = pr(\Theta(o_i, x_l, t))}$, if it is larger than threshold $\mn{M}$, insert the record into buckets.
  \end{fitemize}

\end{columns}

\end{frame}

%------------------------------------------------

\begin{frame}
\frametitle{Processing PTISSJ Queries: Partitioning Phase}

\begin{columns}[c]

  \column{.5\textwidth}
  \textbf{Active State}~\\
  \ssize{
  \textrm{object $\mn{o_i}$ must be in device $\mn{dev}$'s detection range at time $\mn{t}$}.~\\
  \begin{enumerate}
    \item if the detection range is fully covered by an $\mn{X}$-region $\mn{x_l}$, as indicated by $\mn{CovD2X(dev_c) = x_l}$, a record $\mn{(o_i, 1.0)}$ is added to $\mn{x_l}$'s bucket;
    \item otherwise, $\mn{dev_c}$'s detection range intersects with each $\mn{X}$-region in $\mn{CovD2X(dev_c)}$, evaluated the probability, if it is larger than $\mn{M}$, add to the bucket.
  \end{enumerate}
  }

  \column{0.5\textwidth}
  \textbf{Inactive State}~\\
  \ssize{
  \textrm{object $\mn{o_i}$ must be in a cell in $\mn{Cells_{mid} = D2C(dev_p) \cap D2C(dev_c)}$}.~\\
  \begin{enumerate}
    \item if $\mn{Cells_{mid}}$ is the singleton set and the cell is covered by one $\mn{X}$-region $\mn{x_l}$, indicated by $\mn{CovC2X(c)= x_l}$, a record $\mn{(o_i, 1.0)}$ is added to $\mn{x_l}$'s bucket;
    \item otherwise, the single cell $\mn{c}$ in $\mn{Cells_{mid}}$ intersects with several $\mn{X}$-regions (indicated by $\mn{CovC2X(c)}$), or $\mn{Cells_{mid}}$ contains several cells.
  \end{enumerate}
  }

\end{columns}

\end{frame}

% \begin{frame}[allowframebreaks]
\frametitle{References}

\begin{thebibliography}{99} % Beamer does not support BibTeX so references must be inserted manually as below
\fsize{

\bibitem{DBLP:conf/mdm/JensenLY09}
C.S.~Jensen, H.~Lu, B.~Yang.
\newblock Graph model based indoor tracking.
\newblock In {\em {MDM}}, pp. 122--131, 2009.

\bibitem{DBLP:conf/cikm/YangLJ09}
B.~Yang, H.~Lu, C.S.~Jensen.
\newblock Scalable continuous range monitoring of moving objects in symbolic
  indoor space.
\newblock In {\em CIKM}, pp. 671--680, 2009.

\bibitem{DBLP:conf/edbt/YangLJ10}
B.~Yang, H.~Lu, C.S.~Jensen.
\newblock Probabilistic threshold k nearest neighbor queries over moving
  objects in symbolic indoor space.
\newblock In {\em EDBT}, pp. 335--346, 2010.

\bibitem{DBLP:conf/icde/LuYCJ11}
H.~Lu, B.~Yang, C.S.~Jensen.
\newblock Spatio-temporal Joins on Symbolic Indoor Tracking Data.
\newblock In {\em {ICDE}}, pp. 816--827, 2011.

\bibitem{jensen2010indoor}
C.S.~Jensen, H.~Lu, B.~Yang.
\newblock Indoor-A New Data Management Frontier.
\newblock In {\em IEEE Data Eng. Bull.}, pp. 12--17, 2010.

\bibitem{cheng2004querying}
R.~Cheng, D.V.~Kalashnikov, S.~Prabhakar.
\newblock Querying imprecise data in moving object environments.
\newblock In {\em TKDE}, pp. 1112--1127, 2004.

\bibitem{pfoser1999capturing}
D.~Pfoser, C.S.~Jensen.
\newblock Capturing the uncertainty of moving-object representations
\newblock In {\em Advances in Spatial Databases}, pp. 111--131, 1999.

}
\end{thebibliography}

\end{frame}


\subsection{2.5 A Foundation for Efficient Indoor Distance-aware Query Processing} % A subsection can be created just before a set of slides with a common theme to further break down your presentation into chunks

% \begin{frame}
\frametitle{About This Work...}

\emph{A Foundation for Efficient Indoor Distance-Aware Query Processing}.~\cite{DBLP:conf/icde/LuYCJ11} \\
H.~Lu, X.~Cao, and C.~S. Jensen.\\~\\

\begin{itemize}
  \item Published at \emph{ICDE' 2012}.
  \item First time to propose a distance-aware indoor space model that integrates indoor distance seamlessly.
  \item Accompanying, efficient algorithms for computing indoor distances.
  \item Indexing framework that accommodates indoor distances.
\end{itemize}

\end{frame}

%------------------------------------------------

\begin{frame}
\frametitle{Motivation}

\begin{itemize}
  \item A variety of LBS services are useful in indoor space.
    \begin{fitemize}
      \item a museum guidance service in a complex exhibition
      \item boarding reminder service in an airport, to remind the passengers especially those far away from their gates or departures
    \end{fitemize}

  \item Such indoor LBSs will benefit from the availability of accurate indoor distances.
    \begin{fitemize}
      \item indoor space entities enable as well as constrain indoor movement, thus makes traditional space model for Euclidean/spatial network spaces unsuitable.
      \item existing indoor space models~\cite{becker2005location, li2008lattice, becker2009multilayered} pay little attention to indoor distances.
    \end{fitemize}

\end{itemize}

\end{frame}

%------------------------------------------------

\begin{frame}
\frametitle{Indoor Topology Mapping Structures}

Mapping $D2P$ maps a door $d_k$ to one or two partition pairs~\footnote{\ssize{the basic assumption that a door corresponds to two doors can be extended by converting a door to multiple doors.}} $(v_i, v_j)$ such that one can move from partition~\footnote{\ssize{a partition indicates a room, a hallway or a staircase.}} $v_i$ to partition $v_j$ through door $d_k$:
\pause
\begin{equation}
 D2P: \mathcal{S}_{door} \rightarrow 2^{\mathcal{S}_{partition}} \times 2^{\mathcal{S}_{partition}}
\end{equation}
\pause
For \emph{enterable partition} of door $d_k$:
\pause
\begin{equation}
 D2P_{\sqsupset}: \mathcal{S}_{door} \rightarrow 2^{\mathcal{S}_{partition}}
\end{equation}
\pause
For \emph{leaveable partition} of door $d_k$:
\pause
\begin{equation}
 D2P_{\sqsubset}: \mathcal{S}_{door} \rightarrow 2^{\mathcal{S}_{partition}}
\end{equation}
\end{frame}

%------------------------------------------------

\begin{frame}
\frametitle{Indoor Topology Mapping Structures}


The mapping $P2D_{\sqsupset}$ maps a partition $v$ to all the doors through which one can enter $v$:
\pause
\begin{equation}
 P2D_{\sqsupset}: \mathcal{S}_{partition} \rightarrow 2^{\mathcal{S}_{door}}
\end{equation}
\pause
The mapping $P2D_{\sqsubset}$ maps a partition $v$ to all the doors through which one can leave $v$:
\pause
\begin{equation}
 P2D_{\sqsubset}: \mathcal{S}_{partition} \rightarrow 2^{\mathcal{S}_{door}}
\end{equation}
\pause
The mapping $P2D$ is used when there's no need to differentiate the directionality:
\pause
\begin{equation}
 P2D(v_i): P2D_{\sqsupset}(v_i) \cup P2D_{\sqsubset}(v_i)
\end{equation}
\end{frame}

%------------------------------------------------

\begin{frame}
\frametitle{Accessibility Base Graph}

\begin{columns}[c]

  \column{0.52\textwidth}
  \vspace{-15pt}
  \begin{figure}[tb]
    \includegraphics[width=0.85\columnwidth]{figures/2-5/2-5-1.pdf}
  \end{figure}
  \begin{example}
    \textrm{
    \ssize{
      $D2P_{\sqsupset}(d_{12}) = \{ v_{10}\}$, $D2P_{\sqsubset}(d_{12}) = \{ v_{12}\}$
      $P2D_{\sqsupset}(v_{13}) = \{ d_{13}\}$, $P2D_{\sqsubset}(v_{13}) = \{ d_{13}, d_{15}\}$
    }
    }
  \end{example}

  \column{0.48\textwidth}
  \vspace{-15pt}
  \begin{figure}[tb]
    \includegraphics[width=0.8\columnwidth]{figures/2-5/2-5-2.pdf}
  \end{figure}
  \vspace{-10pt}
  \begin{block}{Accessibility Base Graph}
    \textrm{
    \ssize{
    \begin{itemize}
      \item $G_{accs} = \{V, E_a, L\}$
      \item $V = \mathcal{S}_{partition}$ is the set of vertices
      \item $E_a = \{ (v_i, v_j, d_k) | (v_i, v_j) \in D2P(d_k) \}$ is the set of labeled, directed edges
      \item $L = \mathcal{S}_{door}$ is the set of edge labels
    \end{itemize}
    }
    }
  \end{block}

\end{columns}

\end{frame}

%------------------------------------------------

\begin{frame}
\frametitle{Distance-Aware Model}

\fsize{\textrm{The $G_{accs}$ graph does not capture indoor distance information. \conceptbf{Extended Graph Model} is proposed to integrate indoor distances into the graph in a seamless way. \emph{Minimum Indoor Walking Distance}(MIWD) is used. }}

\begin{block}{Extended Graph Model $G_{dist} = \{ V, E_a, L, f_{dv}, f_{d2d} \}$}
  \textrm{
  \begin{sitemize}
    \item $V = \mathcal{S}_{partition}$ is the set of vertices
    \item $E_a = G_{accs}.E_a$
    \item $L = \mathcal{S}_{door}$ is the set of edge labels
    \item $f_{dv} = \mathcal{S} \times V \rightarrow \mathcal{R} \cup \{ \infty \}$ maps an edge to a distance value.
    \begin{equation*}
      f_{dv} = \left\{\begin{matrix}
                \max_{p \in v_j}|| d_i, p || , & if~~ v_j \in D2P_{\sqsupset};
                \\
                \infty , & otherwise.
                \end{matrix}\right.
    \end{equation*}
    \item $f_{d2d} = V \times \mathcal{S}_{door} \times \mathcal{S}_{door} \rightarrow \mathcal{R} \cup \{ \infty \}$ maps a 3-tuple to a distance value.
    \begin{equation*}
      f_{dv} = \left\{\begin{matrix}
                || d_i, d_j ||_{v_k} , & if~~ d_i \in P2D_{\sqsupset}(v_k) and d_j \in P2D_{\sqsubset}(v_k);
                \\
                \infty , & if~~ d_i = d_j and d_i,d_j \in P2D(v_k);
                \\
                0 , & otherwise.
                \end{matrix}\right.
    \end{equation*}
  \end{sitemize}
  }
\end{block}

\end{frame}

%------------------------------------------------

\begin{frame}
\frametitle{Indoor Distance Computation: \emph{door-to-door distance}}

\begin{columns}[c]

  \column{0.52\textwidth}
  \begin{figure}[tb]
    \includegraphics[width=\columnwidth]{figures/2-5/2-5-3.pdf}
  \end{figure}

  \column{0.48\textwidth}


\end{columns}

\end{frame}

%------------------------------------------------

\begin{frame}
\frametitle{Indoor Distance Computation: \emph{point-to-point distance}}

\begin{columns}[c]

  \column{0.52\textwidth}
  \begin{figure}[tb]
    \includegraphics[width=\columnwidth]{figures/2-5/2-5-4.pdf}
  \end{figure}

  \column{0.48\textwidth}


\end{columns}

\end{frame}

%------------------------------------------------

\begin{frame}
\frametitle{Indoor Distance Computation: \emph{point-to-point distance} (I)}

\begin{columns}[c]

  \column{0.52\textwidth}
  \begin{figure}[tb]
    \includegraphics[width=\columnwidth]{figures/2-5/2-5-4.pdf}
  \end{figure}

  \column{0.48\textwidth}


\end{columns}

\end{frame}

%------------------------------------------------

\begin{frame}
\frametitle{Indoor Distance Computation: \emph{point-to-point distance} (II)}

\begin{columns}[c]

  \column{0.4\textwidth}
  \begin{figure}[tb]
    \includegraphics[width=\columnwidth]{figures/2-5/2-5-5.pdf}
  \end{figure}

  \column{0.48\textwidth}


\end{columns}

\end{frame}


% %------------------------------------------------
%
% \begin{frame}
% \frametitle{Indoor Distance Computation: \emph{point-to-point distance} (III)}
%
% \begin{columns}[c]
%
%   \column{0.52\textwidth}
%   \begin{figure}[tb]
%     \includegraphics[width=\columnwidth]{figures/2-5/2-5-5.pdf}
%   \end{figure}
%
%   \column{0.48\textwidth}
%
%
% \end{columns}
%
% \end{frame}

% %------------------------------------------------

\begin{frame}[allowframebreaks]
\frametitle{References}

\begin{thebibliography}{99} % Beamer does not support BibTeX so references must be inserted manually as below
\bibliographystyle{abbrv}
\fsize{

\bibitem{DBLP:conf/mdm/JensenLY09}
C.~S. Jensen, H.~Lu, and B.~Yang.
\newblock Graph model based indoor tracking.
\newblock In {\em {MDM}}, pp. 122--131, 2009.

\bibitem{DBLP:conf/cikm/YangLJ09}
B.~Yang, H.~Lu, and C.~S. Jensen.
\newblock Scalable continuous range monitoring of moving objects in symbolic
  indoor space.
\newblock In {\em CIKM}, pp. 671--680, 2009.

\bibitem{DBLP:conf/edbt/YangLJ10}
B.~Yang, H.~Lu, and C.~S. Jensen.
\newblock Probabilistic threshold k nearest neighbor queries over moving
  objects in symbolic indoor space.
\newblock In {\em EDBT}, pp. 335--346, 2010.

\bibitem{DBLP:conf/icde/LuYCJ11}
H.~Lu, B.~Yang, and C.~S. Jensen.
\newblock Spatio-temporal Joins on Symbolic Indoor Tracking Data.
\newblock In {\em {ICDE}}, pp. 816--827, 2011.

\bibitem{jensen2010indoor}
C.~S. Jensen, H.~Lu and B.~Yang.
\newblock Indoor-A New Data Management Frontier.
\newblock In {\em IEEE Data Eng. Bull.}, pp. 12--17, 2010.

\bibitem{DBLP:conf/icde/LuCJ12}
H.~Lu, X.~Cao, and C.~S. Jensen.
\newblock A foundation for efficient indoor distance-aware query processing.
\newblock In {\em ICDE}, pp. 438--449, 2012.

\bibitem{becker2005location}
C.~Becker and F.~D{\"u}rr.
\newblock On location models for ubiquitous computing.
\newblock In {\em Personal and Ubiquitous Computing}, pp. 20--31, 2005.

\bibitem{li2008lattice}
D.~Li and D.~L.~Lee.
\newblock A lattice-based semantic location model for indoor navigation.
\newblock In {\em MDM}, pp. 17--24, 2008.

\bibitem{becker2009multilayered}
T.~Becker, C.~Nagel and T.~H.~Kolbe
\newblock A multilayered space-event model for navigation in indoor spaces.
\newblock In {\em 3D Geo-Information Sciences}, pp. 61--77, 2009.

}
\end{thebibliography}

\end{frame}


\subsection{2.6 Efficient Distance-aware Query Evaluation on Indoor Moving Objects} % A subsection can be created just before a set of slides with a common theme to further break down your presentation into chunks

% \begin{frame}
\frametitle{About This Work...}

\emph{Efficient Distance-Aware Query Evaluation on Indoor Moving Objects}.~\cite{DBLP:conf/icde/XieLP13} \\
X.~Xie, H.~Lu, and T.~B. Pedersen.\\~\\

\begin{itemize}
  \item Published at \emph{ICDE' 2013}.
  \item Study indoor distances and effective prunning bounds in relation to indoor moving objects.
  \item Design a composite index for indoor spaces and moving objects.
  \item Define and evaluate range queries as well as $k$nn queries on indoor moving objects.
\end{itemize}

\end{frame}

%------------------------------------------------

\begin{frame}
\frametitle{Motivation}

\begin{itemize}
  \item In many indoor LBS scenarios, appropriate handling of indoor distances and relevant queries is of critical.
    \begin{fitemize}
      \item a cafe in a mall may send message to nearby shoppers to boost its business
      \item in a large airport, it important to monitor individuals within a pre-defined range from a sensitive point
    \end{fitemize}

  \item Indoor spaces are characterized by many special entities and thus render distance calculation very complex.

  \item The limitations of indoor positioning technologies create inherent uncertainties in indoor moving objects data.

\end{itemize}

\end{frame}

%------------------------------------------------

\begin{frame}
\frametitle{Notations}

\begin{figure}[tb]
  \includegraphics[width=0.75\columnwidth]{figures/2-6/2-6-1.pdf}
\end{figure}

\end{frame}

%------------------------------------------------

\begin{frame}
\frametitle{Preliminaries: Indoor Space and Indoor Distance}

\conceptbf{Doors Graph} has been proposed to represent the connectivity of indoor partitions as well as door-to-door distances.~\cite{DBLP:conf/edbt/YangLJ10}\\~\\\pause

Given two indoor positions $p$ an $q$, we use $q \overset{\delta}{\rightsquigarrow} p$ to denote a path from $q$ to $p$ where $\delta$ is the sequence of doors on the path.\\~\\\pause

The length of the shortest path as \emph{indoor distance} from $q$ to $p$, and denote it formally as $|q, p|_{I} = min_{\delta}(|q \overset{\delta}{\rightsquigarrow} p|)$, also $q \overset{\delta}{\rightarrow} p$.\\~\\\pause

\emph{indoor distance} consists of \emph{door-door distance} and \emph{intra-partition object-door distance}:\pause
\begin{equation}
  min_{d_q \in D(q), d_p \in D(p)}(|q, d_q|_{E} + |d_q, d_p|_{I} + |d_p, p|_{E})
\end{equation}

\end{frame}

%------------------------------------------------

\begin{frame}
\frametitle{Indoor Moving Objects}

\begin{itemize}
  \item Existing proposals~\cite{pfoser1999capturing, DBLP:conf/edbt/YangLJ10} model a moving object by an \emph{uncertainty region}, where the exact location is considered as a random variable inside.
  \item The possibility of its appearance can be collected by object's velocities~\cite{DBLP:conf/edbt/YangLJ10}, parameters of positioning device~\cite{pfoser1999capturing}, or analysis of historical records (represented by \emph{pdf}).
  \item The \emph{pdf} can be either a close form equation~\cite{cheng2003evaluating,cheng2004querying} or a set of instance representation~\cite{kriegel2007probabilistic}, as it is general for arbitrary distribution.
  \item Thus, an indoor moving object $O$ is represented by a set ${(s_i, p_i)}$, where $s_i$ is an instance and $p_i$ is its \emph{existential probability}, satisfying $\sum_{s_i \in O}p_i = 1$.
\end{itemize}

\end{frame}

%------------------------------------------------

\begin{frame}
\frametitle{Expected Indoor Distance}

\begin{definition}[Expected Indoor Distance for Uncertain Object]
  Given a fixed point $q \in \mathbb{I}$ and an uncertain object $O$, the indoor distance from $q$ to $O$ is
  \begin{equation}
    |q, O|_{I} = E_{s_i \in O}(|q,s_i|_{I}) = \sum_{s_i \in O}|q,s_i|_{I} \cdot p_i
  \end{equation}
\end{definition}
\vspace{10pt}
an object $O$'s uncertainty region may overlap with multiple partitions. Accordingly, all the instances in $O$ are divided into subsets, i.e., $O = \cup_{1 \leq j \leq m}S[j](1 \leq m \leq |O|)$ where each $S[j]$ corresponds to a different partition, it is called $O$'s \emph{uncertainty subregion}.

\end{frame}

%------------------------------------------------

\begin{frame}
\frametitle{Case of Indoor Distance $|q, O|_I$ (I)}

\conceptbf{Single-Partition Single-Path Distance}~$O$'s uncertainty region falls into one single partition $P$. For an arbitrary $s_i \in O$, the shortest path $q \overset{*d}{\rightarrow} s_i$ shares the path enters $P$ to reach $s_i$.

\begin{equation}
  |q, O|_{I} = |q,d|_I + \sum_{s_i \in O}|d, s_i|_E \cdot p_i
\end{equation}

\end{frame}

%------------------------------------------------

\begin{frame}
\frametitle{Case of Indoor Distance $|q, O|_I$ (II)}

\conceptbf{Single-Partition Multi-Path Distance}~$O$'s uncertainty region still falls into one single partition $P$. However, for different instances $s_i$ and $s_j$, the shortest path $q \overset{*}{\rightarrow} s_i$ and $q \overset{*}{\rightarrow} s_j$ do not share the same door sequence.

\begin{equation}
  |q, O|_{I} = \sum_{s_i \in O}|q, s_i|_I \cdot p_i
\end{equation}

\begin{columns}[c]

  \column{0.24\textwidth}
  \begin{figure}[tb]
    \includegraphics[width=\columnwidth]{figures/2-6/2-6-2.pdf}
  \end{figure}

  \column{0.76\textwidth}
  \begin{example}
    $O$ has two instance $s_1$ and $s_2$, the shortest path from $q$ to them are: $q \overset{d_3, d_1}{\rightsquigarrow} s_1$ and $q \overset{d_2}{\rightsquigarrow} s_2$.
  \end{example}

\end{columns}

\end{frame}

%------------------------------------------------

\begin{frame}
\frametitle{Case of Indoor Distance $|q, O|_I$ (II)}

The \emph{solution space} of the single-partition multi-path distance is the \conceptbf{Additive Weighted Voronoi Diagram}.\\~\\

Suppose partition $P$ has doors $\{d_1, ..., d_m\}$, for each door $d_i$, a weight $w_i = |q, d_i|_I$ is assigned. Use \emph{weighted bisectors} to represent the \emph{Additive Weighted Voronoi Diagram}. Given two doors $d_i$ and $d_j$, whose weights are $w_i$ and $w_j$, respectively, the \emph{weighted bisector} $b_{ij}$ is a curve:
\begin{equation}
  b_{ij} = \{ p : |p,d_i|_E + w_i = |p,d_j|_E + w_j \}
\end{equation}

\vspace{-20pt}
\begin{columns}[c]

  \column{0.2\textwidth}
  \begin{figure}[tb]
    \includegraphics[width=\columnwidth]{figures/2-6/2-6-4.pdf}
  \end{figure}

  \column{0.8\textwidth}
  \begin{figure}[tb]
    \includegraphics[width=\columnwidth]{figures/2-6/2-6-3.pdf}
  \end{figure}

\end{columns}

\end{frame}

%------------------------------------------------

\begin{frame}
\frametitle{Case of Indoor Distance $|q, O|_I$ (III)}

\conceptbf{Multi-Partition Multi-Path Distance}~$O$'s uncertainty region overlaps with more than one partition, and thus $O = \cup_{1 \leq j \leq m}S[j](1 \leq m \leq |O|)$.

\begin{equation}
  |q, O|_I = \sum_{1 \leq j \leq m}(|q,S[j]|_I \cdot \sum_{s_i \in S[j]}p_i)
\end{equation}

$|q,S[j]|_I$ is calculated according to case I or case II, by substituting $S[j]$ for $O$.

\vspace{-5pt}
\begin{columns}[c]

  \column{0.2\textwidth}
  \begin{figure}[tb]
    \includegraphics[width=\columnwidth]{figures/2-6/2-6-5.pdf}
  \end{figure}

  \column{0.8\textwidth}
  \begin{example}
    $O$ has three uncertainty subregions $S_1$, $S_2$ and $S_3$. Accordingly, $|q,O|_I = E(\sum_{1 \leq j \leq 3}(|q, S[j]_I|))$.
  \end{example}

\end{columns}

\end{frame}

%------------------------------------------------

\begin{frame}
\frametitle{Bounds for Indoor Distances}

\conceptbf{Euclidean Lower Bounds}

\vspace{10pt}
\begin{lemma}[Euclidean Lower Bounds]
  For point $q$ and object $O$ in an indoor space, the (virtual) Euclidean distance between them is the lower bound of their indoor space. Therefore, it has $|q,O|_{minE} \leq |q,O|_I$, where $|q, O|_{minE} = \min_{s_i \in O}|q,s_i|_E$.
\end{lemma}

\vspace{10pt}
\textrm{it is impossible to derive the indoor upper bounds by using Euclidean distances only.}

\end{frame}

%------------------------------------------------

\begin{frame}[allowframebreaks]
\frametitle{Bounds for Indoor Distances}

\conceptbf{Indoor Topological ULBounds}

\begin{lemma}[Topological Lower Bounds]
  \ssize{
  Let $t_{min}(S[i])$ be: $$\min_{d_q \in D(P(q)), d_s \in D(P(S[i]))}|q,d_q|_{minE} + |d_q \overset{*}{\rightarrow} d_s| + |d_s, S[i]|_{minE}$$. Then, $|q,O|_I \geq min\{ t_{min}(S[i]) \}$.
  }
\end{lemma}

\begin{lemma}[Topological Upper Bounds]
  \ssize{
  Let $t_{max}(S[i])$ be: $$\min_{d_q \in D(P(q)), d_s \in D(P(S[i]))}|q,d_q|_{maxE} + |d_q \overset{*}{\rightarrow} d_s| + |d_s, S[i]|_{maxE}$$. Then, $|q,O|_I \leq max\{ t_{max}(S[i]) \}$.
  }
\end{lemma}

\textrm{a looser topological upper bound is more economic to be derived, it also requires knowing some paths connecting point $q$ and subregion $S[i]$}:

\begin{lemma}[Topological Looser Upper Bounds, TLU]
  \ssize{
  Let $t_{max}(S[i])$ be: $$\min_{d_q \in D(P(q)), d_s \in D(P(S[i]))}|q,d_q|_{maxE} + |d_q \overset{*}{\rightsquigarrow} d_s| + |d_s, S[i]|_{maxE}$$. Then, $|q,O|_I \leq max\{ t_{max}(S[i]) \}$.
  }
\end{lemma}

\end{frame}

%------------------------------------------------

\begin{frame}
\frametitle{Bounds for Indoor Distances}

\conceptbf{Indoor Probabilistic ULBounds}

\vspace{10pt}
\begin{columns}[c]

  \column{0.3\textwidth}
  \begin{figure}[tb]
    \includegraphics[width=\columnwidth]{figures/2-6/2-6-6.pdf}
  \end{figure}

  \column{0.7\textwidth}
  \begin{lemma}[Markov Lower Bounds]
    Suppose object $O$ overlaps with $m$ partitions $(O = \cup_{i=1}^{m}S[i])$, and $S[i]$s are sorted according to the minimum distance to a given point $q$. Use $\widehat{p_i}$ to denote $\sum_{j=1}^{i}p_i$. As $S[i]$ and $S[j]$ do not overlap, using \emph{Markov Inequality}, we have:
    \begin{equation*}
      E(|q, O|_I) \geq |q,S[i]|_{maxI} \cdot (1 - \widehat{p_i})
    \end{equation*}
  \end{lemma}

\end{columns}

\end{frame}

%------------------------------------------------

\begin{frame}
\frametitle{Bounds for Indoor Distances}

\conceptbf{Indoor Probabilistic ULBounds}

\vspace{10pt}
\begin{columns}[c]

  \column{0.3\textwidth}
  \begin{figure}[tb]
    \includegraphics[width=\columnwidth]{figures/2-6/2-6-6.pdf}
  \end{figure}

  \column{0.7\textwidth}
  \begin{lemma}[Probabilistic ULBounds]
  \begin{equation*}
  \begin{split}
      & |q,S[i]|_{maxI} \cdot (1 - \widehat{p_i}) + |q, O|_{minI} \cdot \widehat{p_i} \\
      & \leq E(|q, O|_I) \leq \\
      & |q, O|_{maxI} \cdot (1 - \widehat{p_i}) + |q,S[i]|_{maxI} \cdot \widehat{p_i}
  \end{split}
  \end{equation*}
  \ssize{
  \textbf{Proof:} $ E(|q, O|_I)  = E(|q, \cup_{j \leq i}S[j]|_I) \cdot \widehat{p_i} +$\\
  $ E(|q, \cup_{k > i}S[k]|_I) \cdot (1 - \widehat{p_i})$. Since $|q,S[i]|_{maxI} \geq E(|q, \cup_{j \leq i}S[j]|_I) \geq |q,O|_{minI}$, and $|q,O|_{maxI} \geq E(|q, \cup_{k > i}S[k]|_I) \geq |q,O|_{minI}$, by substitution, the lemma is proved.
  }
  \end{lemma}

\end{columns}

\end{frame}

%------------------------------------------------

\begin{frame}
\frametitle{Summary}

use \emph{topological ULBounds} for the case that an object overlaps with a single partition; \\~\\

use \emph{probabilistic ULBounds} for the case that an object overlaps with multiple partitions.

\begin{figure}[tb]
  \includegraphics[width=0.7\columnwidth]{figures/2-6/2-6-7.pdf}
\end{figure}

with the Upper and Lower Bounds, as well as the approximate indoor distance, one can avoid computing shortest paths for all existential instances of an uncertain objects.

\end{frame}

%------------------------------------------------

\begin{frame}
\frametitle{Composite Index for Indoor Space}

\begin{columns}[c]

  \column{0.5\textwidth}
  \begin{figure}[tb]
    \includegraphics[width=\columnwidth]{figures/2-6/2-6-8.pdf}
  \end{figure}

  \column{0.5\textwidth}
  \begin{fitemize}
    \item \conceptbf{geometric layer} consists of a tree structure that adapts the R$^*$-tree to index all partitions, as well as a skeleton tier that maintains a small number of distances between staircases.
    \item \conceptbf{topological layer} maintains the connectivity information between indoor partitions.
    \item \conceptbf{object layer} stores all indoor moving objects and is associated with the tree through partitions at its leaf level.
  \end{fitemize}

\end{columns}

\end{frame}

%------------------------------------------------

\begin{frame}
\frametitle{Composite Index: Overview}

\begin{figure}[tb]
  \includegraphics[width=\columnwidth]{figures/2-6/2-6-9.pdf}
\end{figure}

\end{frame}

%------------------------------------------------

\begin{frame}
\frametitle{Composite Index: Tree Tier}

\begin{fitemize}
  \item instead of 3D $Minimum Bounding Rectangle$, when creating the tree, set the vertical length for one partition to 1 centimeter. Two advantage: 1) reduce the distance calculation workload; 2) makes the distance reflected in the tree more accurate without the disturbance from the vertical dimension.
  \item the imbalanced partition are decomposed to small but regular region, each is called an \emph{index unit}.
  \item A hash table is used to map such an index unit to its original indoor partition.
\end{fitemize}

\vspace{-10pt}
\begin{figure}[tb]
  \includegraphics[width=0.55\columnwidth]{figures/2-6/2-6-10.pdf}
\end{figure}

\end{frame}

%------------------------------------------------

\begin{frame}
\frametitle{Composite Index: Object Tier}

A hash table $o-table$

\begin{equation*}
  o-table : \{ O \} \rightarrow 2^{\{index~unit\}}
\end{equation*}

$o-table$ maps an object to all the index units it overlaps, and it is tightly tie up with the tree tier.\\~\\

When an object update occurs, $o-table$ needs to be updated accordingly.

\end{frame}

%------------------------------------------------

\begin{frame}
\frametitle{Composite Index: Topological Tier}

This layer maintains the connectivity between partitions. Each leaf node stores a (sub)partition.\\~\\

For accessibility, the doors belonging to the partitions are also stored, as well as the the links to accessible partitions through each door.

\end{frame}

%------------------------------------------------

\begin{frame}
\frametitle{Composite Index: Skeleton Tier}

Skeleton Tier is a graph, each staircase entrance is captured as a graph node, and an edge connects two nodes if their entrances are on the same floor or their entrances belong to the same staircase.\\~\\

The weight of an edge is the indoor distance between the two staircase entrances.

\vspace{-10pt}
\begin{columns}[c]

  \column{0.4\textwidth}
  \begin{figure}[tb]
    \includegraphics[width=\columnwidth]{figures/2-6/2-6-11.pdf}
  \end{figure}

  \column{0.6\textwidth}
  \ssize{
  \begin{definition}[staircase distance matrix $M_{s2s}$]
    \begin{sitemize}
      \item $M_{s2s}[s_i,s_i] = 0$;
      \item $M_{s2s}[s_i,s_j] = |s_i, s_j|_E$ if $s_i$ and $s_j$ are on the same floor;
      \item if $s_i$ and $s_j$ are of a same staircase, $M_{s2s}[s_i,s_j]$ is the shortest distance from $s_i$ to $s_j$ within that staircase;
      \item $M_{s2s}[s_i,s_j]$ is calculated as the shortest path distance from $s_i$ to $s_j$ in the skeleton layer for other cases.
    \end{sitemize}
  \end{definition}
  }

\end{columns}

\end{frame}

%------------------------------------------------

\begin{frame}
\frametitle{Skeleton Distance}

\textrm{Let $q$ be a fixed indoor point, $q.f$ the floor of $q$, and $S(q.f)$ all the staircases on floor $q.f$.}

\vspace{10pt}
\begin{definition}[Skeleton Distance]
  Given two points $p$ and $q$, their skeleton distance $|q,p|_K = |q,p|_E$ if they are on the same floor; otherwise, $|q,p|_K = \min_{s_q \in S(q.f), s_p \in S(p.f)}(|q,s_q|_E + M_{s2s}[s_q,s_p] + |s_p, p|_E)$.
\end{definition}

\vspace{10pt}
Define the skeleton distance as the alternative \emph{Geometric Distance}.

\end{frame}

%------------------------------------------------

\begin{frame}
\frametitle{Indoor Distance Bounds in the Geometric Layer}

\begin{lemma}[Geometric Lower Bound Property]
  Given two points $p$ and $q$, their skeleton distance lower bounds their indoor distance, i.e., $|q,p|_K \leq |q,p|_I$.\\~\\
  \textbf{Proof:}~If $q$ and $p$ are on the same floor, $|q,p|_K = |q,p|_E \leq |q,p|_I$. Otherwise, suppose $s_{q}^{*} \in S(q.f)$ and $s_{p}^{*} \in S(p.f)$ are on the shortest path from $q$ to $p$, denoted by $q \overset{*s_{q}^{*}*s_{p}^{*}}{\rightarrow} p$. Since $|q,p|_K = \min_{s_q \in S(q.f), s_p \in S(p.f)}(|q,s_q|_E + M_{s2s}[s_q,s_p] + |s_p, p|_E) \leq |q,s_{q}^{*}|_E + M_{s2s}[s_{q}^{*},s_{p}^{*}] + |s_{p}^{*}, p|_E = |q,p|_I$, the lemma is proved.
\end{lemma}

\end{frame}

%------------------------------------------------

\begin{frame}
\frametitle{Indoor Distance Bounds in the Geometric Layer}

Consider an entity $e$ that is either an object or an $ind$R-tree node. If $e$ spans multiple floors, we use interval $[e.lf,e.uf]$ to represent all those floors. Note those floors must be consecutive. We define the minimum skeleton distance $|q,e|_{minK}$:

\begin{figure}[tb]
  \includegraphics[width=0.7\columnwidth]{figures/2-6/2-6-12.pdf}
\end{figure}

With $|q,e|_{minK}$, one can constrain the search via the $ind$R-tree to a much smaller range compared to if use the Euclidean distance bounds.

\end{frame}

%------------------------------------------------

\begin{frame}
\frametitle{Dynamic Operations on the Topological Layer}

\textbf{Insertion.}~\textrm{When the topological change leads to a new indoor partition $P$, $P$(or its sub-partitions due to decomposition) is inserted into the $ind$R-tree, its leaf node is connected to the adjacent partitions, and the $h-table$ is updated if a decomposition is involved.}\\~\\

\textbf{Deletion.}~\textrm{From the $ind$R-tree to remove a partition $P$ to be deleted, the links involving $P$ are removed from the adjacent partitions, and $P$'s entry in the $h-table$ is deleted if $P$ is a decomposed sub-partition.}

\end{frame}

%------------------------------------------------

\begin{frame}
\frametitle{Dynamic Operations on the Object Layer}

\textbf{Insertion.}~\textrm{To insert an object $O$, search the $ind$R-tree to find the leaf nodes $\{ P_i \}$ that overlap with $O$'s uncertainty region. Also insert a new entry to $o-table$.}\\~\\

\textbf{Deletion.}~\textrm{To delete an object $O$, use the $o-table$ to find the $ind$R-tree leaf nodes $\{ P_i \}$ that overlap with $O$'s uncertainty region. For each $P_i$, $O$ is removed from its associated bucket. Also the entry for $O$ is deleted from the $o-table$.}

\end{frame}

%------------------------------------------------

\begin{frame}
\frametitle{Query Semantics}

\begin{definition}[Indoor Range Query, iRQ]
  Given a query point $q \in \mathbb{I}$ and a distance value $r$, the \emph{iRQ} returns objects whose indoor distances are smaller than $r$. Formally, $iRQ_{q,r}(\mathbb{O}) = \{ O | |q,O|_I \leq r, O \in \mathbb{O}\}$.
\end{definition}

\begin{definition}[Indoor $k$ Nearest Neighbor Query, ikNNQ]
  Given a query point $q \in \mathbb{I}$ and a parameter $k$, the \emph{ikNNQ} returns $k$ objects whose indoor distances to $q$ are the smallest among all objects. Formally, $ikNN_{q,k}(\mathbb{O}) = \{ O | O \in \mathbb{O}\}$, where $|ikNN_{q,k}(\mathbb{O})| = k, \forall O_i \in ikNN_{q,k}(\mathbb{O}), \forall O_j \in \mathbb{O} \setminus ikNN_{q,k}(\mathbb{O}), |q,O_i|_I \leq |q,O_j|_I$.
\end{definition}

\end{frame}

%------------------------------------------------

\begin{frame}
\frametitle{Efficient Query Evaluation}

\begin{enumerate}
  \fsize{
  \item \conceptbf{Filtering Phase} locates the source partition that contains the query point and retrieves condidate partitions as well as candidate objects.
  \item \conceptbf{Subgraph Phase} constructs a subgraph based on candidate partitions, and uses the doors of the source partition as sources to compute the shortest indoor paths that are to be used in the subsequent two phases.
  \item \conceptbf{Pruning Phase}, upper/lower distance bounds for objects are calculated to further reduce the number of candidate objects.
  \item \conceptbf{Refinement Phase}, the indoor distances for the remaining objects are computed and the qualifying objects are returned as the query results.
  }
\end{enumerate}

\end{frame}

%------------------------------------------------

\begin{frame}
\frametitle{Indoor Range Query}

\begin{columns}[c]

  \column{0.2\textwidth}
  \begin{figure}[tb]
    \includegraphics[width=\columnwidth]{figures/2-6/2-6-13.pdf}
  \end{figure}


  \column{0.8\textwidth}
  \begin{example}
    \ssize{
    The circle $\bigodot(q,r)$ is the query region represented in the Euclidean space. Object $O_1$ is pruned away in filtering phase, since $|q, O_1|_{minE} > r$. After deriving the upper/lower bounds for the remaining objects in the pruning phase, $O_3$ is qualified. For the undetermined object $O_2$, the exact indoor distance is calculated and compared to $r$.
    }
  \end{example}

\end{columns}

\begin{columns}[c]

  \column{0.4\textwidth}
  \begin{figure}[tb]
    \includegraphics[width=\columnwidth]{figures/2-6/2-6-14.pdf}
  \end{figure}

  \column{0.6\textwidth}
  \begin{sitemize}
    \item in the filtering, iRQ calls $RangeSearch$ to search the geometric layer.
    \item lines 5--10: iRQ makes use of the topological upper/lower bounds to approximate indoor distances and compare them to $r$.
    \item lines 11--13: the exact indoor distances are only computed for those objects whose bounds cover $r$.
  \end{sitemize}

\end{columns}

\end{frame}

%------------------------------------------------

\begin{frame}
\frametitle{Indoor $k$ Nearest Neighbor Query}

\begin{columns}[c]

  \column{0.16\textwidth}
  \begin{figure}[tb]
    \includegraphics[width=\columnwidth]{figures/2-6/2-6-15.pdf}
  \end{figure}


  \column{0.84\textwidth}
  \begin{example}
    \ssize{
    $kSeedsSelection$ finds $O_2$ and $O_3$ as seeds. Because $O_2$'s topological looser upper bound is longer, it is chosen as the kbound. Through the range search, $O_1$ is excluded since $|q,O_1|_K > kbound$.
    }
  \end{example}

\end{columns}

\begin{columns}[c]

  \column{0.35\textwidth}
  \begin{figure}[tb]
    \includegraphics[width=\columnwidth]{figures/2-6/2-6-16.pdf}
  \end{figure}

  \column{0.65\textwidth}
  \begin{sitemize}
    \item in the filtering, ikNNQ calls $kSeedsSelection$ to return an object $R_1^o$ and a partition set $R_1^p$.
    \item $R_1^o$ contains $k$ objects taht are in query point $q$'s partition or in the closet adjacent partitions. $R_1^p$ is the set of all those involved partition.
    \item ikNNQ derives \emph{Topological Looser Upper Bounds} for the $k$ objects and choose the longest one as $kounds = \max_{seed_i \in R_1^o}\{ |q,seed_i|_I.TLU \}$.
    \item Line 4: a range search $\bigodot(q,kound)$ is done on the tree tier.
  \end{sitemize}

\end{columns}

\end{frame}

%------------------------------------------------

\begin{frame}
  \frametitle{Research Directions}

  \begin{itemize}
  	\item it is of interest to study other query types using the distance bounds and the composite index proposed in this paper.
    \item it is useful to estimate the selectivity for indoor distance aware queries and make use of it in further optimizing queries over uncertain object.
    \item it is of beneficial to reuse computational efforts on indoor distances when multiple, related queries are issued within a short period of time.
  \end{itemize}

\end{frame}

% %------------------------------------------------

\begin{frame}[allowframebreaks]
\frametitle{References}

\begin{thebibliography}{99} % Beamer does not support BibTeX so references must be inserted manually as below
\bibliographystyle{abbrv}
\fsize{

\bibitem{DBLP:conf/edbt/YangLJ10}
B.~Yang, H.~Lu, C.S.~Jensen.
\newblock Probabilistic threshold k nearest neighbor queries over moving objects in symbolic indoor space.
\newblock In {\em EDBT}, pp. 335--346, 2010.

\bibitem{DBLP:conf/icde/LuCJ12}
H.~Lu, X.~Cao, C.S.~Jensen.
\newblock A foundation for efficient indoor distance-aware query processing.
\newblock In {\em ICDE}, pp. 438--449, 2012.

\bibitem{DBLP:conf/icde/XieLP13}
X.~Xie, H.~Lu, T.B.~Pedersen.
\newblock Efficient distance-aware query evaluation on indoor moving objects.
\newblock In {\em ICDE}, pp. 434--445, 2013.

\bibitem{pfoser1999capturing}
D.~Pfoser, C.S.~Jensen.
\newblock Capturing the uncertainty of moving-object representations
\newblock In {\em Advances in Spatial Databases}, pp. 111--131, 1999.

\bibitem{cheng2003evaluating}
R.~Cheng, D.V.~Kalashnikov, S.~Prabhakar.
\newblock Evaluating probabilistic queries over imprecise data.
\newblock In {\em SIGMOD}, pp. 551--562, 2003.

\bibitem{cheng2004querying}
R.~Cheng, D.V.~Kalashnikov, S.~Prabhakar.
\newblock Querying imprecise data in moving object environments.
\newblock In {\em TKDE}, pp. 1112--1127, 2004.

\bibitem{kriegel2007probabilistic}
H.-P.~Kriegel, P.~Kunath, M.~Renz.
\newblock Probabilistic nearest-neighbor query on uncertain objects.
\newblock In {\em DSAFAA}, pp. 337--348, 2007.

}
\end{thebibliography}

\end{frame}


\subsection{2.7 Distance-Aware Join for Indoor Moving Objects} % A subsection can be created just before a set of slides with a common theme to further break down your presentation into chunks

% \begin{frame}
\frametitle{About This Work...}

\emph{Leveraging Spatio-Temporal Redundancy for RFID Data Cleansing}.~\cite{xie2015distance} \\
X.~Xie, H.~Lu, and T.~B. Pedersen.\\~\\

\begin{itemize}
  \item Published at \emph{TKDE' 2015}.
  \item Study efficient evaluation of distance-aware join operations on indoor moving objects, semi-range join and semi-neighborhood join.
  \item Design a composite index for indoor space as well as objects.
\end{itemize}

\end{frame}

%------------------------------------------------

\begin{frame}
\frametitle{Motivation}

\begin{itemize}
  \item People spend a large part of their lives in indoor spaces.

  \item the New Town Plaza in Hong Kong covers 200,000 square meters and consists of 34 interconnected buildings. The weekend traffic is as high as 320,000 people as reported in 2004.

  \item A large Danish hospital logistic system, requires tracking up to 164,000 objects, including around 10,000 persons, 10,000 pieces of equipment, 70,000 aids and 70,000 materials over 10 floors.
\end{itemize}

\end{frame}

%------------------------------------------------

\begin{frame}
\frametitle{Distance-Aware Joins}

\begin{example}[Indoor distance-based monitoring]
  \ssize{
  In a shopping plaza, a covey of mobile security guards are patrolling and monitoring the surrounding people for the suspicious, which may appear within a range. The range can be specified by a distance threshold $\epsilon$.
  }
\end{example}

\begin{example}[Indoor facility tracking]
  \ssize{
  In a large hospital logistic system, it is time-critical to monitor patients in special care or nurses on the ward with their nearest medical facilities, such as a medical staff. The number of nearby facilities can be specified by a parameter $k$.
  }
\end{example}

\begin{example}[Indoor data analysis]
  \ssize{
  Many algorithms related to similarity search and data mining can be constructed on top of a join query. For indoor spatial databases, the join operator is an important primitive that allows efficient distance-aware analysis.
  }
\end{example}

\end{frame}

%------------------------------------------------

\begin{frame}
\frametitle{Problem Definition}

\begin{block}{}
  Given two indoor objects $Q$ and $O$, let $|Q, O|_I$ denote the indoor distance from $Q$ to $O$.
\end{block}

\begin{definition}[Semi-range Join]
  \ssize{
  Given two sets of indoor objects $\mathbb{Q}$ and $\mathbb{O}$, and a distance theshold $\epsilon$, the semi-range join of the two sets returns all pairs $\{ \langle Q, O \rangle \}$ of objects, such that the distance from $Q$ to $O$ are within $\epsilon$. Formally:
  }
  \begin{equation*}
    \mathbb{Q} \underset{\epsilon}{\ltimes} \mathbb{O} = \{ (Q, O) \in \mathbb{Q} \times \mathbb{O} | |Q,O|_I \leq \epsilon \}
  \end{equation*}

\end{definition}

\begin{definition}[Semi-neighborhood Join]
  \ssize{
  Given two sets of indoor objects $\mathbb{Q}$ and $\mathbb{O}$, and an integer $k$, the semi-neighborhood join all object pairs as follows:
  }
  \begin{equation*}
    \mathbb{Q} \underset{k}{\ltimes} \mathbb{O} = \{ (Q, O) \in \mathbb{Q} \times \mathbb{O} | O \in kNN(Q) \}
  \end{equation*}

\end{definition}

\end{frame}

%------------------------------------------------

\begin{frame}
\frametitle{Problem Definition}

\begin{itemize}
  \setlength{\itemsep}{30pt}
  \item to study semi-joins instead of ful joins, $\underset{k}{\Join}$ and $\underset{\epsilon}{\Join}$
  \begin{fitemize}
    \item indoor space is a quasimetric space where distances are not symmetrics
    \item full joins can be easily implemented by semi-joins, e.g., $Q \underset{k}{\Join} O = Q \underset{k}{\ltimes} O \cap O \underset{k}{\ltimes} Q$
  \end{fitemize}
  \item call $\mathbb{Q}$ the \conceptbf{query objects}, and $\mathbb{O}$ the \conceptbf{target objects}.
\end{itemize}

\end{frame}

%------------------------------------------------

\begin{frame}
\frametitle{Challenges in Indoor Spaces}

\begin{itemize}
  \item the indoor space $\mathbb{I}$ is a quasimetric space, given two points $p, q \in \mathbb{I}$, the distance $|p, q|_I$ satisfies:
  \begin{fitemize}
    \item $|p, q|_I \geq 0$ (non-negativity);
    \item $|p, q|_I \neq |q, p|_I$ (non-symmetry);
    \item $|p, q|_I \leq |p, e|_I + |e, q|_I$ (triangle inequality)
  \end{fitemize}
  \item indoor entities can also be associated with temporal variations
  \begin{fitemize}
    \item a room may be only temporarily available due to its opening hours, or being blocked in a fire emergency
    \item a conference hall may be partitioned into several smaller rooms
  \end{fitemize}
  \item the accuracy of indoor positioning is limited, typically varying from a few to 100 meters
\end{itemize}

\textrm{To address these challenges, we need to support indoor distances taht take into account topological constraints, temporal variations and location uncertainties.}

\end{frame}

%------------------------------------------------

\begin{frame}
\frametitle{Notations}

\vspace{-15pt}
\begin{figure}[tb]
  \includegraphics[width=0.57\columnwidth]{figures/2-7/2-7-1.pdf}
\end{figure}

\end{frame}

%------------------------------------------------

\begin{frame}
\frametitle{Preliminaries: Indoor Space and Indoor Distance}

\conceptbf{Doors Graph} has been proposed to represent the connectivity of indoor partitions as well as door-to-door distances.~\cite{DBLP:conf/edbt/YangLJ10}\\~\\\pause

Given two indoor positions $p$ an $q$, we use $q \overset{\delta}{\rightsquigarrow} p$ to denote a path from $q$ to $p$ where $\delta$ is the sequence of doors on the path.\\~\\\pause

The length of the shortest path as \emph{indoor distance} from $q$ to $p$, and denote it formally as $|q, p|_{I} = min_{\delta}(|q \overset{\delta}{\rightsquigarrow} p|)$, also $q \overset{\delta}{\rightarrow} p$.\\~\\\pause

\emph{indoor distance} consists of \emph{door-door distance} and \emph{intra-partition object-door distance}:\pause
\begin{equation}
  min_{d_q \in D(q), d_p \in D(p)}(|q, d_q|_{E} + |d_q, d_p|_{I} + |d_p, p|_{E})
\end{equation}

\end{frame}

%------------------------------------------------

\begin{frame}
\frametitle{Indoor Moving Objects}

\begin{itemize}
  \item Existing proposals~\cite{pfoser1999capturing, DBLP:conf/edbt/YangLJ10} model a moving object by an \emph{uncertainty region}, where the exact location is considered as a random variable inside.
  \item The possibility of its appearance can be collected by object's velocities~\cite{DBLP:conf/edbt/YangLJ10}, parameters of positioning device~\cite{pfoser1999capturing}, or analysis of historical records (represented by \emph{pdf}).
  \item The \emph{pdf} can be either a close form equation~\cite{cheng2003evaluating,cheng2004querying} or a set of instance representation~\cite{kriegel2007probabilistic}, as it is general for arbitrary distribution.
  \item Thus, an indoor moving object $O$ is represented by a set ${(o, o.\rho)}$, where $o$ is an instance and $o.\rho$ is its \emph{existential probability}, satisfying $\sum_{o \in O}o.\rho = 1$.
\end{itemize}

\end{frame}

%------------------------------------------------

\begin{frame}
\frametitle{Expected Indoor Distance}

\begin{definition}[Expected Indoor Distance for Uncertain Object]
  Given two uncertain object $Q$ and $O$, the indoor distance between $Q$ to $O$ is
  \begin{equation}
    |Q, O|_{I} = E_{q \in Q, o \in O}(|q,o|_{I}) = \sum_{q \in Q}\sum_{o \in O}|q,o|_{I} \cdot q.\rho \cdot o.\rho
  \end{equation}
\end{definition}
\vspace{10pt}
an object $O$'s uncertainty region may overlap with multiple partitions. Accordingly, all the instances in $O$ are divided into subsets, i.e., $O = \cup_{1 \leq j \leq m}O[j](1 \leq m \leq |O|)$ where each $O[j]$ corresponds to a different partition, it is called $O$'s \emph{uncertainty subregion}.

\end{frame}

%------------------------------------------------

\begin{frame}
\frametitle{Case of Indoor Distance $|Q, O|_I$ (I)}

\conceptbf{Single-Partition Single-Path Distance} \quad $O$'s uncertainty region falls into one single partition $P$, so does $Q$. Let $P_Q$ ($P_O$) be the partition containing $Q$ ($O$). For an arbitrary pair $(q, o)_{q \in Q, o \in O}$, the shortest path $q \overset{d_Q*d_O}{\rightarrow} o$ shares the same door sequence starting with $d_Q$ and ending with $d_O$, through which the path reaches $o$ from $q$.

\begin{equation}
  \begin{split}
  |Q, O|_{I} & = \sum_{q \in Q}\sum_{o \in O} (|q, d_Q|_E + |d_Q, d_O|_I + |d_O, o|_E)\cdot q.\rho \cdot o.\rho \\
             & = \sum_{q \in Q} |q, d_Q|_E + |d_Q, d_O|_I + \sum_{o \in O} |d_O, o|_E
  \end{split}
\end{equation}

\end{frame}

%------------------------------------------------

\begin{frame}
\frametitle{Case of Indoor Distance $|Q, O|_I$ (II)}

\conceptbf{Single-Partition Multi-Path Distance} \quad $O$ and $Q$'s uncertainty region still falls into one single partition $P$. However, for different instances $o_i$ and $o_j$, the shortest path $q \overset{*}{\rightarrow} o_i$ and $q \overset{*}{\rightarrow} o_j$ do not share the same door sequence.

\begin{equation}
  |Q, O|_{I} = \sum_{o_i \in O}|q, o_i|_I \cdot q.\rho \cdot o_i.\rho
\end{equation}

\begin{columns}[c]

  \column{0.24\textwidth}
  \begin{figure}[tb]
    \includegraphics[width=\columnwidth]{figures/2-7/2-7-2.pdf}
  \end{figure}

  \column{0.76\textwidth}
  \begin{example}
    $O$ has two instance $o_1$ and $o_2$, the shortest path from $q$ to them are: $q \overset{d_3, d_1}{\rightsquigarrow} o_1$ and $q \overset{d_2}{\rightsquigarrow} o_2$.
  \end{example}

\end{columns}

\end{frame}

%------------------------------------------------

\begin{frame}
\frametitle{Case of Indoor Distance $|Q, O|_I$ (III)}

\conceptbf{Multi-Partition Multi-Path Distance} \quad either $Q$ or $O$'s uncertainty region overlaps with more than one partition, and thus $O = \cup_{1 \leq j \leq m}O[j](1 \leq m \leq |O|)$.

\begin{equation}
  |Q, O|_I = \sum_{i}\sum_{j}(|Q[i],O[j]|_I \cdot \sum_{q \in Q[i]}q.\rho \cdot \sum_{o \in O[j]}o.\rho)
\end{equation}

$|Q[i],O[j]|_I$ is calculated according to case I or case II.

\vspace{-5pt}
\begin{columns}[c]

  \column{0.2\textwidth}
  \begin{figure}[tb]
    \includegraphics[width=\columnwidth]{figures/2-7/2-7-3.pdf}
  \end{figure}

  \column{0.8\textwidth}
  \begin{example}
    $O$ has three uncertainty subregions $O[1]$, $O[2]$ and $O[3]$. Accordingly, $|Q,O|_I = E(\sum_{1 \leq j \leq 3}(|q, O[j]_I|))$.
  \end{example}

\end{columns}

\end{frame}

%------------------------------------------------

\begin{frame}
\frametitle{Bounds for Indoor Distances}

\conceptbf{Geometric Layer Lower Bounds}\\
\ssize{
Fot two indoor uncertain objects $Q$ and $O$, the (virtual) euclidean distance between them is the lower bound of their distance in the indoor space. Therefore, it has $|Q,O|_{minE} \leq |Q,O|_{minI}$, where $|Q, O|_{minE} = \min_{q \in Q, o \in O}|q,o|_E$.
}

\vspace{10pt}
\begin{lemma}[Geometric Lower Bounds]
  Given indoor object $Q$ denoted by $\odot(c_Q, r_Q)$, and $O$ denoted by $\odot(c_O, r_O)$, the geomtric lower bound property can be rewritten as:
  \begin{equation}
    |c_Q, c_O|_E - r_Q - r_O \geq |Q, O|_{minI}
  \end{equation}
\end{lemma}

\vspace{10pt}
\textrm{it is impossible to derive the indoor upper bounds by using Euclidean distances only.}

\end{frame}

%------------------------------------------------

\begin{frame}
\frametitle{Bounds for Indoor Distances}

\conceptbf{Indoor Topological ULBounds}

\vspace{30pt}

For two objects $Q = \bigcup_{i = 1}^{m} Q[i]$ and $O = \bigcup_{j = 1}^{n} O[j]$, suppose that $P(Q[i])$ is the partition containing subregion $Q[i]$ and $P(Q)$ are the partitions overlapping with $Q$.

\end{frame}


%------------------------------------------------

\begin{frame}
\frametitle{Bounds for Indoor Distances}

\begin{lemma}[Topological Lower Bounds]
  \ssize{
  Let $t_{min}(Q[i], O[j])$ be: $$\min_{d_q \in D(P(Q[i])), d_s \in D(P(O[j]))}|Q[i],d_q|_{minE} + |d_q \overset{*}{\rightarrow} d_s| + |d_s, O[j]|_{minE}$$. Then, $|Q,O|_I \geq min_{i,j}\{ t_{min}(Q[i], O[j]) \}$.
  }
\end{lemma}

\begin{lemma}[Topological Upper Bounds]
  \ssize{
  Let $t_{max}(Q[i], O[j])$ be: $$\min_{d_q \in D(P(Q[i])), d_s \in D(P(O[j]))}|Q[i],d_q|_{maxE} + |d_q \overset{*}{\rightarrow} d_s| + |d_s, O[j]|_{maxE}$$. Then, $|Q,O|_I \leq max_{i,j}\{ t_{max}(Q[i], O[j]) \}$.
  }
\end{lemma}

\ssize{\textrm{Suppose $Q$ and $O$ overlap with $m$ and $n$ partitions respectively. The above two lemmas involve $O(mn)$ shortest paths.}}

\end{frame}

%------------------------------------------------

\begin{frame}
\frametitle{Bounds for Indoor Distances}

\ssize{\textrm{If $Q$ and $O$'s uncertainty regions both overlap with one partition, the above two lemmas can be rewritten.}}

\begin{lemma}[]
  \ssize{
  Given two indoor object $Q$ and $O$, denoted by $\odot(c_Q, r_Q)$ and $\odot(c_O, r_O)$ respectively, the topological ULBounds can be rewritten as:
  }
  \begin{equation}
    |c_Q, c_O|_I - r_Q - r_O \leq |Q, O|_I \leq |c_Q, c_O|_I + r_Q + r_O
    \label{equation:simplified_1}
  \end{equation}

\end{lemma}

\begin{proof}{}
  \ssize{
  \begin{equation*}
    \begin{split}
    |Q, O|_{minI} & = \min_{\forall q \in Q}(|q, O|_I) \geq \min_{\forall q \in Q}(|q, c_O|_I - r_O) = \min_{\forall q \in Q}(|q, c_O|_I) - r_O \\
    & \geq |c_Q, c_O|_I - r_Q - r_O \\
    & \Rightarrow |Q, O|_{minI} \geq |c_Q, c_O|_I - r_Q - r_O \\
    & \Rightarrow |Q, O|_{I} \geq |c_Q, c_O|_I - r_Q - r_O \\
    \end{split}
  \end{equation*}
  $|Q, O|_I \leq |c_Q, c_O|_I + r_Q + r_O$ can be proved likewise.
  }
\end{proof}

\end{frame}

%------------------------------------------------

\begin{frame}
\frametitle{Bounds for Indoor Distances}

\textrm{The shortest path $|d_q \overset{*}{\rightarrow} d_s|$ computation is not economic, a \conceptbf{Topological Looser UBound} is proposed. }

\begin{lemma}[Topological Looser Upper Bounds]
  \ssize{
  Let $t_{max}(Q[i], O[j])$ be: $$\min_{d_q \in D(P(Q[i])), d_s \in D(P(O[j]))}|Q[i],d_q|_{maxE} + |d_q \overset{*}{\rightsquigarrow} d_s| + |d_s, O[j]|_{maxE}$$. Then, $|Q,O|_I \leq max_{i,j}\{ t_{max}(Q[i], O[j]) \}$.
  }
\end{lemma}

\textrm{In the case that both $Q$ and $O$ overlap with one partition, the lemma can be simplified as:}
\begin{equation}
  \min_{d_q \in D(P(Q[i])), d_s \in D(P(O[j]))} |d_q \overset{*}{\rightsquigarrow} d_s| + |d_q, c_Q|_E + |d_o, c_O|_E + r_Q + r_O
  \label{equation:simplified_2}
\end{equation}

\end{frame}

%------------------------------------------------

\begin{frame}
\frametitle{Bounds for Indoor Distances}

\fsize{
The simplified versions of ULBounds are more efficient since they only take one shortest path instead of $O(mn)$ paths. To generalize the single-parition case to multiple-partition scenarios, \conceptbf{star-connected region} is defined.
}

\begin{definition}[Star-connected regions]

  Let $O = \odot(c, r)$ be an indoor object overlapping with more than one partition, i.e., $O = \bigcup_{i=1}^{n}O[i]$. Let the subregion containing $c$ be the central region $C$. If all other subregions are connected to $C$ by doors, we call $O$'s region a star-connected region, formally:

  \begin{equation*}
    \forall O[i] \neq C, \exists~\text{door}~d , \text{such that}~d \in C~\text{and}~d \in O[i]
  \end{equation*}

\end{definition}


\end{frame}

%------------------------------------------------

\begin{frame}
\frametitle{Bounds for Indoor Distances}

\begin{columns}[c]

  \column{0.2\textwidth}
  \begin{figure}[tb]
    \includegraphics[width=\columnwidth]{figures/2-7/2-7-3.pdf}
  \end{figure}

  \column{0.8\textwidth}
  \begin{example}
    \ssize{
    $Q$ is a star-connected region, since $Q[1]$ and $Q[2]$ are connected by door $d_{14}$, $O$ is not a star-connected region, since $O[2]$ and $O[3]$ are separated into two partitions, and there is no door connecting the two partitions.
    }
  \end{example}

\end{columns}

\vspace{10pt}

\textrm{Then, we can define $O$ by $\odot(c, r_I)$, where $r_I$ is the maximum indoor distance from centroid $c$ to all subregions, $r_I = \max_{i} |c, O[i]|_{maxI}$. By defining star-connected regions, we can benefit from the simplifications in topological ULBounds by substituting $r_I$ into Equations.(\ref{equation:simplified_1}), (\ref{equation:simplified_2}).}

\end{frame}

%------------------------------------------------

\begin{frame}
\frametitle{Bounds for Indoor Distances}

\begin{columns}[c]

  \column{0.2\textwidth}
  \begin{figure}[tb]
    \includegraphics[width=\columnwidth]{figures/2-7/2-7-3.pdf}
  \end{figure}

  \column{0.8\textwidth}
  \begin{example}
    \ssize{
    the distance from $q$ to $O[1]$ is short, while the distance to $O[3]$ is long. If the gap between topological upper and lower bounds is large, the expected distance is only constrained by a loose range and thus not well approximated.
    }
  \end{example}

\end{columns}

\vspace{15pt}

\textrm{Geometric and topological ULBounds bound the distance by the minimum/maximum distance between sample pairs. The \conceptbf{Object Layer ULBounds} make a difference by considering the probability distributions among sample points.}

\end{frame}

%------------------------------------------------

\begin{frame}
\frametitle{Bounds for Indoor Distances}

\begin{definition}[$\beta$-region~\cite{chen2007efficient,lian2011similarity}]
  Given an indoor object $O$, the $\beta$-region is a closed region such that the probability of $O$ being located inside the region is greater than $\beta$.
\end{definition}

\end{frame}

%------------------------------------------------

\begin{frame}
\frametitle{Bounds for Indoor Distances}

\textbf{CASE I} \quad Object's uncertainty region is relatively big compared to its indoor distance.

\vspace{30pt}

\begin{columns}[c]

  \column{0.3\textwidth}
  \begin{figure}[tb]
    \includegraphics[width=\columnwidth]{figures/2-7/2-7-4.pdf}
  \end{figure}

  \column{0.7\textwidth}
  \begin{example}
    \ssize{
      Given a predefined $\beta$ balue, the $\beta$-region can be constructed b first sorting an object's samples according to their distances from the centroid. Count and summarize their probabilities until $\beta$ is reached. The distance between the last counted sample point and the centroid is $r^\beta$. Thus, $O$'s $\beta$-region $O^\beta$ is determined by a circle $\odot(c, r^\beta)$.
    }
  \end{example}

\end{columns}
\end{frame}

%------------------------------------------------

\begin{frame}
\frametitle{Bounds for Indoor Distances}

\textbf{CASE II} \quad Object overlaps with multiple partitions that are not interconnected. (not star-connected regions)

\vspace{30pt}

\begin{columns}[c]

  \column{0.3\textwidth}
  \begin{figure}[tb]
    \includegraphics[width=\columnwidth]{figures/2-7/2-7-5.pdf}
  \end{figure}

  \column{0.7\textwidth}
  \begin{example}
    \ssize{
      Randomly select a subregion $O[i]$ as the $\beta$-region. Here the value of $\beta$ equals to the summation of probabilities for samples inside $O[i]$, i.e., $\beta = \sum_{s \in O[i]}s.\rho$. The shape of the $\beta$-region is a rectangle, which is the intersection of $O$'s MBR and the partition containing $O[i]$.
    }
  \end{example}

\end{columns}

\end{frame}

%------------------------------------------------

\begin{frame}
\frametitle{Bounds for Indoor Distances}

\begin{lemma}[\small Simplified Case for Deriving Object Layer ULBounds]
  \tiny
  Given a point $q$ and an object $O$, we have:
  \begin{equation*}
    \begin{split}
    & (1 - \beta) \cdot |q, O|_{minI} + \beta \cdot |q, O^\beta|_{minI} \leq |q, O|_I  \\
    & \leq (1 - \beta) \cdot |q, O|_{maxI} + \beta \cdot |q, O^\beta|_{maxI}
    \end{split}
  \end{equation*}
\end{lemma}

\begin{proof}{\small}
  \tiny
  We first prove $|q, O|_I \leq (1 - \beta) \cdot |q, O|_{maxI} + \beta \cdot |q, O^\beta|_{maxI}$:
  \begin{equation*}
    \begin{split}
      & |q, O|_I = E(|q, O|_I) = E_{s \in O^\beta}(|q, s|_I) \cdot Pr\{s \in O^\beta \} + \\
      & E_{s \in O \setminus O^\beta} (|q, s|_I) \cdot Pr\{s \in O \setminus O^\beta \} \\
      & = E_{s \in O^\beta}(|q, s|_I) \cdot \beta + E_{s \in O \setminus O^\beta} (|q, s|_I) \cdot (1 - \beta) \} \\
      & \leq q, O|_{maxI} + \beta \cdot |q, O^\beta|_{maxI} (\forall s \in O^\beta, |q,s|_I \leq |q, O^\beta|_{maxI})
    \end{split}
  \end{equation*}
  Likewise, we can prove $(1 - \beta) \cdot |q, O|_{minI} + \beta \cdot |q, O^\beta|_{minI} \leq |q, O|_I$, thus the lemma is proved.
\end{proof}

\end{frame}

%------------------------------------------------

\begin{frame}
\frametitle{Bounds for Indoor Distances}

\begin{lemma}[\small Object Layer ULBounds]
  Suppose two objects $Q$ and $O$, $Q$'s $\beta_Q$-region is $Q^\beta$, $o$'s $\beta_O$-region is $O^\beta$ we have:
  \begin{equation*}
    \begin{split}
    & (1 - \beta_Q) \beta_O \cdot |Q, O^\beta|_{minI} + \beta_Q \beta_O \cdot |Q^\beta, O^\beta|_{minI} + \\
    & (1 - \beta_Q) (1 - \beta_O) \cdot |Q, O|_{minI} + \beta_Q (1 - \beta_O) \cdot |Q^\beta, O|_{minI} \\
    & \leq |Q, O|_I  \\
    & (1 - \beta_Q) \beta_O \cdot |Q, O^\beta|_{maxI} + \beta_Q \beta_O \cdot |Q^\beta, O^\beta|_{maxI} + \\
    & (1 - \beta_Q) (1 - \beta_O) \cdot |Q, O|_{maxI} + \beta_Q (1 - \beta_O) \cdot |Q^\beta, O|_{maxI}
    \end{split}
  \end{equation*}
\end{lemma}

\end{frame}

%------------------------------------------------

\begin{frame}
\frametitle{Bounds for Indoor Distances}

\begin{proof}{\small}
  Assume $q$ is a point inside $Q$, we have:
  \begin{equation*}
    \begin{split}
      & E(|Q, O|) = E(|Q,O|_I ~|~ q \in Q^\beta) \cdot Pr\{q \in Q^\beta\} + \\
      & E(|q,O|_I ~|~ q \in Q \setminus Q^\beta) \cdot Pr\{q \notin Q^\beta\} = \\
      & E(|Q,O|_I ~|~ q \in Q^\beta) \cdot \beta_Q + E(|q,O|_I ~|~ q \in Q \setminus Q^\beta) \cdot (1 - \beta_Q)
    \end{split}
  \end{equation*}
  for Case $q \in Q^\beta$, $E(|q, O|_I ~|~ q \in Q^\beta) \cdot \beta_Q \leq \beta_Q \beta_O \cdot |Q^\beta, O^\beta|_{maxI} + \beta_Q (1-\beta_O) \cdot |Q^\beta, O|_{maxI}$.\\
  for Case $q \in Q \setminus Q^\beta$, $E(|q, O|_I ~|~ q \in Q \setminus Q^\beta) \cdot (1 - \beta_Q) \leq (1 - \beta_Q) \beta_O \cdot |Q, O^\beta|_{maxI} + (1 - \beta_Q) (1-\beta_O) \cdot |Q, O|_{maxI}$.\\
  Summarize the two cases, we can get the conclusion. The lower bound part can be proved in a similar way.
\end{proof}

\end{frame}

%------------------------------------------------

\begin{frame}
\frametitle{Bounds for Indoor Distances}

\centering
\begin{tabular}{|c|c|}
\hline
\textbf{Cases} & Bounds \\
\hline
\tabincell{c}{single-partitioned region \\ star-connected region} & \tabincell{c}{Geometric Layer ULBounds \\ Topological Layer ULBounds} \\
\hline
\tabincell{c}{multi-partitioned region \\ big uncertainty region} & Object Layer ULBounds \\
\hline
\end{tabular}

\begin{itemize}
  \item to use geometric and topological ULBounds for the case that an object overlaps a single partition.
  \item to use probabilistic ULBounds for the case that an object overlaps with multiple partitions.
\end{itemize}

\end{frame}

%------------------------------------------------

\begin{frame}
\frametitle{Composite Index for Indoor Space}

\begin{columns}[c]

  \column{0.5\textwidth}
  \begin{figure}[tb]
    \includegraphics[width=\columnwidth]{figures/2-6/2-6-8.pdf}
  \end{figure}

  \column{0.5\textwidth}
  \begin{fitemize}
    \item \conceptbf{geometric layer} consists of a tree structure that adapts the R$^*$-tree to index all partitions, as well as a skeleton tier that maintains a small number of distances between staircases.
    \item \conceptbf{topological layer} maintains the connectivity information between indoor partitions.
    \item \conceptbf{object layer} stores all indoor moving objects and is associated with the tree through partitions at its leaf level.
  \end{fitemize}

\end{columns}

\end{frame}

%------------------------------------------------

\begin{frame}
\frametitle{Composite Index: Overview}

\begin{figure}[tb]
  \includegraphics[width=\columnwidth]{figures/2-6/2-6-9.pdf}
\end{figure}

\end{frame}

%------------------------------------------------

\begin{frame}
\frametitle{Composite Index: Tree Tier}

\begin{fitemize}
  \item instead of 3D $Minimum Bounding Rectangle$, when creating the tree, set the vertical length for one partition to 1 centimeter. Two advantage: 1) reduce the distance calculation workload; 2) makes the distance reflected in the tree more accurate without the disturbance from the vertical dimension.
  \item the imbalanced partition are decomposed to small but regular region, each is called an \emph{index unit}.
  \item A hash table is used to map such an index unit to its original indoor partition.
  \item in addition to the MBRs, a leaf node (index unit) also stores:
    \begin{sitemize}
      \item a linked bucket for all objects inside it (Object Layer)
      \item links to its connected partitions (Topological Layer)
    \end{sitemize}
\end{fitemize}

\end{frame}

%------------------------------------------------

\begin{frame}
\frametitle{Composite Index: Tree Tier - Augmented Tree Tier}

\textbf{Basic Idea} of performing a spatial join is \emph{to use the property that the MBR of an index node covers the MBRs of its subtree}.

\vspace{20pt}

To maintain the partial order property taht eases join processing, it augments each tree node $t$ with two attributes, $\{ t.r_{max}, t.r_{count} \}$.
\begin{fitemize}
  \item measure an object's size by the length of its MBR's longest dimension
  \item $t.r_{max}$ to represent the largest object size of $t$'s subtree
  \item $t.r_{count}$ to represent the number of objects associated with $t$'s subtree
\end{fitemize}

\vspace{20pt}

Consequently, the \emph{augmented area} of $t$ is the Minkovski sum of $t$'s sum of $t$'s MBR and its $r_{max}$, denoted by $t \oplus t.r_{max}$.

\end{frame}

%------------------------------------------------

\begin{frame}
\frametitle{Composite Index: Object Tier}

A hash table $o-table$

\begin{equation*}
  o-table : \{ O \} \rightarrow 2^{\{index~unit\}}
\end{equation*}

$o-table$ maps an object to all the index units it overlaps, and it is tightly tie up with the tree tier.\\~\\

When an object update occurs, $o-table$ needs to be updated accordingly.

\end{frame}

%------------------------------------------------

\begin{frame}
\frametitle{Composite Index: Topological Tier}

This layer maintains the connectivity between partitions. Each leaf node stores a (sub)partition.\\~\\

For accessibility, the doors belonging to the partitions are also stored, as well as the the links to accessible partitions through each door.

\end{frame}

%------------------------------------------------

\begin{frame}
\frametitle{Composite Index: Skeleton Tier}

Skeleton Tier is a graph, each staircase entrance is captured as a graph node, and an edge connects two nodes if their entrances are on the same floor or their entrances belong to the same staircase.\\~\\

The weight of an edge is the indoor distance between the two staircase entrances.

\vspace{-10pt}
\begin{columns}[c]

  \column{0.4\textwidth}
  \begin{figure}[tb]
    \includegraphics[width=\columnwidth]{figures/2-6/2-6-11.pdf}
  \end{figure}

  \column{0.6\textwidth}
  \ssize{
  \begin{definition}[staircase distance matrix $M_{s2s}$]
    \begin{sitemize}
      \item $M_{s2s}[s_i,s_i] = 0$;
      \item $M_{s2s}[s_i,s_j] = |s_i, s_j|_E$ if $s_i$ and $s_j$ are on the same floor;
      \item if $s_i$ and $s_j$ are of a same staircase, $M_{s2s}[s_i,s_j]$ is the shortest distance from $s_i$ to $s_j$ within that staircase;
      \item $M_{s2s}[s_i,s_j]$ is calculated as the shortest path distance from $s_i$ to $s_j$ in the skeleton layer for other cases.
    \end{sitemize}
  \end{definition}
  }

\end{columns}

\end{frame}

%------------------------------------------------

\begin{frame}
\frametitle{Skeleton Distance}

\textrm{Let $q$ be a fixed indoor point, $q.f$ the floor of $q$, and $S(q.f)$ all the staircases on floor $q.f$.}

\vspace{10pt}
\begin{definition}[Skeleton Distance]
  Given two points $p$ and $q$, their skeleton distance $|q,p|_K = |q,p|_E$ if they are on the same floor; otherwise, $|q,p|_K = \min_{s_q \in S(q.f), s_p \in S(p.f)}(|q,s_q|_E + M_{s2s}[s_q,s_p] + |s_p, p|_E)$.
\end{definition}

\vspace{10pt}
Define the skeleton distance as the alternative \emph{Geometric Distance}.

\end{frame}

%------------------------------------------------

\begin{frame}
\frametitle{Indoor Distance Bounds in the Geometric Layer}

\begin{lemma}[Geometric Lower Bound Property]
  Given two points $p$ and $q$, their skeleton distance lower bounds their indoor distance, i.e., $|q,p|_K \leq |q,p|_I$.\\~\\
  \textbf{Proof:}~If $q$ and $p$ are on the same floor, $|q,p|_K = |q,p|_E \leq |q,p|_I$. Otherwise, suppose $s_{q}^{*} \in S(q.f)$ and $s_{p}^{*} \in S(p.f)$ are on the shortest path from $q$ to $p$, denoted by $q \overset{*s_{q}^{*}*s_{p}^{*}}{\rightarrow} p$. Since $|q,p|_K = \min_{s_q \in S(q.f), s_p \in S(p.f)}(|q,s_q|_E + M_{s2s}[s_q,s_p] + |s_p, p|_E) \leq |q,s_{q}^{*}|_E + M_{s2s}[s_{q}^{*},s_{p}^{*}] + |s_{p}^{*}, p|_E = |q,p|_I$, the lemma is proved.
\end{lemma}

\end{frame}

%------------------------------------------------

\begin{frame}
\frametitle{Indoor Distance Bounds in the Geometric Layer}

Consider an entity $e$ that is either an object or an $ind$R-tree node. If $e$ spans multiple floors, we use interval $[e.lf,e.uf]$ to represent all those floors. Note those floors must be consecutive. We define the minimum skeleton distance $|q,e|_{minK}$:

\begin{figure}[tb]
  \includegraphics[width=0.7\columnwidth]{figures/2-6/2-6-12.pdf}
\end{figure}

With $|q,e|_{minK}$, one can constrain the search via the $ind$R-tree to a much smaller range compared to if use the Euclidean distance bounds.

\end{frame}

% %------------------------------------------------

\begin{frame}[allowframebreaks]
\frametitle{References}

\begin{thebibliography}{99} % Beamer does not support BibTeX so references must be inserted manually as below
\bibliographystyle{abbrv}
\fsize{

% \bibitem{DBLP:conf/mdm/JensenLY09}
% C.~S. Jensen, H.~Lu, and B.~Yang.
% \newblock Graph model based indoor tracking.
% \newblock In {\em {MDM}}, pp. 122--131, 2009.

% \bibitem{DBLP:conf/cikm/YangLJ09}
% B.~Yang, H.~Lu, and C.~S. Jensen.
% \newblock Scalable continuous range monitoring of moving objects in symbolic
%   indoor space.
% \newblock In {\em CIKM}, pp. 671--680, 2009.

% \bibitem{DBLP:conf/edbt/YangLJ10}
% B.~Yang, H.~Lu, and C.~S. Jensen.
% \newblock Probabilistic threshold k nearest neighbor queries over moving
%   objects in symbolic indoor space.
% \newblock In {\em EDBT}, pp. 335--346, 2010.

% \bibitem{DBLP:conf/icde/LuYCJ11}
% H.~Lu, B.~Yang, and C.~S. Jensen.
% \newblock Spatio-temporal Joins on Symbolic Indoor Tracking Data.
% \newblock In {\em {ICDE}}, pp. 816--827, 2011.

% \bibitem{jensen2010indoor}
% C.~S. Jensen, H.~Lu and B.~Yang.
% \newblock Indoor-A New Data Management Frontier.
% \newblock In {\em IEEE Data Eng. Bull.}, pp. 12--17, 2010.

% \bibitem{DBLP:conf/icde/LuCJ12}
% H.~Lu, X.~Cao, and C.~S. Jensen.
% \newblock A foundation for efficient indoor distance-aware query processing.
% \newblock In {\em ICDE}, pp. 438--449, 2012.
%
% \bibitem{DBLP:conf/icde/XieLP13}
% X.~Xie, H.~Lu, and T.~B. Pedersen.
% \newblock Efficient distance-aware query evaluation on indoor moving objects.
% \newblock In {\em ICDE}, pp. 434--445, 2013.

% \bibitem{becker2005location}
% C.~Becker and F.~D{\"u}rr.
% \newblock On location models for ubiquitous computing.
% \newblock In {\em Personal and Ubiquitous Computing}, pp. 20--31, 2005.

% \bibitem{li2008lattice}
% D.~Li and D.~L.~Lee.
% \newblock A lattice-based semantic location model for indoor navigation.
% \newblock In {\em MDM}, pp. 17--24, 2008.

% \bibitem{becker2009multilayered}
% T.~Becker, C.~Nagel and T.~H.~Kolbe
% \newblock A multilayered space-event model for navigation in indoor spaces.
% \newblock In {\em 3D Geo-Information Sciences}, pp. 61--77, 2009.

% \bibitem{pfoser1999capturing}
% D.~Pfoser and C.~S. Jensen.
% \newblock Capturing the uncertainty of moving-object representations
% \newblock In {\em Advances in Spatial Databases}, pp. 111--131, 1999.
%
% \bibitem{cheng2003evaluating}
% R.~Cheng, D.~V.~Kalashnikov and S.Prabhakar.
% \newblock Evaluating probabilistic queries over imprecise data.
% \newblock In {\em SIGMOD}, pp. 551--562, 2003.
%
% \bibitem{cheng2004querying}
% R.~Cheng, D.~V.~Kalashnikov and S.Prabhakar.
% \newblock Querying imprecise data in moving object environments.
% \newblock In {\em TKDE}, pp. 1112--1127, 2004.
%
% \bibitem{kriegel2007probabilistic}
% H.-P.~Kriegel,P.~Kunath and M.~Renz.
% \newblock Probabilistic nearest-neighbor query on uncertain objects.
% \newblock In {\em DSAFAA}, pp. 337--348, 2007.

\bibitem{xie2015distance}
X.~Xie, H.~Lu, and T.~B. Pedersen.
\newblock Distance-aware join for indoor moving objects.
\newblock In {\em TKDE}, pp. 428--442, 2015.

}
\end{thebibliography}

\end{frame}


\subsection{2.8 Extracting Indoor Spatial Objects from CAD Models} % A subsection can be created just before a set of slides with a common theme to further break down your presentation into chunks

% \begin{frame}
\frametitle{About This Work...}

\emph{Extracting Indoor Spatial Objects from CAD Models: A Database Approach}.~\cite{xu2015extracting} \\
D.~Xu, P.~Jin, X.~Zhang, J.~Du, and L.~Yue.\\~\\

\begin{itemize}
  \item propose a database approach to extracting indoor spatial objects from CAD models.
  \item further integrate them into an indoor moving-object database.
\end{itemize}

\end{frame}

%------------------------------------------------

\begin{frame}
\frametitle{Motivation}

\begin{itemize}
  \item A fundamental issue in indoor moving-object management is the construction of indoor-space maps
    \begin{fitemize}
      \item outdoor spaces have Google Maps and city road-network maps
      \item for indoor spaces, there are no existing solutions for automatically generating indoor maps
      \item manually draw the floor plan of an indoor space is time-consuming and costly
    \end{fitemize}

  \item present a database approach to automatically extract indoor spatial objects and generate indoor maps from CAD models
  \begin{fitemize}
    \item CAD is commonly used in indoor-space design
    \item resulting in a great number of CAD files depicting structure of buildings
  \end{fitemize}
\end{itemize}

\end{frame}

%------------------------------------------------

\begin{frame}
\frametitle{Drawing Exchange Format File}

CAD models are typically represented by the \conceptbf{Drawing Exchange Format}(DXF)~\cite{autoc77:online}.\\~\\

A DXF files stores in a key-value style: all the k-v pairs are organized into seven sections, namely \emph{HEADER}, \emph{CLASSES}, \emph{TABLES}, \emph{BLOCKS}, \emph{ENTITIES}, \emph{OBJECTS}, \emph{THUMBNAILIMAGE}.\\~\\

\emph{ENTITIES} section contains all graphical objects in the drawing.

\end{frame}

%------------------------------------------------

\begin{frame}
\frametitle{Data in Drawing Exchange Format File}

original data in DXF files are low-level graphical elements including \emph{points}, \emph{lines}, \emph{arcs}, \emph{ploylines} and \emph{circles}.\\~\\

However, one indoor spatial object may involve many graphical elements. Due to the large number of graphical elements in a file, it is not trivial to find the right elements that describe an ondoor spatial object.

\begin{columns}

  \column{0.45\textwidth}
  \begin{figure}[tb]
    \includegraphics[width=0.9\columnwidth]{figures/2-8/2-8-1.pdf}
  \end{figure}

  \column{0.55\textwidth}
  \fsize{\textrm{in the example right, the walls of a room are separated by some arcs, lines and white squares, it is not feasible to simply extract rooms from CAD models}}

\end{columns}

\end{frame}

%------------------------------------------------

\begin{frame}
\frametitle{Problem State}

\begin{problem}[Extracting CAD Spatial Objects]
  Given a CAD model (DXF file), return a set of rooms, doors, and the topological relationship between doors and rooms, where each room is represented as a ploygon, each door is represented as a point, and each topological relationship is a pair of (door, room) indicating that a door is connected with a room.
\end{problem}

\end{frame}

%------------------------------------------------

\begin{frame}
\frametitle{The Proposed Method: Basic Idea}

\textbf{DOORS}~~As doors are usually represented by arcs, just simply extract all arcs as door candidates and further make a refinement to remove those candidates that are not connected with rooms (after rooms have been recognized)\\~\\

\textbf{ROOMS}~~A rule-based approach is used, considering the following two problems: 1. a room's wall may involve too many lines in CAD models; 2. rooms are not always rectangles and ther may be more than one door associated.\\~\\

\textbf{BASIC IDEA}~~to first employ a line-reduction preprocessing to simplify the line set that composed of a room; after that, use a line-extending technique to divide the entire indoor space into a set of geometric shapes; finally, use an MBR-based method to remove duplicated rooms.

\end{frame}

%------------------------------------------------

\begin{frame}
\frametitle{The Proposed Method: Difinitions}

\begin{definition}[point-point adjacency]
  Two points are adjacent if their Euclidean distance is below a predefined threshold $\alpha$
  \begin{equation}
    p \overset{adj}{\leftrightarrow} q \Leftrightarrow dist(p,q) \leq \alpha
  \end{equation}
\end{definition}

\begin{definition}[point-line adjacency]
  A point $p$ is adjacent to a line $ln(s,e)$ if there is a \emph{point-point adjacency} relation between $ln$'s start point or end point and $p$
  \begin{equation}
    p \overset{adj}{\leftrightarrow} ln \Leftrightarrow (p \overset{adj}{\leftrightarrow} s \textbf ~or~ p \overset{adj}{\leftrightarrow} s)
  \end{equation}
\end{definition}

\end{frame}

%------------------------------------------------

\begin{frame}
\frametitle{The Proposed Method: Difinitions}

\begin{definition}[line-line adjacency]
  Let $ln_1(s_1, e_1)$ and $ln_2(s_2, e_2)$ be two lines. $ln_1$ and $ln_2$ are adjancent if there is a \emph{point-line adjacency} relation between an end point of one of the two lines and another line.
  \begin{equation}
    ln_1 \overset{adj}{\leftrightarrow} ln_2 \Leftrightarrow (s_1 \overset{adj}{\leftrightarrow} ln_2 ~or~e_1 \overset{adj}{\leftrightarrow} ln_2) ~or~ (s_2 \overset{adj}{\leftrightarrow} ln_1 ~or~e_2 \overset{adj}{\leftrightarrow} ln_1)
  \end{equation}
\end{definition}

\begin{definition}[parallel lines]
  Let $ln_1(s_1, e_1)$ and $ln_2(s_2, e_2)$ be two lines. $ln_1$ and $ln_2$ are parallel such that they satisfy the following condition.
  \begin{equation}
    ln_1 \overset{par}{\leftrightarrow} ln_2 \Leftrightarrow dot\_product(ln_1, ln_2) - dist(s_1, e_1) \cdot dist(s_2,e_2) < \beta
  \end{equation}
\end{definition}

\end{frame}

%------------------------------------------------

\begin{frame}
\frametitle{The Proposed Method: Difinitions}

Distance between two lines are computed as follows
\begin{figure}[tb]
  \includegraphics[width=\columnwidth]{figures/2-8/2-8-2.pdf}
\end{figure}

\begin{definition}[near lines]
  Two parallel lines are near if the distance between them is below a predefined threshold $\varepsilon$
\end{definition}

\end{frame}

%------------------------------------------------

\begin{frame}
\frametitle{The Proposed Method: Algorithm}

\begin{columns}

  \column{0.58\textwidth}
  \begin{figure}[tb]
    \includegraphics[width=\columnwidth]{figures/2-8/2-8-3.pdf}
  \end{figure}


  \column{0.42\textwidth}
  \ssize{
  \begin{enumerate}
    \item first obtain line clusters through a near-line-based clustering step
    \item simplify the lines in each cluster by merging near lines into a single line
    \item construct polygons using a line-growing technique, and then remove duplicated polygons by constructing MBRs for all polygons
    \item after extracting rooms, continue to refine door candidates that are extracted from arcs in the CAD model
    \item those candidates not connected with any rooms are removed from the door set.
  \end{enumerate}
  }

\end{columns}

\end{frame}

%------------------------------------------------

\begin{frame}
\frametitle{Integration into Indoor Moving-Object Databases}

Current relational database systems do not provide inherited support for complex object representation and storage. \\~\\

Object-relational database systems such as Oracle and PostgreSQL offer user-defined type extensions. \\~\\

Three indoor spatial types on Oracle, which are \emph{Indoor Position}, \emph{Indoor Space}, \emph{Indoor Geometry}. Oracle Spatial provides \emph{SDO\_GEOMETRY} for representing spatial data.

\end{frame}

%------------------------------------------------

\begin{frame}
\frametitle{Integration into Indoor Moving-Object Databases}

\textbf{Indoor Space} type describes rooms, doors, as well as the topological relationship between doors and rooms, representing as a tripe $(RoomSet, DoorSet, Room-Door)$. Where $RoomSet$ is the set of room identifiers, $DoorSet$ is the set of door identifiers, and $Room-Door$ represents the topological relationship between $RoomSet$ and $DoorSet$.\\~\\

\textbf{Indoor Geometry} type stores the exact geometric information about rooms and doors with the form $(ObjectID, Geo)$, where $ObjectID$ is a room ID or a door ID and $Geo$ is an \emph{SDO\_GEOMETRY} type supported by Oracle Spatial.

\end{frame}

% %------------------------------------------------

\begin{frame}[allowframebreaks]
\frametitle{References}

\begin{thebibliography}{99} % Beamer does not support BibTeX so references must be inserted manually as below
\bibliographystyle{abbrv}
\fsize{

\bibitem{xu2015extracting}
D.~Xu, P.~Jin, X.~Zhang, J.~Du, and L.~Yue.
\newblock Extracting Indoor Spatial Objects from CAD Models: A Database Approach.
\newblock In {\em Database Systems for Advanced Applications}, pp. 273--279, 2015.

\bibitem{autoc77:online}
\url{http://images.autodesk.com/adsk/files/autocad_2012_pdf_dxf-reference_enu.pdf}.
\newblock DXF Reference (2012).
\newblock Accessed on 05/12/2016.

}
\end{thebibliography}

\end{frame}


%------------------------------------------------
\section{3. Indoor Data Cleansing} % Sections can be created in order to organize your presentation into discrete blocks, all sections and subsections are automatically printed in the table of contents as an overview of the talk
%------------------------------------------------

\subsection{3.1 Leveraging Spatio-Temporal Redundancy for RFID Data Cleansing} % A subsection can be created just before a set of slides with a common theme to further break down your presentation into chunks

% \begin{frame}
\frametitle{About This Work...}

\emph{Leveraging Spatio-Temporal Redundancy for RFID Data Cleansing}.~\cite{chen2010leveraging} \\
H.~Chen, W-S.~Ku, H.~W and M-T.~S.\\~\\

\begin{itemize}
  \item Published at \emph{SIGMOD' 2010}.
  \item Proposed a Bayesian inference based approach for cleaning RFID raw data.
  \item Desined an $n$-state detection model to capture the likelihood.
  \item Devised a Metropolis-Hastings sampler with Constraints (MH-C) to sample from the posterior.
\end{itemize}

\end{frame}

%------------------------------------------------

\begin{frame}
\frametitle{Motivation}

To take full advantage of:

\begin{itemize}
  \item duplicate readings (by multiple readers simultaneously or by a single reader over a period of time) of the same object are very common.
  \item prior knowledge about the readers and the environment (e.g., prior data distribution, false negative rates of readers) may help improve data quality and remove data anomalies.
  \item given constraints in target applications (e.g., the number of objects in a same location cannot exceed a given value).
\end{itemize}

\end{frame}

%------------------------------------------------

\begin{frame}
\frametitle{Data Redundancy: Spatial Redundancy}

\fsize{\textrm{The challenge is how to take advantage of redundancy while avoiding its undesirable effect in data cleansing.}}

\vspace{-10pt}
\begin{columns}[c]

  \column{0.5\textwidth}
  \begin{figure}[tb]
    \includegraphics[width=\columnwidth]{figures/3-1/3-1-2.pdf}
  \end{figure}

  \begin{example}
    \ssize{
    \textrm{
    the target area is divided into 6 zones, an RFID reader is located in the center of each zone. Spatial overlap of readers' detection regions leads to duplicate readings, i.e., an object is in the detection regions of multiple readers.
    }}
  \end{example}

  \column{0.5\textwidth}
  \begin{figure}[tb]
    \includegraphics[width=\columnwidth]{figures/3-1/3-1-1.pdf}
  \end{figure}

  \ssize{
  The above table shows two effects of redundancy:
  \begin{enumerate}
    \item Object 2 is detected by both readers in Zone 2 and 3, at least one of the readings belongs to spatial redundancy.
    \item Object 3 is detected in Zone 4 only, however, it does not necessarily mean that the Object 3 is in Zone 4 for sure.
  \end{enumerate}
  }

\end{columns}

\end{frame}

%------------------------------------------------

\begin{frame}
\frametitle{Data Redundancy: Temporal Redundancy}

Many applications monitor the target area using a \emph{mobile reader} instead of employing multiple \emph{stationary readers}.\\~\\

Because the exact location of the mobile reader is always changing, the detection regions at different time points may overlap.\\~\\

\textbf{The temporal redundancy problem can be reduced to the spatial redundancy problem:} by treating the same reader at different time points as different readers.

\end{frame}

%------------------------------------------------

\begin{frame}
\frametitle{Prior Knowledge}

As false negatives and false positives abound in raw RFID readings, in order to revoer the true information, the data cleasing system should take prior knowledge into account.\\~\\

For example, the detection areas of readers in Zone 2 and 3 have significant overlapping, the positioning of the reader in Zone 4 makes it more likely to detect objects in Zone 3 than objects in Zone 5, or the reader in Zone 3 has high false negative rate.

\end{frame}

%------------------------------------------------

\begin{frame}
\frametitle{Constraints}

\emph{Environmental constraints} can be utilized to improved data cleansing.\\~\\

For example, the maximal capacity of each zone (the number of objects that can reside the same zone) is a constraint.\\~\\

In addition to these pysical constraints, information obtained from other channels can be translated into constraints. E.g., if an extra source indicates that two certain objects are in the same zone, it may help cleanse the data of there two.

\end{frame}

%------------------------------------------------

\begin{frame}
\frametitle{Overview of the Approach}

\begin{enumerate}
  \item By using Bayesian inference, it derives a universal framework of computing the posterior probabilities (of the location of each object).
  \item Based on the physical characteristic of RFID readers, it proposes an $n$-state detection model to capture likelihoods.
  \item It devised MH-C, an improved \emph{Metropolis-Hastings} sampler, to sample from the posterior while taking the environmental constraints into consideration.
\end{enumerate}

\end{frame}

%------------------------------------------------

\begin{frame}
\frametitle{Notations}

\begin{figure}[tb]
  \includegraphics[width=\columnwidth]{figures/3-1/3-1-3.pdf}
\end{figure}

\end{frame}

%------------------------------------------------

\begin{frame}
\frametitle{A Bayesian Interface Base Approach}

\conceptbf{Bayesian Interface} estimates the probalility of a hypothesis $(x)$ based on observations $(y)$, showing that posterior is proportional to the multiplication of likelihood and prior, i.e., $p(x|y) \propto p(y|x)p(x)$.\\~\\\pause

Suppose there're $m$ zones (each with a reader mounted in the center) and $n$ objects. For each object $o_i$, its location is represented by a random variable $h_i$. A possible distribution of $n$ objects in $m$ zones can be denoted as an instance of the random vector:\pause
\begin{equation}
  \hat{H} = \{ h_1, h_2, ..., h_n \}
\end{equation}
\pause

where $h_i$ represents the zone ID where object $o_i$ is in. E.g., $h_1 = 2$ denotes that object $o_1$ is in zone 2 in the current instance.

\end{frame}

%------------------------------------------------

\begin{frame}
\frametitle{A Bayesian Interface Base Approach}

For the reader in zone $j$, the raw data (0 or 1) it receives from the RFID tag of objects $o_i$ is denoted as $z_{ij}$. Thr \emph{raw data matrix} for each complete scan from $m$ readers can then be represented as an $n \times m$ matrix $\mathbb{Z} = [z_{ij}]$. \\~\\ \pause

Using the \conceptbf{Bayesian theorem}, where $post(\hat{H}|\mathbb{Z})$ denotes the posterior probability of location vector $\hat{H}$ given the raw data matrix $\mathbb{Z}$. The hypothesis should satisfy all constraints: \pause
\begin{equation}
  \begin{split}
  post(\hat{H}|\mathbb{Z}) = 0 & :\hat{H} \textrm{ is not valid} \\
  post(\hat{H}|\mathbb{Z}) > 0 & :\hat{H} \textrm{ is valid} \\
  post(\hat{H_1}|\mathbb{Z}) > post(\hat{H_2}|\mathbb{Z}) & :\hat{H_1} \textrm{ is more likely than } \hat{H_2}
  \end{split}
\end{equation}

\end{frame}

% %------------------------------------------------

\begin{frame}[allowframebreaks]
\frametitle{References}

\begin{thebibliography}{99} % Beamer does not support BibTeX so references must be inserted manually as below
\bibliographystyle{abbrv}
\fsize{

\bibitem{chen2010leveraging}
H.~Chen, W-S.~Ku, H.~W and M-T.~S.
\newblock Leveraging spatio-temporal redundancy for RFID data cleansing.
\newblock In {\em SIGMOD}, pp. 51--62, 2010.

\bibitem{jeffery2006adaptive}
S.R.~Shawnm, M.~Garofalakis, M.J.~Franklin.
\newblock Adaptive cleaning for RFID data streams.
\newblock In {\em VLDB}, pp. 163--174, 2006.

\bibitem{rao2006deferred}
J.~Rao, S.~Doraiswamy, H.~Thakkar, L.S.~Latha.
\newblock A deferred cleansing method for RFID data analytics.
\newblock In {\em VLDB}, pp. 175--186, 2006.

\bibitem{andrieu2003introduction}
C.~Andrieu, N.~De~Freitas, A.~Doucet, M.I.~Jordan.
\newblock An introduction to MCMC for machine learning.
\newblock In {\em Machine learning}, pp. 5--43, 2003.

}
\end{thebibliography}

\end{frame}


\subsection{3.2 Spatiotemporal Data Cleansing for Indoor RFID Tracking Data} % A subsection can be created just before a set of slides with a common theme to further break down your presentation into chunks

% \begin{frame}
\frametitle{About This Work...}

\emph{Spatiotemporal Cleansing for Indoor RFID Tracking Data}.~\cite{baba2013spatiotemporal} \\
A.I.~Baba, H.~Lu, X.~Xie, T.B.~Pedersen\\~\\

\begin{itemize}
  \item Published at \emph{MDM' 2013}.
  \item Focused on two quality aspects in raw indoor RFID data, temporal redundancy and spatial ambiguity.
  \item Investigated the spatiotemporal characteristics of indoor spaces as well as RFID reader deployment to be exploited in cleansing.
\end{itemize}

\end{frame}

%------------------------------------------------

\begin{frame}
\frametitle{Motivation}

\begin{itemize}
  \item Effective RFID tracking data management is expected to support various applications.
  \begin{sitemize}
    \item range from monitoring to analysis of indoor moving objects
    \item RFID reader reports the object's presence to the database that manages the object positions
  \end{sitemize}
  \item However, noises and errors abound in raw data.
  \begin{sitemize}
    \item radio frequency waves are not steady and therefore the detection range may change from time to time
    \item such dirtiness hinders the progress of applying meaningful high level application.
  \end{sitemize}
  \item Cleansing RFID data is therefore needed.
\end{itemize}

\end{frame}

%------------------------------------------------

\begin{frame}
\frametitle{Dirtiness in Indoor RFID Tracking Data}

\textrm{An RFID reader report $(readerID, objectID, time)$ means the object identified by $objectID$ is seen by the reader identified by $readerID$ at time point $time$}.\\~\\

\conceptbf{Temporal Redundancy}~~~~A tagged object can be read many times by the same reader within a short period, depending on the sampling frequency configured for a reader.\\~\\

\conceptbf{Spatial Ambiguity}~~~~A tagged object can be read by multiple readers simultaneously. This may result from the unexpected change of the detection range of a reader nearby.

\end{frame}

%------------------------------------------------

\begin{frame}
\frametitle{Dirtiness in Indoor RFID Tracking Data}

\begin{figure}[tb]
  \includegraphics[width=0.6\columnwidth]{figures/3-2/3-2-1.pdf}
\end{figure}

\vspace{-15pt}
\begin{columns}

  \column{0.5\textwidth}
  \begin{example}
    \ssize{
    Two readers $r_1$, $r_2$ in two rooms respectively. Object $O_1$ moves in the left room from time point $t_1$ to time point $t_5$, which yields five reports by $r_1$ as the trajectory is within $r_1$'s detection range. If times points are very close to each other, the five reports can be compressed into a single tuple $\langle r_1, O_1, [t_1,t_5] \rangle$.
    }
  \end{example}

  \column{0.5\textwidth}
  \begin{example}
    \ssize{
    At time point $t_3$, object $O_1$ is detected by both readers. This is due to the unexpected expansion of $r_2$'s detection range. Consequently, from the RFID data, $O_1$ seems to be in both rooms at same time $t_3$. Thus spatial ambiguity is caused. By considering such spatiotemporal constraints, our cleansing technique is able to remove spatial ambiguous reports such as $\langle r_2, O_1, t_3 \rangle$.
    }
  \end{example}

\end{columns}

\end{frame}

%------------------------------------------------

\begin{frame}
\frametitle{Raw Readings Table (RRT)}

\begin{columns}[c]

  \column{0.55\textwidth}
  \begin{figure}[tb]
    \includegraphics[width=\columnwidth]{figures/3-2/3-2-2.pdf}
  \end{figure}

  \column{0.45\textwidth}
  \begin{sitemize}
    \item each raw reading is in the format of $(deviceID, objectID, t)$, which means that the object identified by $objectID$ is detected by the device identified by $deviceID$ at time $t$.
    \item usually the detection range of a positioning device is a circular region with a pre-specified radius, a positioning device continuously detects objects that are in its range, with the frequency determined by its sampling rate.
  \end{sitemize}

\end{columns}

\end{frame}

%------------------------------------------------

\begin{frame}
\frametitle{Compared with Graph Based Indoor Tracking~\cite{DBLP:conf/mdm/JensenLY09}}

This paper distinguishes itself from the work~\cite{DBLP:conf/mdm/JensenLY09} introduced in Section 2.1:

\begin{itemize}
  \item This paper considers the overlapping between different positioning devices, whereas work~\cite{DBLP:conf/mdm/JensenLY09} assumes that devices do not overlap.
  \item This paper is intended to decide where an object really is when is is seen by multiple devices, whereas work~\cite{DBLP:conf/mdm/JensenLY09} focuses on tracking the object when it is not seen by any devices.
  \item In order to support data cleansing, this paper proposes a \emph{Distance-Aware Graph} different from the \emph{Deployment Graph} introduced in Section 2.
\end{itemize}

\end{frame}

%------------------------------------------------

\begin{frame}
\frametitle{Notations}

\begin{figure}[tb]
  \includegraphics[width=\columnwidth]{figures/3-2/3-2-3.pdf}
\end{figure}

\end{frame}

%------------------------------------------------

\begin{frame}
\frametitle{Definitions and Tasks}

\begin{definition}[Temporal Redundant Readings]
  Two raw readings $rr_i$ and $rr_j$ are temporal redundant readings if $|rr_i.t - rr_j.t| \leq \tau$, $rr_i.deviceID = rr_j.deviceID$ and $rr_i.objectID = rr_j.objectID$, where $\tau$ is an application-specific threshold.
\end{definition}

\begin{definition}[Tracking Record]
  Given a series of temporal redundant readings $rr_1, ..., rr_k$, a tracking record $tr$ is a temporal aggregation of them. Formally, $tr$ is in the format $(deviceID, objectID, t_s, t_e)$, where $tr.deviceID = rr_i.deviceID$, $tr.objectID = rr_i.objectID$, $tr.t_s = rr_1.t$ and $tr.t_e = tt_k.t$ for $1 \leq i \leq k$.
\end{definition}

\end{frame}

%------------------------------------------------

\begin{frame}
\frametitle{Definitions and Tasks}

\begin{definition}[Spatial Ambiguous Tracking Records]
  Two tracking records $tr'$ and $tr$ are spatial ambiguous if $tr'.deviceID \neq tr.deviceID$ and $tr'.objectID = tr.objectID$, if $tr'.[t_s, t_e] \cap tr.[t_s, t_e] \neq \varnothing $, or if $tr.t_s - tr'.t_e \leq min\_tt$, where $min\_tt$ is the minimum travelling time for an object to move from $tr'$'s device to $tr$'s device.
\end{definition}

\end{frame}

%------------------------------------------------

\begin{frame}
\frametitle{Definitions and Tasks}

\begin{block}{Task (Temporal Redundancy Elimination)}
  This task is to aggregate raw readings into more meaningful tracking records. This way is expected to significantly reduce the data size without any information loss.
\end{block}

\begin{block}{Task (Spatial Ambiguity Reduction)}
  Given a large set of tracking records, this task is to identify spatial ambiguous tracking records and reduce such spatial ambiguities by referring to the spatiotemporal constrains imposed by the positioning device deployment as well as the indoor topology.
\end{block}

\textrm{these two tasks together are called \conceptbf{spatiotemporal data cleansing}}.

\end{frame}

%------------------------------------------------

\begin{frame}
\frametitle{Overview of Spatiotemporal Data Cleansing}

\begin{figure}[tb]
  \includegraphics[width=\columnwidth]{figures/3-2/3-2-4.pdf}
\end{figure}

\end{frame}

%------------------------------------------------

\begin{frame}
\frametitle{Phase 1: Temporal Cleansing}

\begin{columns}[c]

  \column{0.5\textwidth}
  \begin{figure}[tb]
    \includegraphics[width=\columnwidth]{figures/3-2/3-2-2.pdf}
  \end{figure}

  \column{0.5\textwidth}
  \begin{figure}[tb]
    \includegraphics[width=\columnwidth]{figures/3-2/3-2-5.pdf}
  \end{figure}

\end{columns}

\vspace{15pt}
\begin{sitemize}
  \item sequentially scan data from the raw reading table and generates more meaningful tracking records by aggregating on the time.
  \item the aggregation results are controlled by the threshold $\tau$.
  \item all generated tracking records are stored in the \conceptbf{Aggregate Tracking Table} (ATT).
  \item the threshold $\tau$ in the example is 4 time units.
\end{sitemize}

\end{frame}

%------------------------------------------------

\begin{frame}
\frametitle{Phase 2: Spatial Cleansing}

\begin{columns}[c]

  \column{0.5\textwidth}
  \begin{figure}[tb]
    \includegraphics[width=\columnwidth]{figures/3-2/3-2-5.pdf}
  \end{figure}

  \column{0.5\textwidth}
  \begin{figure}[tb]
    \includegraphics[width=\columnwidth]{figures/3-2/3-2-6.pdf}
  \end{figure}

\end{columns}

\vspace{10pt}
\begin{sitemize}
  \item take the aggregate tracking table as input, identify possible spatial ambiguities, and reduce them by a distance-aware graph.
  \item the distance-aware graph is constructed by all positioning devices in the indoor space.
  \item assuming the minimum movement time between device $r_6$ and $r_7$ is one unit, $object_1$'s tracking record $(r_7, object_1, t_3, t_{10})$ is truncated to $(t_7, object_1, t_8, t_{10})$.
  \item it assumes that the first tracking record in $ATT$ is correct, while it is not always true in practice.
\end{sitemize}

\end{frame}

%------------------------------------------------

\begin{frame}
\frametitle{Temporal Cleansing Algorithm}

\begin{columns}[c]

  \column{0.5\textwidth}
  \begin{figure}[tb]
    \includegraphics[width=\columnwidth]{figures/3-2/3-2-7.pdf}
  \end{figure}

  \column{0.5\textwidth}
  \begin{sitemize}
    \item Line 1: starts with the initialization of a new aggregate tracking table $ATT$.
    \item Line 4: for each raw reading $rr$ from $RRT$, all the aggregate tracking records with the same device and object as $rr$ in the current $ATT$ are fetched into set $trs$.
    \item Line 5--10: if $trs$ is not empty and $rr$ is temporally close enough to an existing tracking record $tr$ in $trs$, $tr$'s time interval is extended to $rr.t$ and $rr$ is processed.
    \item Line 11--12: otherwise, $rr$ cannot be combined to any existing tracking record, and therefore a new tracking record is created and inserted.
  \end{sitemize}

\end{columns}

\end{frame}

%------------------------------------------------

\begin{frame}
\frametitle{Temporal Cleansing Algorithm: Threshold Setting}

\begin{equation}
  Threshold(\tau) = \frac{\text{detection range diameter}}{\text{object moving speed}}
\end{equation}

\begin{figure}[tb]
  \includegraphics[width=0.7\columnwidth]{figures/3-2/3-2-8.pdf}
\end{figure}

\end{frame}

%------------------------------------------------

\begin{frame}
\frametitle{Distance-Aware Deployment Graph}

\begin{columns}[c]

  \column{0.62\textwidth}
  \ssize{
  \begin{block}{Distance-Aware Deployment Graph}
    $G_{dd} = (V,E,\mathcal{L}_V,\mathcal{L}_E)$
    \begin{sitemize}
      \item $V$ is a set of vertices, each vertex represents a deployed positioning device $r_i$;
      \item $E$ is the set of edges, where $E = \{ (r_i, r_j) | r_i, r_j \in V \wedge r_i \neq r_j \wedge D2V(r_i) \cap D2V(r_j) \neq \varnothing \}$;
      \item $\mathcal{L}_V : V \rightarrow R$ assigns to a vertex $v_i$ the minimum dwell time that an object $o$ should spend in device $r_i$'s detection range such that $o$ is detected by device $r_i$;
      \item $\mathcal{L}_E : E \rightarrow R \times R $ assigns to a edge $(r_i, r_j)$ the minimum indoor walking distance between two devices $r_i$ and $r_j$ and the maximum speed with which an object can move between them, i.e., $\mathcal{L}_E((r_i,r_j)) = (d_{i,j}, S_{i,j})$.
    \end{sitemize}
  \end{block}
  }

  \column{0.38\textwidth}
  \begin{figure}[tb]
    \includegraphics[width=\columnwidth]{figures/3-2/3-2-9.pdf}
  \end{figure}
  \ssize{\textrm{The goal of \emph{Distance-Aware Deployment Graph} is to enable deriving the minimum travel time from one reader to another.}}

\end{columns}

\end{frame}

%------------------------------------------------

\begin{frame}
\frametitle{Distance-Aware Deployment Graph Construction}

\begin{columns}[c]

  \column{0.5\textwidth}
  \begin{figure}[tb]
    \includegraphics[width=\columnwidth]{figures/3-2/3-2-10.pdf}
  \end{figure}

  \column{0.5\textwidth}
  \begin{sitemize}
    \item Lines 2--6: for each pair of device $r_i$ and $r_j$, the indoor shortest path $P$ is found if the edge $(r_i, r_j)$ is not in the graph yet;
    \item Lines 7--8: if $P$ only contains devices $r_i$ and $r_j$, a new edge is created with the corresponding weights;
    \item Lines 10--14: otherwise each pair of consecutive devices on $P$ are processed likewise.
  \end{sitemize}

\end{columns}

\end{frame}

%------------------------------------------------

\begin{frame}
\frametitle{Spatial Cleansing Algorithm}

\begin{columns}[c]

  \column{0.5\textwidth}
  \begin{figure}[tb]
    \includegraphics[width=\columnwidth]{figures/3-2/3-2-11.pdf}
  \end{figure}
  \ssize{
  \textrm{
  To identify and reduce the possible spatial ambiguity involving two RFID readers $r_s$ and $r_t$, one should first compute the minimum traveling time $(min\_tt(r_s, r_t))$ that a moving object needs to reach from $r_s$ to $r_t$.
  }
  }

  \column{0.5\textwidth}
  \begin{sitemize}
    \item Line 2: for each tracking record $tr$ in $ATT$, check its dwell time $tr.t_e - tr.t_s$;
    \item Line 3: get $tr$'s previous tracking record $tr'$ from $ATT$ that involves the same object and device as $tr$;
    \item Line 6: get the minimum traveling time from $tr'.deviceID$ and $tr.deviceID$;
    \item Line 7: if the idle time between $tr'$ and $tr$ is too short compared to the minimum traveling time plus the minimum dwell time for $tr.deviceID$;
    \item Line 8: truncate $tr.t_s$ accordingly to make sure the idle time is sufficient with respect to the spatiotemporal constraints;
    \item Lines 9--10: delete $tr$ from $ATT$ if its updated dwell time is too short.
  \end{sitemize}

\end{columns}

\end{frame}

%------------------------------------------------

\begin{frame}
\frametitle{Cases of Spatial Cleansing}

\begin{figure}[tb]
  \includegraphics[width=\columnwidth]{figures/3-2/3-2-12.pdf}
\end{figure}

\begin{sitemize}
  \item Case(a): the idle time between two consecutive tracking records is sufficiently long, i.e., longer than the minimum traveling time between devices $r_6$ and $r_7$;
  \item Case(b): it is need to truncate $tr.[t_s, t_e]$, cutting $tr$'s part shown in black, because the idle time between $tr'$ and $tr$ is too short;
  \item Case(c): $tr$ is deleted after the spatial cleansing because its remaining dwell time is too short.
\end{sitemize}

\end{frame}

%------------------------------------------------

\begin{frame}
\frametitle{Conclusions}



It designs a temporal cleansing algorithm to aggregate raw RFID readings temporally such that the data size is compressed without information loss.\\~\\

It designs a spatial cleansing technique: proposing a distance-aware deployment graph to capture the spatiotemporal constrains implied by the deployment of RFID readers as well as the indoor topology.\\~\\

The techniques proposed in this paper also apply to indoor tracking data obtained by other symbolic positioning technologies, like Bluetooth.


\end{frame}

% %------------------------------------------------

\begin{frame}[allowframebreaks]
\frametitle{References}

\begin{thebibliography}{99} % Beamer does not support BibTeX so references must be inserted manually as below
\bibliographystyle{abbrv}
\fsize{

\bibitem{baba2013spatiotemporal}
A.I.~Baba, H.~Lu, X.~Xie, T.B.~Pedersen.
\newblock Spatiotemporal data cleansing for indoor RFID tracking data.
\newblock In {\em MDM}, pp. 187--196, 2013.

\bibitem{chen2010leveraging}
H.~Chen, W-S.~Ku, H.~Wang, M-T.~Sun.
\newblock Leveraging spatio-temporal redundancy for RFID data cleansing.
\newblock In {\em SIGMOD}, pp. 51--62, 2010.

\bibitem{DBLP:conf/mdm/JensenLY09}
C.S.~Jensen, H.~Lu, B.~Yang.
\newblock Graph model based indoor tracking.
\newblock In {\em {MDM}}, pp. 122--131, 2009.

}
\end{thebibliography}

\end{frame}


\subsection{3.3 Handling False Negatives in Indoor RFID Data} % A subsection can be created just before a set of slides with a common theme to further break down your presentation into chunks

% \begin{frame}
\frametitle{About This Work...}

\emph{A Graph Model for False Negative Handling in Indoor RFID Tracking Data}.~\cite{baba2013graph}\\
A.I.~Baba, H.~Lu, T.B.~Pedersen, X.~Xie\\~\\

\emph{Handling False Negatives in Indoor RFID Data}.~\cite{baba2014handling} \\
A.I.~Baba, H.~Lu, T.B.~Pedersen, X.~Xie\\~\\

\begin{itemize}
  \item Published at \emph{SIGSPAITAL' 2013}, \emph{MDM' 2014}.
  \item Focuses on handling \emph{false negatives} which occur when a moving object passes the detection range of an RFID reader but the reader fails to produce any readings.
  \item Proposes the transition probabilities that capture how likely objects move from one RFID reader to another.
\end{itemize}

\end{frame}

%------------------------------------------------

\begin{frame}
\frametitle{Motivation}

\begin{itemize}
  \item RFID emerges to be one of the key technologies to modernize object tracking and monitoring systems in indoor environments.
  \begin{sitemize}
    \item airport baggage tracking
  \end{sitemize}
  \item However, the unreliable nature of raw data captured by readers is a major factor hindering the development of various applications.
  \begin{sitemize}
    \item loss and error rate can be between 30-40\%.~\cite{floerkemeier2004issues}
    \item read events are frequently missed due to the detection ability of a reader, the quality of an RFID tag, and constraints of the environment.~\cite{derakhshan2007rfid}
  \end{sitemize}
  \item Critical to cleanse the RFID raw data, and provide clean data to high level applications to make correct interpretations and analysis of the physical world.
\end{itemize}

\end{frame}

%------------------------------------------------

\begin{frame}
\frametitle{False Negatives}

\begin{figure}[tb]
  \includegraphics[width=0.8\columnwidth]{figures/3-3/3-3-1.pdf}
\end{figure}
\vspace{-10pt}
\begin{example}
  \ssize{
  two readers $r_1$ and $r_2$ in one hall and $r_3$ in another hall. Object $O_1$ enters the hall at time $t_1$ and is continuously tracked by $r_1$ until $t_3$. After that, $O_1$ is detected by $r_3$ at $t_7$ and $r_3$ keeps tracking $O_1$ until $t_9$. However, $O_1$ is not supposed to be detected by $r_3$ before it's detected by reader $r_2$ on its way, because to enter into the hall where $r_3$ is, $O_1$ must pass the detection range of $r_1$ and $r_2$ and cannot remain undetected for time interval $[t_3, t_7]$.
  }
\end{example}

\end{frame}

%------------------------------------------------

\begin{frame}
\frametitle{Aggregate Tracking Table (ATT)~\cite{baba2013spatiotemporal}}

\begin{itemize}

  \item All raw readings are ordered by their detection times and are aggregated into tracking records in \conceptbf{Aggregate Tracking Table}($ATT$).

  \item Each generated tracking record is in the format of $(deviceID, objectID, t_s, t_e, r_{Count})$.

  \item The meaning is that object identified by $objectID$ is detected by device identified by $deviceID$ during time interval $[t_s, t_e]$ for $r_{Count}$ times.
\end{itemize}

\end{frame}

%------------------------------------------------

\begin{frame}
\frametitle{Definitions}

\begin{definition}[False Negatives]
  Given a reader $r_i$ and a moving object $O$, if object $O$ goes through the detection range of reader $r_i$ during time interval $[t, t']$, but $r_i$ does not generate any reading about $O$'s presence during time interval $[t, t']$, a false negative occurs in the data.
\end{definition}

\begin{definition}[False Negatives Handling]
  Given an $ATT$, the false negative handling process detects false negatives and inserts recovered tracking records into the $ATT$.
\end{definition}

\end{frame}

%------------------------------------------------

\begin{frame}
\frametitle{Transition Probabilities}

\conceptbf{Markov Property}: each step a move to next reader (vertice) only depends on current reader and not the readers that precede it, holding a ``memorylessness''.\\~\\ \pause

A moving object at reader $r_i$ moves to a neighboring reader $r_j$ with a probability proportional to the probability weight of the edge $(r_i, r_j)$, i.e., the probability of a transition from $r_i$ to neighboring $r_j$ is : \pause
\begin{equation}
  \frac{p(r_i, p_j)}{\sum_{k \in N(r_i)}p(r_i, K)}
\end{equation}

\pause
Here, $N(r_i)$ is the set of neighbors of reader $r_i$. The transitional probability $p(r_i, r_j)$ denotes the probability that a moving object is detected by $r_j$ at some time later after $t$ given that it was detected by $r_i$ at time $t$ without any third reader involved in-between.

\end{frame}

%------------------------------------------------

\begin{frame}
\frametitle{Transition Probabilities}

\begin{definition}[Transition]
  A transition takes place when a moving object moves from a reader $r_i$ to another reader $r_j$ without passing through any intermediate readers.
\end{definition}

\vspace{10pt}

\conceptbf{A transition probability matrix} represents the transition relationship between a finite number of readers:
\begin{equation}
  P = \begin{bmatrix}
p(r_1, r_1) & p(r_1, r_2) & \cdots & p(r_1, r_n)\\
p(r_2, r_1) & p(r_2, r_2) & \cdots & p(r_2, r_n)\\
\vdots & \vdots & \vdots & \vdots\\
p(r_n, r_1) & p(r_n, r_2) & \cdots & p(r_n, r_n)
\end{bmatrix}
\end{equation}

\vspace{5pt}

For each deployed reader $r_i$, $\sum_{j=1}^{n}p(p_i,p_j) = 1$.

\end{frame}

%------------------------------------------------

\begin{frame}
\frametitle{Transition Probabilities}

The elements of the transition matrix $P$ are then obtained as follows:

\vspace{10pt}

\begin{equation}
  p(r_i, p_j) = \left\{\begin{matrix}
\frac{N_{ij}}{N_i} & \text{if}~(r_i, r_j) \in G_{pdm}.E\\
0 & \text{otherwise}
\end{matrix}\right.
\end{equation}

\vspace{10pt}

where $N_{ij}$ is the number of objects moving from $r_i$ to $r_j$, $N_i$ is the total number of objects moving from $r_i$, $G_{pdm}$ is a probabilistic distance-aware graph model.

\end{frame}

%------------------------------------------------

\begin{frame}
\frametitle{Probabilistic Distance-Aware Graph Model}

\begin{block}{Probabilistic Distance-Aware Graph Model}
  $G_{pdm} = (V, E, \mathcal{L}_V, \mathcal{L}_E)$
  \begin{fitemize}
    \item $V$ is a set of vertices, each represents a deployed device $r_i$.
    \item $E$ is the set of edges, where $E = \{ (r_i, r_j) | r_i, r _j \in V \cup r_i \neq r_j \}$, one can move from $r_i$ to $r_j$ without being detected by third reader.
    \item $\mathcal{L}_V: V \rightarrow \mathcal{R} \times \mathcal{R}$ assigns to a vertex $v_i$ a minimum dwell time and sampling frequency of a corresponding reader $r_i$, i.e., $mathcal{L}_V(r_i) = (d_t, S_f)$.
    \item $\mathcal{L}_E: E \rightarrow \mathcal{R} \times \mathcal{R}$ assigns to an edge $(r_i, r_j)$ the minimum indoor walking time between two devices $r_i$ and $r_j$ and the probability with which an object can move between them, i.e., $mathcal{L}_E(r_i, r_j) = (t_{i,j}, p_{r_i,r_j})$.
  \end{fitemize}
\end{block}

\end{frame}

%------------------------------------------------

\begin{frame}
\frametitle{Probabilistic Distance-Aware Graph Model}

\begin{columns}

  \column{0.6\textwidth}
  \begin{figure}[tb]
    \includegraphics[width=\columnwidth]{figures/3-3/3-3-2.pdf}
  \end{figure}

  \column{0.4\textwidth}
  \begin{fitemize}
    \item the graph is enhanced from a distance deployment graph proposed in \cite{baba2013spatiotemporal}.
    \item different readers usually imply different minimum dwell times for detection and different sampling rates.
    \item the probability weight between two readers will be used to predict the most likely path an object may have taken.
  \end{fitemize}

\end{columns}

\end{frame}

%------------------------------------------------

\begin{frame}
\frametitle{Probabilistic Distance-Aware Graph Construction}

\begin{columns}

  \column{0.55\textwidth}
  \begin{figure}[tb]
    \includegraphics[width=\columnwidth]{figures/3-3/3-3-3.pdf}
  \end{figure}

  \column{0.45\textwidth}
  \begin{enumerate}
    \ssize{
    \item lines 2--6: the indoor shortest path $sp$ is found if the edge $(r_i, r_j)$ is not in the graph yet.
    \item lines 7--9: if $sp$ only contains readers $r_i$ and $r_j$, a new edge is created with the corresponding weights.
    \item lines 11-16: otherwise, each pair of consecutive readers on $sp$ are processed likewise.
    \item indoor shortest paths are computed according to the algorithms in \cite{DBLP:conf/icde/LuCJ12}.
    }
  \end{enumerate}

\end{columns}

\end{frame}

%------------------------------------------------

\begin{frame}
\frametitle{Handling False Negatives}

\begin{definition}[Path]
  A path is a sequence of readers $r_1, r_2, ..., r_n$ where there is an edge connecting $r_i$ and $r_{r+1}$ for $i = 1, 2,...,n$. A path is simple if all $r_i$ are distinct.
\end{definition}

\begin{definition}[Candidate Path]
  Given a source reader $R_s$ and a destination reader $R_d$, a path from $R_s$ to $R_d$, represented as $R_s \overset{\delta}{\rightsquigarrow} R_d$ is a candidate path, if a path $\delta$ satisfies the spatio-temporal constraints captured by a graph.
\end{definition}

\end{frame}

%------------------------------------------------

\begin{frame}
\frametitle{Handling False Negatives}

\begin{definition}[Most Likely Path]
  Given a set of candidate path $\{ \delta_m \}_{m=1}^n$, a most likely path is:
  \begin{equation}
    \delta^* = {\arg\max}_m \prod_{E_{ij} \in \delta_m} p_{i,j}
  \end{equation}
\end{definition}

A two-phase solution is designed to handle false negatives in indoor tracking data, \emph{detecting false negatives} and \emph{recovering false negatives}.

\end{frame}

%------------------------------------------------

\begin{frame}
\frametitle{Topological Scenarios I}

\begin{figure}[tb]
  \includegraphics[width=0.8\columnwidth]{figures/3-3/3-3-4.pdf}
\end{figure}

a sigle path from source reader $R_s$ goes to a destination reader $R_d$. Between $R_i$ and $R_j$ there can be none or many readers deployed all connected in linear fashion.

\end{frame}

%------------------------------------------------

\begin{frame}
\frametitle{Topological Scenarios II}

\begin{figure}[tb]
  \includegraphics[width=0.6\columnwidth]{figures/3-3/3-3-5.pdf}
\end{figure}
\vspace{-10pt}
Source reader $R_s$ have many outgoing edges to $R_i$, $R_j$ to $R_n$. Once the candidates paths are found, transitions probabilities captured by each edge of graph can be used to choose the most probable path to reach destination reader $R_d$.

\end{frame}

%------------------------------------------------

\begin{frame}
\frametitle{Topological Scenarios III}

\begin{figure}[tb]
  \includegraphics[width=0.55\columnwidth]{figures/3-3/3-3-6.pdf}
\end{figure}
\vspace{-10pt}
A sub-case of previous case. Both paths satisfies the spatio-temporal constraints equally, making it difficult to directly find a first candidate path. A most likely path will finally be decided on the transition probablities.

\end{frame}

%------------------------------------------------

\begin{frame}
\frametitle{Detecting False Negatives}

\begin{itemize}

  \item take aggregated tracking table $ATT$ as an input, identifies possible false negatives using the probabilistic distance-aware graph model of all deployed readers.

  \item look at each tracking record $tr$ in $ATT$ and find out the neighbors of $tr.deviceID$.

  \item take next tracking record $tr'$ and check if $tr'.deviceID$ is in a list of neighbors of $tr.deviceID$.

  \item if none of neighbors matches, it can be concluded that one or more readers between the two devices have failed to detect a moving object with identity $tr.objectID$.

  \item find the path between the two devices using the above definitions.

\end{itemize}

\end{frame}

%------------------------------------------------

\begin{frame}
\frametitle{Detecting False Negatives: Algorithm}

\begin{columns}

  \column{0.5\textwidth}
  \begin{figure}[tb]
    \includegraphics[width=\columnwidth]{figures/3-3/3-3-7.pdf}
  \end{figure}

  \column{0.5\textwidth}
  \begin{sitemize}
    \item line 3: find out all the outgoing neighbors of the first tracking device ($tr.deviceID$).
    \item line 4: check if the current device $tr.deviceID$ is not the last device
    \item lines 5--9: get $tr$'s next tracking record $tr'$ from $ATT$ that involves the same object as $tr$, if next tracking record is null, simply contunue with another object. If there is and $tr'.deviceID$ is equal to one of the neighbors, simply continue the process.
    \item lines 10-13: otherwise there is false negatives, call the function \emph{FalseNegativeRecovery}.
  \end{sitemize}

\end{columns}

\end{frame}

%------------------------------------------------

\begin{frame}
\frametitle{Algorithm: \emph{findAllPaths}}

\begin{columns}

  \column{0.5\textwidth}
  \begin{figure}[tb]
    \includegraphics[width=\columnwidth]{figures/3-3/3-3-9.pdf}
  \end{figure}

  \column{0.5\textwidth}
  \begin{sitemize}
    \item takes a graph $G_{pdm}$, source reader $R_s$ and destination reader $R_d$ as input.
    \item returns all the paths between $R_s$ and $R_d$.
    \item follows an improved depth-first search paradigm, which estimates all the possible paths between two readers.
  \end{sitemize}

\end{columns}

\end{frame}

%------------------------------------------------

\begin{frame}
\frametitle{Algorithm: \emph{findPath}}

\begin{columns}

  \column{0.5\textwidth}
  \begin{figure}[tb]
    \includegraphics[width=\columnwidth]{figures/3-3/3-3-8.pdf}
  \end{figure}

  \column{0.5\textwidth}
  \begin{sitemize}
    \item takes graph model $G_{pdm}$ and two tracking records $tr$, $tr'$ as input.
    \item calls \emph{findAllPaths} to get all the possible paths between $R_s$ and $R_d$.
    \item lines 5-15: to check if the path satisfies the temporal constraint, a total traverse time ($T_{cal}$) is calculated, which is a time a moving object takes to traverse a path from $R_s$ to $R_d$.
    \item lines 17: if $T_{cal}$ is larger than the time object was first tracked the $R_d$, the path is treated as invalid and is deleted from the candidate paths.
    \item line 20: a most likely path is then estimated based on the maximum likelihood decision rule by using transition probabilities.
  \end{sitemize}

\end{columns}

\end{frame}


%------------------------------------------------

\begin{frame}
\frametitle{Recovering False Negatives}

\begin{itemize}

  \item retrieve a path between source reader $R_s$(tr.deviceID) and destination reader $R_d$(tr'.deviceID) in the graph and fills the missing readings of each reader a path contains between $R_s$ and $R_d$.

  \item the path retrieved is the most likely path, in a set of paths which satisfy the spatio-temporal constraints of subgraph between $R_s$ and $R_d$.

  \item to fill the missing readings, the parameters like, minimum travelling time between readers, minimum dwell time and sampling rate of corresponding reader are used.

  \item the approximate number of raw readings to be filled for each reader is determined as $\frac{\text{minimum dwell time}}{\text{sampling rate}}$.

\end{itemize}

\end{frame}

%------------------------------------------------

\begin{frame}
\frametitle{Conclusion}

This paper studies data cleansing for indoor RFID tracking data.\\~\\

It focuses one of the main aspects in raw indoor RFID tracking data, namely, \emph{false negatives}.\\~\\

A probabilistic distance-aware graph model is proposed to capture probabilities together with the spatio-temporal constraints implied by the deployment of RFID readers as well as indoor topology.\\~\\

The transition probabilities that capture likely an object move from one reader to another are obtained from the existing data.

\end{frame}

% %------------------------------------------------

\begin{frame}[allowframebreaks]
\frametitle{References}

\begin{thebibliography}{99} % Beamer does not support BibTeX so references must be inserted manually as below
\bibliographystyle{abbrv}
\fsize{

\bibitem{baba2013spatiotemporal}
A.I.~Baba, H.~Lu, X.~Xie, T.B.~Pedersen.
\newblock Spatiotemporal data cleansing for indoor RFID tracking data.
\newblock In {\em MDM}, pp. 187--196, 2013.

\bibitem{baba2013graph}
A.I.~Baba, H.~Lu, T.B.~Pedersen, X.~Xie.
\newblock A graph model for false negative handling in indoor RFID tracking data.
\newblock In {\em SIGSPATIAL}, pp. 464--467, 2013.

\bibitem{floerkemeier2004issues}
C.~Floerkemeier, M.~Lampe.
\newblock Issues with RFID usage in ubiquitous computing applications.
\newblock In {\em Pervasive Computing}, pp. 188--193, 2004.

\bibitem{derakhshan2007rfid}
R.~Derakhshan, M.E.~Orlowska, X.~Li.
\newblock RFID data management: challenges and opportunities.
\newblock In {\em IEEE International conference on RFID}, vol. 10, 2007.

\bibitem{DBLP:conf/icde/LuCJ12}
H.~Lu, X.~Cao, and C.~S. Jensen.
\newblock A foundation for efficient indoor distance-aware query processing.
\newblock In {\em ICDE}, pp. 438--449, 2012.

}
\end{thebibliography}

\end{frame}


\subsection{3.4 Offline Cleaning of RFID Trajectory Data (PART I)} % A subsection can be created just before a set of slides with a common theme to further break down your presentation into chunks

% \include{sec_3_4_1}
% %------------------------------------------------

\begin{frame}[allowframebreaks]
\frametitle{References}

\begin{thebibliography}{99} % Beamer does not support BibTeX so references must be inserted manually as below
\bibliographystyle{abbrv}
\fsize{

\bibitem{DBLP:conf/edbt/FazzingaFFP14}
B.~Fazzinga, S.~Flesca, F.~Furfaro, F.~Parisi.
\newblock Cleaning Trajectory Data of RFID-monitored Objects through Conditioning under Integrity Constraints.
\newblock In {\em EDBT}, pp. 379--390, 2014.

\bibitem{koch2008conditioning}
C.~Koch, D.~Olteanu.
\newblock Conditioning probabilistic databases.
\newblock In {\em VLDB}, pp. 313--325, 2008.

\bibitem{flesca2014consistency}
S.~Flesca, F.~Furfaro, F.~Parisi.
\newblock Consistency checking and querying in probabilistic databases under integrity constraints.
\newblock In {\em Journal of Computer and System Sciences}, pp. 1448--1489, 2014.


}
\end{thebibliography}

\end{frame}


\subsection{3.4 Offline Cleaning of RFID Trajectory Data (PART II)} % A subsection can be created just before a set of slides with a common theme to further break down your presentation into chunks

% \begin{frame}
\frametitle{About This Work...}

\emph{Offline cleaning of RFID trajectory data}~\cite{fazzinga2014offline}\\
B.~Fazzinga, S.~Flesca, F.~Furfaro, F.~Parisi.\\~\\

\begin{itemize}
  \item Published at \emph{SSDBM' 2014}.
  \item A smoothing technique following a two-way filtering scheme that embeds a sampling strategy for efficiently dealing with missing detections.
\end{itemize}

\end{frame}

%------------------------------------------------

\begin{frame}
\frametitle{Motivation}

\begin{itemize}
  \item Pervasive use of RFID devices as a support for object tracking
  \begin{fitemize}
    \item monitoring of people, animals and objects inside museums, schools, hospital etc.
    \item context-aware information.
  \end{fitemize}
  \item There exists ambiguity in the raw RFID data.
  \begin{fitemize}
    \item no one-to-one correspondence between locations and readers, no way to deterministically decide the location given that a set of readers detected an object.
    \item the same location may constrain different reader's detection zones, the same reader may detect different objects at different locations, also false negatives.
  \end{fitemize}
\end{itemize}

\end{frame}

%------------------------------------------------

\begin{frame}
\frametitle{Ambiguity of RFID Data}

\begin{columns}

  \column{0.4\textwidth}
  \begin{figure}[tb]
    \includegraphics[width=\columnwidth]{figures/3-4/3-4-1.pdf}
  \end{figure}

  \column{0.6\textwidth}
  \begin{example}
    \fsize{
    an object $o$ was detected at some instant by both reader $r_1$ and $r_5$, two locations are possible, $L_1$ and $L_4$. Analogously, if $o$ was detected by $r_3$ only, we cannot conclude that it was surely in $L_3$, as it could be the case that $r_2$ failed to detect it despite it was close enough to its antenna.
    }
  \end{example}

  % \ssize{\textrm{\\this suggests that the association readers/locations can be naturally modeled in probabilistic terms. For instance by a probability distribution $p^a(l|R)$ defined for each location $l$ and set $R$ of readers.}}
  % $p^a(L_1|\{r_1, r_5\}) = p^a(L_4|\{r_1, r_5\}) = 0.5$
  % $p^a(L_0|\{r_0\}) = 1$

\end{columns}

\end{frame}

%------------------------------------------------

\begin{frame}
\frametitle{Ambiguity of RFID Data}

\begin{columns}

  \column{0.4\textwidth}
  \begin{figure}[tb]
    \includegraphics[width=\columnwidth]{figures/3-4/3-4-1.pdf}
  \end{figure}

  \column{0.6\textwidth}
  \begin{figure}[tb]
    \includegraphics[width=\columnwidth]{figures/3-4/3-4-13.pdf}
  \end{figure}

  \vspace{-10pt}

  \begin{example}
    \ssize{
    $t = 1$: an object detected by $r_0$ is surely at $L_0$ (since the area covered by $r_0$ is totally inside $L_0$)\\
    $t = 2$: an object detected by $r_1$ can be either at $L_1$ or $L_4$ (if object only detected by $r_1$ could also be laid in the area covered by $r_1$ and $r_5$, as $r_5$ may fail to detect the object, the false negative)\\
    $t = 3$: an object detected by no reader can be almost anywhere\\
    $t = 4$: reader $r_6$ covers portions of $L_3, L_4, L_8, L_9$ only.
    }
  \end{example}


\end{columns}

\end{frame}

%------------------------------------------------

\begin{frame}
\frametitle{Ambiguity of RFID Data}

\ssize{\textrm{assume that $o$ is a person, his maximum speed is $v_{max} = 2m/s$. Consider the size of the floor is $15m \times 11m$, re-examine the possible location as follow:}}

\vspace{10pt}

\begin{columns}

  \column{0.3\textwidth}
  \begin{figure}[tb]
    \includegraphics[width=\columnwidth]{figures/3-4/3-4-1.pdf}
  \end{figure}


  \column{0.7\textwidth}
  \ssize{
  A.  As regards $t = 2s$, $o$ must be at $L_1$, as it cannot be at $L_4$. In fact, $o$ was at $L_0$ at $t = 1s$, and the minimum (indoor) distance between $L_0$ and $L_4$ (about 7m) cannot be covered in $1sec$ without exceeding $v_{max}$. \\~\\
  B.  As regards $t = 3s$, now that it has been established that $o$ was at $L_1$ at $t = 2s$, we can restrict the set of possbile locations from $\{ L_1,...,L_{10} \}$ to $\{ L_0, L_1, L_2 \}$ (due to the size of the floor, reaching a room not in this set from any position inside $L_1$ in one second would require a speed greater than $v_{max}$). \\~\\
  C.  As regards $t = 4s$, the only location in $\{ L_3, L_4, L_8, L_9 \}$ (the set of location covered by $r_6$) that can be reached from at least one location in $\{ L_0, L_1, L_2 \}$ (the possible locations at time $t = 3s$, as established at point B) is $L_4$.
  }


\end{columns}

\end{frame}

%------------------------------------------------

\begin{frame}
\frametitle{Ambiguity of RFID Data}

\ssize{\textrm{all the arguments used in Point A,B,C show that the exploiting the correlation with the past positions can reduce (or even eliminate) the uncertainty in determining the position at any time point. Trying to further reduce the uncertainty by exploiting the correlation:}}

\vspace{10pt}

\begin{columns}

  \column{0.3\textwidth}
  \begin{figure}[tb]
    \includegraphics[width=\columnwidth]{figures/3-4/3-4-1.pdf}
  \end{figure}


  \column{0.7\textwidth}
  \ssize{
  D.  Consider $t = 3s$, we have already established that the position of $o$ at the subsequent time point is $L_4$, and that the possible locations for this time point are $L_0, L_1, L_2$. Hence, at $t = 3s$, $o$ must have been at $L_2$, as this is the only among the possible locations for this time point from which $L_4$ can be reached in one second without exceeding $v_{max}$.
  }

  \begin{figure}[tb]
    \includegraphics[width=\columnwidth]{figures/3-4/3-4-14.pdf}
  \end{figure}


\end{columns}

\end{frame}

%------------------------------------------------

\begin{frame}
\frametitle{The Problem Setting for This Work}

\fsize{

An instance of \conceptbf{smoothing problem}~\cite{smith2013sequential}: estimating the state of an observed entity at a time point $t \in [1...T]$ on the basis of all the observations over $[1...T]$ in terms of a posterior probability distribution over the possible states.\\~\\

This work proposes a smoothing technique following a two-way-filtering scheme that embeds a sampling strategy for efficiently dealing with missing detections.\\~\\

In the first \emph{forward} phase, time points are processed from the first to the last, and the possible positions at each time point are filtered by taking into account the results of the previous steps.\\~\\

In the \emph{backward} phase, it proceeds from the last to the first time point: for each time point, filter the positions detected as admissible in the forward phase and revise their weights, by checking whether they are a valid position when looking at the results obtained for the future time point.

}

\end{frame}

%------------------------------------------------

\begin{frame}
\frametitle{The Problem Setting for This Work}

\fsize{

The map of the locations is assumed to be partitioned according to a grid, thus the positions to be filtered and weighted at each step of the forward phase are the grid cells compatible with the readers that performed the detection.\\~\\

Thus, the grid cells may result in too many candidate positions, a variant of the cleaning algorithm where missing detections are treated by integrating the grid-based approach with sampling strategy is proposed.\\~\\

At the end of each forward step involving a missing detection, the cells that survived the filtering are further filtered according to a sampling technique, where the probability of a cell to be sampled is an estimate of the probability it would have been assigned if it was considered also in the following phases.

}

\end{frame}

%------------------------------------------------

\begin{frame}
\frametitle{The Preliminaries}

\begin{figure}[tb]
  \includegraphics[width=0.55\columnwidth]{figures/3-4/3-4-15.pdf}
\end{figure}

\end{frame}

%------------------------------------------------

\begin{frame}
\frametitle{The Preliminaries}

Assume the set $\mathcal{R} = \{ r_1,...,r_n \}$ of readers, and the set $\mathcal{L} = \{ L_1,...,L_n \}$ of locations on map $M$.\\~\\ \pause

The map of locations is partitioned according to a regular grid, whose generic cell will be denoted as $c$. The set of cells intersecting the area covered by a set $R \in \mathcal{R}$ of readers will be denoted as $Cells(R)$. \\~\\ \pause

A \emph{detection-sequence} (d-sequence for short) $D$ over time interval $I$ is a sequence $R_1,...,R_T$, where $R_t \subset \mathcal{R}$ (with $t \in [1...T]$) is the (possibly empty) set of readers that detected $o$ at time point $t$.

\end{frame}

%------------------------------------------------

\begin{frame}
\frametitle{The Preliminaries}

\begin{problem}[The Cleaning Problem]
  Given a d-sequence $D = R_1,...R_T$, representing the sets of readers that consecutively detected object $o$ while it was moving over the time interval $I = [1...T]$ and over the map $M$, the \emph{cleaning problem} addressed in this work is that of estimating the actual position (at the granularity level of location) of $o$ at each $t \in I$. This will be accomplished by modeling the position of $o$ at $t$ as a random variable over $\mathcal{L}$, and providing a posterior PDF $p_t$ which is conditional to the relevant evidence encoded by the d-sequence $D$. The output of this cleaning task is the sequence $p_1,...p_T$ of these posterior PDFs, and will be called \emph{probabilistic trajectory} (p-trajectory) over time interval $I$.
\end{problem}

\end{frame}

%------------------------------------------------

\begin{frame}
\frametitle{The Cleaning Algorithm}

\begin{block}{A Motion Model for estimating the probability of movements}
  Given two cells $c, c'$, to estimate the probability that object $o$ has reached $c'$ starting from $c$ in a certain amount of time. To this end, we exploit the knowledge of $v_{max}$, and we assume that the possible speeds of $o$ are uniformly distributed over $[0,v_{max}]$. Then, denoting the probability that the speed of $o$ is at least $x$ as $p^{mov}(v \geq x)$, we estimate the probability that $o$ reached $c'$ from $c$ in time $\Delta$ as the probability that $o$ had at least the speed needed to walk through the shortest path from $c$ to $c'$ in time $\Delta$, i.e., $p^{mov} (v \geq d_{min}(c,c')/\Delta)$.
\end{block}

\textit{\textrm{this way, the probability in motion model takes into account both the maximum speed and the topology of the map (that implies the value of $d_{min}$).}}

\end{frame}

%------------------------------------------------

\begin{frame}
\frametitle{The Cleaning Algorithm: Overview}

\begin{columns}

  \column{0.5\textwidth}
  \begin{figure}[tb]
    \includegraphics[width=\columnwidth]{figures/3-4/3-4-16.pdf}
  \end{figure}

  \column{0.5\textwidth}
  \begin{sitemize}
    \item consists of the \emph{forward} phase and the \emph{backward} phase.
    \item during the forward phase, the input d-sequence is progressively scanned from the first to the last time point, for each time point $t \in [1...T]$, a set $C(t)$ of cells is determined, meaning the \emph{possible} positions of $o$ at $t$.
    \item ``possible'' means these positions are consistent both with the detections at $t$ itself and with the positions which have been recognized as possible for $o$ at previous time point.
    \item each cell $c$ in $C(t)$ is assigned a \emph{forward probability} $p^{fw}_t(c)$, which represents the probability that $c$ is the position at $t$ (only the past time points have been taken into account), it's evaluated using the motion model.
  \end{sitemize}

\end{columns}

\end{frame}

%------------------------------------------------

\begin{frame}
\frametitle{The Cleaning Algorithm: Overview}

\begin{columns}

  \column{0.5\textwidth}
  \begin{figure}[tb]
    \includegraphics[width=\columnwidth]{figures/3-4/3-4-16.pdf}
  \end{figure}

  \column{0.5\textwidth}
  \begin{sitemize}
    \item the backward phase is analogous to the forward one but processes in inverse order on the result by forward phase.
    \item for each cell in $C(t)$, the \emph{backward probability} $p^{bw}_t(c)$ is computed: it represents the probability that $c$ is the position at $t$, when only the possible positions at time points in $[t+1,T]$ are taken into account.
    \item cells with backward porbability equal to $0$ are removed from $C(t)$, since they are turne out to be inconsistent with the possible positions in $[t+1,T]$.
  \end{sitemize}

\end{columns}

\end{frame}

%------------------------------------------------

\begin{frame}
\frametitle{The Cleaning Algorithm: Overview}

\begin{columns}

  \column{0.5\textwidth}
  \begin{figure}[tb]
    \includegraphics[width=\columnwidth]{figures/3-4/3-4-16.pdf}
  \end{figure}

  \column{0.5\textwidth}
  \begin{sitemize}
    \item at the end of backward phase, for each $t \in [1...T]$ and cell $c \in C(t)$, the posterior probability $p_t(c)$ is finally computed as the product among the following 3 constributions: $p^{fw}_t(c)$, $p^{bw}_t(c)$ and likelihood $h(R_t|c)$ (the probability that $o$ is detected by the reaaders in $R_t$ given that $o$ is at cell $c$).
    \item The so obtained functions $p_1, ..., p_T$ are then projected over the locations and normalized. A probabilistic trajectory where the PDF at each time point is computed by considering the past, the present and the future readings.
  \end{sitemize}

\end{columns}

\end{frame}

%------------------------------------------------

\begin{frame}
\frametitle{The Cleaning Algorithm: Overview}

The above-sketch approach may turn out to be inefficient in the presence of missing detections.\\~\\

If exhaustively consider as possible positions as all the cells compatible with the set of readers (that detected the object and that are reachable in one time unit from the previous possible position), may result in too large number of possible locations in one time point. \\~\\

Two variant of cleaning algorithms are proposed, namely the \emph{Exhaustive Approach} and a \emph{Sampling-based Approach}.

\end{frame}

%------------------------------------------------

\begin{frame}
\frametitle{Exhaustive Approach: Forward Phase}

\begin{columns}

\column{0.55\textwidth}
\begin{figure}[tb]
  \includegraphics[width=\columnwidth]{figures/3-4/3-4-22.pdf}
\end{figure}

\column{0.45\textwidth}
\ssize{
\begin{enumerate}
  \item lines 1--3: $C(1)$ is assigned the set of the cells in $Cells(R_1)$, while $p^{fw}_t$ is assumed uniformly distributed.
  \item lines 4--7: for each time point $t$ in $[2...T]$, $C(t)$ is evaluated as the set of cells of $Cells(R_t)$ which are reachable by at least one cell $c' \in C(t-1)$ in one time unit, considering the movement limitations implied by the map and the maximum speed.
\end{enumerate}
}

\end{columns}

\end{frame}

%------------------------------------------------

\begin{frame}
\frametitle{Exhaustive Approach: Forward Phase}

\begin{columns}

\column{0.4\textwidth}
\begin{figure}[tb]
  \includegraphics[width=\columnwidth]{figures/3-4/3-4-17.pdf}
\end{figure}

\column{0.6\textwidth}
\begin{figure}[tb]
  \includegraphics[width=\columnwidth]{figures/3-4/3-4-18.pdf}
\end{figure}

\ssize{

\textbf{STEP I-1}\quad set $C(1) = Cells(R_1) = \{ e13,e14,e15 \}$, assign probability $p^{fw}_1(c) = \frac{1}{3}$ to each cell $c \in C(1)$.\\~\\

\textbf{STEP I-2}\quad At $t = 2$, $Cells(R_2) = \{ f7,f8,f9,f10,g7,g8,g9,g10 \}$. Among these cells, only the dark-gray colored ones are reachable in one time unit. I.e., cells $f8,f9,f10,g10$ are reachable from $e13$ and $e14$, while cell $g10$ is reachable from $e15$. The $p^{mov}$ is computed as figure(b).

}

\end{columns}

\end{frame}

%------------------------------------------------

\begin{frame}
\frametitle{Exhaustive Approach: Forward Phase}

\begin{columns}

\column{0.4\textwidth}
\begin{figure}[tb]
  \includegraphics[width=\columnwidth]{figures/3-4/3-4-17.pdf}
\end{figure}

\column{0.6\textwidth}
\begin{figure}[tb]
  \includegraphics[width=\columnwidth]{figures/3-4/3-4-19.pdf}
\end{figure}

\ssize{

\textbf{STEP I-3}\quad the forward probability shown in figure(c) is obtained by summing the products of forward probabilities of cell $c_i$ shown in figure(a) with the corresponding $p^{mov}$ in figure(b). For instance, $p^{fw}_2(f10) = \frac{1}{3} \cdot 0 + \frac{1}{3} \cdot \frac{6}{10} + \frac{1}{3} \cdot \frac{8}{10} = \frac{7}{15}$.
}

\end{columns}

\end{frame}

%------------------------------------------------

\begin{frame}
\frametitle{Exhaustive Approach: Forward Phase}

\begin{columns}

\column{0.4\textwidth}
\begin{figure}[tb]
  \includegraphics[width=\columnwidth]{figures/3-4/3-4-17.pdf}
\end{figure}

\column{0.6\textwidth}
\begin{figure}[tb]
  \includegraphics[width=\columnwidth]{figures/3-4/3-4-20.pdf}
\end{figure}

\ssize{

\textbf{STEP II-1}\quad from $\{ g10,f10,f9,f8 \}$ shown in figure(a), only $\{ f4,f3,f2,f1 \}$ can be reached while $\{ g4,g3,g2,g1 \}$ cannot.\\~\\

\textbf{STEP II-2}\quad only the dark-grayed cells $C(3) = \{ f3,f4 \}$ can meet the maximum speed condition, the $p^{mov}(v \geq \frac{d_{min}(c_i,c_j)}{\Delta})$ are computed in figure(b).
}

\end{columns}

\end{frame}

%------------------------------------------------

\begin{frame}
\frametitle{Exhaustive Approach: Forward Phase}

\begin{columns}

\column{0.4\textwidth}
\begin{figure}[tb]
  \includegraphics[width=\columnwidth]{figures/3-4/3-4-17.pdf}
\end{figure}

\column{0.6\textwidth}
\begin{figure}[tb]
  \includegraphics[width=\columnwidth]{figures/3-4/3-4-21.pdf}
\end{figure}

\ssize{

\textbf{STEP II-3}\quad the forward probability shown in figure(c) is obtained by summing the products of forward probabilities of cell $c_i$ shown in figure(a) with the corresponding $p^{mov}$ in figure(b). For instance, $p^{fw}_3(f4) = \frac{1}{15} \cdot 0 + \frac{7}{15} \cdot \frac{2}{10} + \frac{1}{3} \cdot \frac{4}{10} + \frac{1}{5} \cdot \frac{4}{10} = \frac{46}{150}$.
}

\end{columns}

\end{frame}

%------------------------------------------------

\begin{frame}
\frametitle{Exhaustive Approach: Backward Phase and Final Steps}

\begin{columns}

\column{0.5\textwidth}
\begin{figure}[tb]
  \includegraphics[width=\columnwidth]{figures/3-4/3-4-23.pdf}
\end{figure}

\column{0.5\textwidth}
\ssize{
\begin{enumerate}
  \item line 11: scan the time points in $[1...T]$ in inverse order, and assigns uniform probability $p^{bw}_T$ to the cells in $C(T)$.
  \item line 15: if a cell has backward probability equal to 0, it's removed from $C(t)$, since it is incompatible with the future positions.
  \item line 17: the algorithm computes the probability $p_t(c)$ by multiplying the backward probability with the forward probablity resulting from the forward phase and the likelihood $h(R_t|c)$.
  \item line 18: the so obtained probabilities are projected on $\mathcal{L}$, fianlly, these probabilities are normalized, and the p-trajectories are returned.
\end{enumerate}
}

\end{columns}

\end{frame}

%------------------------------------------------

\begin{frame}
\frametitle{Exhaustive Approach: Backward Phase and Final Steps}

\begin{columns}

\column{0.4\textwidth}
\begin{figure}[tb]
  \includegraphics[width=\columnwidth]{figures/3-4/3-4-24.pdf}
\end{figure}

\column{0.6\textwidth}

\ssize{

\textbf{STEP III-1}\quad assigns $p^{bw}_3(c) = \frac{1}{2}$ for each of the two cells in $C(3)$.\\~\\

\textbf{STEP III-2}\quad $p^{bw}_2(g10) = 0$ (remove it), $p^{bw}_2(f10) = \frac{1}{2} \cdot \frac{2}{10} = \frac{1}{10}$, $p^{bw}_2(f9) = \frac{1}{2} \cdot \frac{6}{10} = \frac{3}{10}$, $p^{bw}_2(f8) = \frac{1}{2} \cdot \frac{6}{10} = \frac{3}{10}$.\\~\\

\textbf{STEP III-3}\quad $p^{bw}_3(e14) = \frac{1}{10} \cdot \frac{6}{10} + \frac{3}{10} \cdot \frac{4}{10} + \frac{3}{10} \cdot \frac{2}{10} = \frac{24}{100}$, $p^{bw}_3(e13) = \frac{1}{10} \cdot \frac{8}{10} + \frac{3}{10} \cdot \frac{6}{10} + \frac{3}{10} \cdot \frac{4}{10} = \frac{38}{100}$.\\~\\

\textbf{STEP III-4}\quad the posterior probability $p_t(c)$ is computed by projection: $p_1(L_2) = p_1(e13) + p_1(e14)$, $p_2(L_2) = p_2(f10) + p_2(f9) + p_2(f8)$, $p_3(L_4) = p_3(f4) + p_3(f3)$. After normalization, the probabilities that $o$ was at location $L_2$ at time points 1 and 2, and at $L_4$ at time 3 are $100\%$.

}

\end{columns}

\end{frame}

%------------------------------------------------

\begin{frame}
\frametitle{Exhaustive Approach: Unsuitability for Missing Detections}

\begin{block}{}
  \ssize{
  $C(t)$ is the set of cells with two properties:

  \begin{itemize}
    \item i. they are covered by the set $R_t$;
    \item ii. they are reachable in one time unit from at least a cell in $C(t-1)$ without exceeding the speed limit imposed by $v_{max}$.
  \end{itemize}

  In the case of missing detection (i.e., $R_t = \varnothing$) , the size of $C(t)$ strongly depends on the maximum speed and the size of $C(t-1)$.
  }
\end{block}

Even if $C(t-1)$ and $v_{max}$ are such that the size of $C(t)$ is smaller than the number of cells covered by one reader, it is easy to see that, in the presence of a sequence of $m$ missing detections starting at time $t$, the size of $C(t+m)$ may become extremely large.

\end{frame}

%------------------------------------------------

% \begin{frame}
% \frametitle{Sampling-based Approach}
%
%
%
% \end{frame}

% %------------------------------------------------

\begin{frame}[allowframebreaks]
\frametitle{References}

\begin{thebibliography}{99} % Beamer does not support BibTeX so references must be inserted manually as below
\bibliographystyle{abbrv}
\fsize{

\bibitem{fazzinga2014offline}
B.~Fazzinga, S.~Flesca, F.~Furfaro, F.~Parisi.
\newblock Offline cleaning of RFID trajectory data.
\newblock In {\em SSDBM}, pp. 5:1 -- 5:12, 2014.

\bibitem{DBLP:conf/edbt/FazzingaFFP14}
B.~Fazzinga, S.~Flesca, F.~Furfaro, F.~Parisi.
\newblock Cleaning Trajectory Data of RFID-monitored Objects through Conditioning under Integrity Constraints.
\newblock In {\em EDBT}, pp. 379--390, 2014.

\bibitem{smith2013sequential}
A.~Smith, A.~Doucet, N.~de Freitas, N.~Gordon.
\newblock Sequential Monte Carlo methods in practice.
\newblock In {\em Science \& Business Media}, 2013.

% \bibitem{koch2008conditioning}
% C.~Koch, D.~Olteanu.
% \newblock Conditioning probabilistic databases.
% \newblock In {\em VLDB}, pp. 313--325, 2008.

% \bibitem{flesca2014consistency}
% S.~Flesca, F.~Furfaro, F.~Parisi.
% \newblock Consistency checking and querying in probabilistic databases under integrity constraints.
% \newblock In {\em Journal of Computer and System Sciences}, pp. 1448--1489, 2014.


% \bibitem{baba2013spatiotemporal}
% A.I.~Baba, H.~Lu, X.~Xie, T.B.~Pedersen.
% \newblock Spatiotemporal data cleansing for indoor RFID tracking data.
% \newblock In {\em MDM}, pp. 187--196, 2013.
%
% \bibitem{baba2013graph}
% A.I.~Baba, H.~Lu, T.B.~Pedersen, X.~Xie.
% \newblock A graph model for false negative handling in indoor RFID tracking data.
% \newblock In {\em SIGSPATIAL}, pp. 464--467, 2013.
%
% \bibitem{floerkemeier2004issues}
% C.~Floerkemeier, M.~Lampe.
% \newblock Issues with RFID usage in ubiquitous computing applications.
% \newblock In {\em Pervasive Computing}, pp. 188--193, 2004.
%
% \bibitem{derakhshan2007rfid}
% R.~Derakhshan, M.E.~Orlowska, X.~Li.
% \newblock RFID data management: challenges and opportunities.
% \newblock In {\em IEEE International conference on RFID}, vol. 10, 2007.
%
% \bibitem{DBLP:conf/icde/LuCJ12}
% H.~Lu, X.~Cao, and C.~S. Jensen.
% \newblock A foundation for efficient indoor distance-aware query processing.
% \newblock In {\em ICDE}, pp. 438--449, 2012.
%
% \bibitem{baba2014handling}
% A.I.~Baba, H.~Lu, T.B.~Pedersen, X.~Xie.
% \newblock Handling false negatives in indoor RFID data.
% \newblock In {\em MDM}, pp. 117--126, 2014.


}
\end{thebibliography}

\end{frame}


\subsection{3.5 Learning-Based Cleansing for Indoor RFID Data} % A subsection can be created just before a set of slides with a common theme to further break down your presentation into chunks

% \begin{frame}
\frametitle{About This Work...}

\emph{Learning-Based Cleansing for Indoor RFID Data}~\cite{baba2016learning}\\
A.I.~Baba, M.~Jaeger, H.~Lu, T.B.~Pedersen, W.-S.~Ku, X.~Xie.\\~\\

\begin{itemize}
  \item Published at \emph{SIGMOD' 2016}.
  \item Proposed a learning-based data cleansing approach that, requires no detailed prior knowledge about the spatio-temporal properties of the indoor space and the RFID reader deployment.
  \item Proposed the Indoor RFID Multi-variate Hidden Markov Model (IR-MHMM) to capture the uncertainties of indoor RFID data as well as the correlation of moving object locations and object readings.
\end{itemize}

\end{frame}

%------------------------------------------------

\begin{frame}
\frametitle{Motivation}

\begin{itemize}
  \item Recently there has been a remarkable proliferation of RFID in indoor tracking and monitoring systems
  \begin{fitemize}
    \item airport baggage monitoring. ~\cite{baba2013spatiotemporal,baba2013graph}
  \end{fitemize}
  \item The dirtiness of RFID data poses challenges to high-level RFID data querying and analysis.~\cite{chen2010leveraging}
  \begin{fitemize}
    \item \conceptbf{false negatives} (missing readings) occur when a reader fails to read a tag in its detection range.
    \item \conceptbf{false positives} (cross readings) occur when a tagged object is unexpectedly read by multiple readers simultaneously.
  \end{fitemize}
  \item This work focuses on cleansing \emph{false negatives} and \emph{false positives}.
\end{itemize}

\end{frame}

%------------------------------------------------

\begin{frame}
\frametitle{False Negatives \& False Positives}

\begin{columns}

  \column{0.4\textwidth}
  \begin{figure}[tb]
    \includegraphics[width=\columnwidth]{figures/3-5/3-5-1.pdf}
  \end{figure}
  \ssize{\textit{an object with tag $tag_1$ moved into the building and was first detected by reader $R_1$ from time point $t_0$ to time point $t_3$, yielding four observations by reader $R_1$.}}

  \column{0.6\textwidth}
  \begin{figure}[tb]
    \includegraphics[width=\columnwidth]{figures/3-5/3-5-2.pdf}
  \end{figure}
  \ssize{
    1. at $t_{12}$ and $t_{13}$, $tag_1$ was detected by readers $R_4$ and $R_9$, it seems to be present in both locations, thus giving rise to a \emph{false positives}.\\~\\
    2. after $t_{16}$, $tag_1$ was detected by reader $R_{17}$ that kept detecting $tag_1$ until $t_{29}$. However $tag_1$ is not supposed to be detected by $R_{17}$ before it's detected by $R_{10}$. $tag_1$ passed through $R_{10}$ but failed to generate any information, giving rise to \emph{false negatives}.
  }

\end{columns}

\end{frame}

%------------------------------------------------

\begin{frame}
\frametitle{Motivation}

To support high-level RFID business logic processing, it is necessary to perform data cleansing to remove false negatives and false positives.\\~\\

Existing RFID data cleansing techniques require considerable specific prior knowledge for cleansing operations.

\begin{sitemize}
  \item to cleanse historical indoor RFID data, the graph model based cleansing approaches~\cite{baba2013spatiotemporal,baba2013graph,DBLP:conf/edbt/FazzingaFFP14} rely on graphs that capture the indoor topology, the deployment of readers, and multiple pertinet spatial-temporal properties.
  \item in the context of streaming RFID data, a probabilistic approach~\cite{tran2009probabilistic} demands to build four domain-specific probabilistic models before any data cleansing.
  \item also streaming RFID data, ~\cite{nie2009probabilistic} assumes that locations of neighboring objects are correlated to each other, lifting such assumptions and needs only minimal prior knowledge about indoor settings.
\end{sitemize}

\end{frame}

%------------------------------------------------

\begin{frame}
\frametitle{Preliminaries}



\end{frame}

% %------------------------------------------------

\begin{frame}[allowframebreaks]
\frametitle{References}

\begin{thebibliography}{99} % Beamer does not support BibTeX so references must be inserted manually as below
\bibliographystyle{abbrv}
\fsize{

\bibitem{baba2016learning}
A.I.~Baba, M.~Jaeger, H.~Lu, T.B.~Pedersen, W-S.~Ku, X.~Xie.
\newblock Learning-Based Cleansing for Indoor RFID Data.
\newblock In {\em SIGMOD}, 2016.

\bibitem{chen2010leveraging}
H.~Chen, W-S.~Ku, H.~Wang, M-T.~Sun.
\newblock Leveraging spatio-temporal redundancy for RFID data cleansing.
\newblock In {\em SIGMOD}, pp. 51--62, 2010.

\bibitem{baba2013spatiotemporal}
A.I.~Baba, H.~Lu, X.~Xie, T.B.~Pedersen.
\newblock Spatiotemporal data cleansing for indoor RFID tracking data.
\newblock In {\em MDM}, pp. 187--196, 2013.

\bibitem{baba2013graph}
A.I.~Baba, H.~Lu, T.B.~Pedersen, X.~Xie.
\newblock A graph model for false negative handling in indoor RFID tracking data.
\newblock In {\em SIGSPATIAL}, pp. 464--467, 2013.

\bibitem{tran2009probabilistic}
T.~Tran, C.~Sutton, R.~Cocci, Y.~Nie, Y.~Diao, P.~Shenoy.
\newblock Probabilistic inference over RFID streams in mobile environments.
\newblock In {\em ICDE}, pp. 1096--1107, 2009.

\bibitem{DBLP:conf/edbt/FazzingaFFP14}
B.~Fazzinga, S.~Flesca, F.~Furfaro, F.~Parisi.
\newblock Cleaning Trajectory Data of RFID-monitored Objects through Conditioning under Integrity Constraints.
\newblock In {\em EDBT}, pp. 379--390, 2014.

\bibitem{nie2009probabilistic}
Y.~Nie, Z.~Li, S.~Peng, Q.~Chen.
\newblock Probabilistic Modeling of Streaming RFID Data by Using Correlated Variable-duration HMMs.
\newblock In {SERA}, pp. 72--77, 2009.

\bibitem{kirshner2005modeling}
S.~Kirshner.
\newblock Modeling of multivariate time series using hidden Markov models.
\newblock PhD thesis, Long Beach, CA, USA, 2005.

\bibitem{viterbi1967error}
A J.~Viterbi.
\newblock Error bounds for convolutional codes and an asymptotically optimum decoding algorithm.
\newblock In {{IEEE} Trans. Information Theory}, pp. 13(2):260--269, 1967.


% \bibitem{koch2008conditioning}
% C.~Koch, D.~Olteanu.
% \newblock Conditioning probabilistic databases.
% \newblock In {\em VLDB}, pp. 313--325, 2008.

%
% \bibitem{floerkemeier2004issues}
% C.~Floerkemeier, M.~Lampe.
% \newblock Issues with RFID usage in ubiquitous computing applications.
% \newblock In {\em Pervasive Computing}, pp. 188--193, 2004.
%
% \bibitem{derakhshan2007rfid}
% R.~Derakhshan, M.E.~Orlowska, X.~Li.
% \newblock RFID data management: challenges and opportunities.
% \newblock In {\em IEEE International conference on RFID}, vol. 10, 2007.
%
% \bibitem{DBLP:conf/icde/LuCJ12}
% H.~Lu, X.~Cao, and C.~S. Jensen.
% \newblock A foundation for efficient indoor distance-aware query processing.
% \newblock In {\em ICDE}, pp. 438--449, 2012.
%
% \bibitem{baba2014handling}
% A.I.~Baba, H.~Lu, T.B.~Pedersen, X.~Xie.
% \newblock Handling false negatives in indoor RFID data.
% \newblock In {\em MDM}, pp. 117--126, 2014.

}
\end{thebibliography}

\end{frame}


%------------------------------------------------
\section{4. Indoor Movement Analysis} % Sections can be created in order to organize your presentation into discrete blocks, all sections and subsections are automatically printed in the table of contents as an overview of the talk
%------------------------------------------------

%------------------------------------------------
\section{5. Appendix} % Sections can be created in order to organize your presentation into discrete blocks, all sections and subsections are automatically printed in the table of contents as an overview of the talk
%------------------------------------------------

\subsection{5.1 Managing Evolving Uncertainty in Trajectory Databases} % A subsection can be created just before a set of slides with a common theme to further break down your presentation into chunks

\begin{frame}
\frametitle{About This Work...}

\emph{Managing Evolving Uncertainty in Trajectory Databases}~\cite{jeung2014managing}\\
H.~Jeung, H.~Lu, T.B.~Pedersen, S.~Sathe, M L.~Yiu.\\~\\

\begin{itemize}
  \item Published at \emph{TKDE' 2014}.
  \item A flexible trajectory modeling approach were proposed that takes into account model-inferred actual positions, time-varying uncertainty, and nondeterministic uncertainty ranges.
  \item Three estimators that effectively infer evolving densities of trajectory data were developed.
\end{itemize}

\end{frame}

%------------------------------------------------

\begin{frame}
\frametitle{Motivation}

\begin{itemize}
  \item \emph{Uncertainty management} is a central issue in trajectory databases
  \begin{fitemize}
    \item a common principle -- ``location uncertainty is captured by a certain range centered on the position recorded in the database''
  \end{fitemize}
  \item The old principle as the basis for the uncertain trajectory modelings is incapable of effectively capturing various types of uncertainty caused from different positioning sources
  \begin{fitemize}
    \item A location reported from a positioning system already bears some positional error, which implies that the exact location may not be identical to the reported position
    \item Positional errors may vary over time, as a result, the uncertainty should also change along time
    \item Bounding an area of uncertainty may cause loss of information
  \end{fitemize}
\end{itemize}

\end{frame}

%------------------------------------------------

\begin{frame}
\frametitle{Contributions}

\begin{itemize}

  \item \textbf{Evolving-density trajectory model}\quad to introduce a new uncertain trajectory model that represents a trajectory as time-dependent Gaussian distribution.

  \item \textbf{Evolving density estimators}\quad to propose three \emph{evolving density estimators} that infer time-varying densities of location data.

  \item \textbf{Efficient query processing}\quad to present an effective mechanism that indexes evolving density trajectories, and efficiently evaluates probabilistic range queries using the indexes.

\end{itemize}

\end{frame}

%------------------------------------------------

\begin{frame}
\frametitle{Existing Uncetain Trajectory Models}

There are two major reasons why uncertainty occurs in trajectory data:~\cite{pfoser1999capturing}
\begin{fitemize}
  \item one is known as \emph{measurement error} which is caused by limited accuracy of positioning technology, e.g., GPS error.
  \item the other is \emph{sampling error} that originates from discrete sampling of continuous movements of an object.
\end{fitemize}

Various \conceptbf{uncertainty models} have been proposed. These models commonly represent a trajectory using a sequence of uncertainty areas, so-called \emph{uncertain trajectory}.\\~\\

Each of the uncertainty areas captures the measurement and sampling errors.

\end{frame}

%------------------------------------------------

\begin{frame}
\frametitle{Existing Uncetain Trajectory Models}

\begin{figure}[tb]
  \includegraphics[width=\columnwidth]{figures/5-1/5-1-1.pdf}
\end{figure}

\vspace{10pt}

\footnotesize
\begin{tabular}{|l|c|c|c|c|}
\hline
\textbf{Models} & (a)Beads & (b)Cylinder & (c)Grid & (d)Network-constrained\\
\hline
\textbf{Works} & \cite{hornsby2002modeling,kuijpers2007trajectory} & \cite{frentzos2009effect,trajcevski2004managing,trajcevski2002geometry} & \cite{pelekis2009clustering,zhang2009effectively} & \cite{ding2008utr,ding2004uncertainty}\\
\hline
\end{tabular}

\end{frame}

%------------------------------------------------

\begin{frame}
\frametitle{Pitfalls of Existing Uncetain Trajectory Models}

\textrm{I.} \quad the uncertain trajectory models generally regard a location measured from positioning technology as a precise, actual location of an object, while modeling an uncertainty range based on the reported position as center.\\~\\

\textrm{II.} \quad some of the uncertain trajectory models assume that the degree of uncertainty is constant regardless of the range of location or time.\\~\\

\textrm{III.} \quad the uncertain trajectory models bound the area of location uncertainty, typically using a circle with a user-specified radius. (\emph{it is inevitable to miss out some information when data processing is performed over any bounded uncertainty areas on unbounded distributions})

\end{frame}

%------------------------------------------------

\begin{frame}
\frametitle{Pitfalls of Existing Uncetain Trajectory Models}

\textrm{IV.} \quad most of the models assume that the probability density function (PDF) of a location is given, \emph{it is however a non-trival problem to compute the parameters of a PDF}. \\~\\

\textrm{V.} \quad some models require additional data beyond location coordinates: the beads model uses maximum speed of object to determine the thickness of ellipse, while the network-constrained model needs map data encompassing the coverage of trajectories.

\end{frame}

%------------------------------------------------

\begin{frame}
\frametitle{Evolving-Density Trajectory: Key Principles}

\fsize{
\begin{table}
\begin{tabular}{|c|c|c|}
\hline
\textbf{Approaches} & \textbf{Applications} & \textbf{Works}\\
\hline
\emph{Kalman Filtering} & GPS data & \cite{lamarca2008location}\\
\hline
\emph{Particle Filtering} & mobile RFID readings & \cite{tran2009probabilistic}\\
\hline
\emph{Map Matching} & network-constrained objects' locations & \cite{brakatsoulas2005map}\\
\hline
\emph{Sensor Fusion} &  & \cite{bessho2009location}\\
\hline
\end{tabular}
\caption{Approaches for processing noise-contaminated raw location data}
\end{table}
}

\begin{block}{Actual Position}
  Above approaches provide means that can infer more reliable positions where an object was actually located. The evolving-density trajectory model supports such an inferred location as an actual location of the object, which serves as the center point of an uncertainty range.
\end{block}

\end{frame}

%------------------------------------------------

\begin{frame}
\frametitle{Evolving-Density Trajectory: Key Principles}

\begin{block}{Gaussian Distribution}
  Measurement errors in positioning typically obey Gaussian distributions. The approach models that an object's location follows Gaussian distributions.
\end{block}

\begin{columns}

  \column{0.5\textwidth}
  \begin{figure}[tb]
    \includegraphics[width=\columnwidth]{figures/5-1/5-1-2.pdf}
  \end{figure}

  \column{0.5\textwidth}
  \begin{example}
    \fsize{
      The uncertain position models the object's acutal location as the mean $\mu$, corresponding to the raw position $p$. In addition, the standard deviation $\sigma$ reflects the degree of uncertainty with respect to $\mu$.
    }
  \end{example}

\end{columns}

\end{frame}

%------------------------------------------------

\begin{frame}
\frametitle{Evolving-Density Trajectory: Key Principles}

\begin{block}{$\sigma$-Driven, Nondeterministic Uncertainty Range}
  Do not represent an uncertainty range in a deterministic manner. Instead, keep only the information of deviation $\sigma$, and then dynamically compute the minimum bound for each data point (distribution) that can satisfy a given query condition.
\end{block}

\end{frame}

%------------------------------------------------

\begin{frame}
\frametitle{Evolving-Density Trajectory: Key Principles}

\begin{block}{Time-Dependent Uncertainty}
  To reflect the time-varying errors caused from positioning systems, it models each uncertain position in an evolving-density trajectory differs from another, meaning that each uncertain position has different values for $\mu$ and $\sigma$.
\end{block}

\end{frame}

%------------------------------------------------

\begin{frame}
\frametitle{Evolving-Density Trajectory: Key Principles}

\begin{block}{Linear Evolution of Distribution}
  In-between two consecutive uncertain positions, it assumes that the distributions of actual positions evolve linearly. Consider an object's movements between two positions are commonly modeled as linear in certain trajectories, moreover, it is efficient to compute the probabilistic queries when evolving the linear evolution of probability distribution.
\end{block}

\end{frame}

%------------------------------------------------

\begin{frame}
\frametitle{Framework Overview}

\begin{columns}

  \column{0.5\textwidth}
  \begin{figure}[tb]
    \includegraphics[width=\columnwidth]{figures/5-1/5-1-3.pdf}
  \end{figure}

  \column{0.5\textwidth}
  \ssize{
  \conceptbf{Evolving Density Estimator}\quad computes the probability distribution of an object's position at each time. The component takes a certain number of recent positions in a trajectory, and infers a Gaussian distribution at a current time. Domian-specific knowledge is supported for user-given estimator.
  }
\end{columns}

\vspace{25pt}

\begin{columns}

  \column{0.5\textwidth}
  \ssize{
  \conceptbf{Trajectory Database}\quad manages not only the raw positions, but also the correponding probability distributions derived from an evolving density estimator.
  }

  \column{0.5\textwidth}
  \ssize{
  \conceptbf{Query Processor}\quad supports efficient processing of probabilistic range queries on the evolving-density trajectories managed in the system.
  }

\end{columns}

\end{frame}

%------------------------------------------------

\begin{frame}
\frametitle{Evolving Density Estimator}

\fsize{

Generalized AutoRegressive Conditional Heteroskedasticity - \conceptbf{GARCH} model~\cite{shumway2010time} is a well-established stochastic volatility model that is generally used to assess an investment risk in finance, since volatility represents the degree of deviation from what the data is supposed to be, reflecting a measure of risk for investing.\\~\\

The implemented estimators have extended the GARCH model to handle multi-dimensional location data since the model can assess the deviation of a raw position from where the corresponding actual position is supposed to reside on.\\~\\

The estimators employ a sliding window that takes a $H$ number of consecutive positions for the estimation, and then repeat the same estimation process using the next $H$ positions.\\~\\

Given a sliding window of positions $S^{H}_{t-1} \in \mathbb{R}^{H \times 2}$, the estimators infer two quantities: \textbf{1) the expected true position $\hat{p_t} = (\hat{x_t}, \hat{y_t})$ at time $t$; 2) a standard deviation $\hat{\sigma_t}$}.

}

\end{frame}

%------------------------------------------------

\begin{frame}
\frametitle{Conditional Correlation Estimator}

\conceptbf{Conditional Correlation Estimator ($C^2$-Est)} uses a multivariate mean inference model for estimating the expected true position $\hat{p_t} = (\hat{x_t}, \hat{y_t})$, where $\hat{x_t}$ and $\hat{y_t}$ are the $x$- and $y$-coordinate, respectively.\\~\\

VAR (\underline{V}ector \underline{A}uto\underline{R}egressive) Model, $VAR(k)$\footnote{$k$ represents the model order} can exploit the interdependencies of $\hat{x_t}$ and $\hat{y_t}$ for inferring the actual position $\hat{p_t}$.

\begin{equation}
  \hat{p_t} = \phi_0 + \sum_{j=1}^{k}\phi_j p_{t-j}
\end{equation}

where $\phi_1,...,\phi_k$ are autoregressive coefficients of each size $2 \times 2$, $\phi_0$ is a $2$-dimensional vector, and $t > k$.~\cite{shumway2010time}

\end{frame}

%------------------------------------------------

\begin{frame}
\frametitle{Conditional Correlation Estimator}

For inferring the deviation $\hat{\sigma_t}$, $C^2$-Est uses the constant conditional correlation (CCC) model~\cite{bauwens2006multivariate}. \\~\\

The 2-by-2 conditional variance matrix of $p_i$, denoted as $\Lambda_i$, is defined as:

\begin{equation}
  \Lambda_i = Var(p_i - \hat{p_i} | F_{i-1}), \Lambda_i = Var(a_i | F_{i-1})
  \label{equation:variance_matrix}
\end{equation}

where $Var(a_i | F_{i-1})$ is the variance matrix of $a_i$, given all the information $F_{i-1}$ available until time $i-1$.

\end{frame}

%------------------------------------------------

\begin{frame}
\frametitle{Conditional Correlation Estimator}

The CCC model uses the errors $a_i$ from the VAR($k$) model to represent the conditional variance matrix $\Lambda_i$ in Equation.~(\ref{equation:variance_matrix}).

\begin{equation}
  a_i = \Lambda_i^{1/2}, \Lambda_{nm,i} = \rho_{nm} \sqrt{\lambda_{nn,i}\lambda_{mm,i}}
\end{equation}

where $\Lambda_{nm,i}$ is the value on row $n$ and column $m$ of matrix $\lambda_{i}$, $\lambda_{nn,i}$ and $\lambda_{mm,i}$ are defined using an univariate GARCH(1,1) model, $\rho_{nm}$ are the constant conditional correlations with $\rho_{nn} =1$.\\~\\

Given $\rho_{nm}$ and the univariate GARCH(1,1) models for $\lambda_{nn,i}$ and $\lambda_{mm,i}$, $C^2$-Est first infers $\lambda_{nn,t}$ and $\lambda_{mm,t}$ using the univariate GARCH models.

\begin{equation}
  \hat{\Lambda_t} = (\rho_{nm} \sqrt{\lambda_{nn,t} \lambda_{mm,t}})
\end{equation}

\end{frame}

%------------------------------------------------

\begin{frame}
\frametitle{Radial Estimator}

\conceptbf{Radial Estimator (R-Est)} also uses VAR model for inferring the expected true position $\hat{p_t}$. Differently, it employs a more efficient model, i.e., Radial GARCH model, for inferring $\sigma_t$.\\~\\

Specifically, the R-Est estimator starts by computing the Euclidean norms of the errors $a_i$ by the VAR($k$) model, denoted at $\gamma_i$.\\~\\

$a_i$ are the possible varations, such that positions $p_i$ are derived from the expected true positions $\hat{p_i}$. This means that $\gamma_i$ for each $i$ gives the uncertainty. Thus, R-Est uses $\gamma_i$ for inferring the deviation $\hat{\sigma_t}$.

\end{frame}

%------------------------------------------------

\begin{frame}
\frametitle{AutoRegressive Radial Estimator}

\conceptbf{AutoRegressive Radial Estimator (AR-Est)} is a variant of R-Est. For inferring the deviation $\hat{\sigma_t}$, this new estimator takes the same RGARCH model used in R-Est.\\~\\

AR-Est takes a different approach for the inference of an actual position. It decomposes the multivariate inference into two separate one-dimensional inferences, it uses univariate AutoRegressive models for inferring $\hat{x_t}$ and $\hat{y_t}$ separately.\\~\\

Given a sliding window $S^{H}_{t-1}$, the AR($l$) model models $x_i = \hat{x_i} + a_{x,i}$, where $t-H \leq i \leq t-1$. Given an AR($l$) model, it infers the expected true position $\hat{x_t}$ as:~\cite{shumway2010time}

\begin{equation}
  \hat{x_t} = \phi_{x0} + \sum_{j=1}^{l}\phi_{xj}p_{t-j}
\end{equation}

\end{frame}

%------------------------------------------------

\begin{frame}
\frametitle{Comparision of Estimators}

\begin{block}{Comparison of $C^2$-Est, R-Est, AR-Est}
\fsize{
  A multivariate model $\rightarrow$ A univariate model : requires a considerably higher number of parameters; might perform very well while incurring more computation for setting appropriate parameters. \\

  Accurate Inference for Actual Position: $C^2$-Est $>$ AR-Est $>$ R-Est. \\

  Running Time: $C^2$-Est $>$ AR-Est $>$ R-Est.
}
\end{block}

\begin{footnotesize}
\begin{table}
\begin{tabular}{|c|c|c|c|}
\hline
 & \textbf{\tabincell{c}{inferred point \\ $\hat{p_t} = (\hat{x_t}, \hat{y_t})$}} & \textbf{\tabincell{c}{deviation \\  $\hat{\sigma_t}$}} & \textbf{\tabincell{c}{time \\ complexity}} \\
\hline
$C^2$-Est & $VAR(2)$ & CCC GRAPH(1,1) & $O(14 \cdot H)$ \\
\hline
R-Est & $VAR(2)$ & radial GRAPH(1,1) & $O(9 \cdot H)$ \\
\hline
AR-Est & $AR(2)$ & radial GRAPH(1,1) & $O(3 \cdot H)$ \\
\hline
\end{tabular}
\caption{\tiny Summary of Evolving Density Estimators}
\end{table}
\end{footnotesize}

\end{frame}

%------------------------------------------------

\begin{frame}
\frametitle{Comparision of Estimators}

\begin{block}{Comparison with Alternatives}
\fsize{
  Kalman filters~\cite{lamarca2008location} / Particle filters~\cite{tran2009probabilistic} could also be used to implement the evolving density estimator. However, Kalman filters are capable of estimating an expected actual position only; Particle filters may have a high time complexity for the smoothing.
}
\end{block}

\begin{tiny}
\begin{table}
\begin{tabular}{|c|c|c|c|c|c|}
\hline
\textbf{Approaches} & \textbf{\tabincell{c}{Estimation \\ Accuracy}} & \textbf{\tabincell{c}{Estimation \\  Efficienty}} & \textbf{\tabincell{c}{Actual \\ Position}} & \textbf{\tabincell{c}{Uncetainty \\ Range}} & \textbf{\tabincell{c}{Multi \\ Dimensional}}\\
\hline
\tabincell{c}{$C^2$-Est  R-Est \\ AR-Est} & high & high & \checkmark & \checkmark & \checkmark \\
\hline
\tabincell{c}{dynamic density \\ metrics ~\cite{sathe2011creating}} & high & high & \checkmark & \checkmark & $\times$ \\
\hline
\tabincell{c}{kalman \\ filtering~\cite{lamarca2008location}} & high & medium & \checkmark & $\times$ & \checkmark\\
\hline
\tabincell{c}{particle \\ filtering~\cite{tran2009probabilistic}} & very high & low & \checkmark & \checkmark & \checkmark\\
\hline
\tabincell{c}{map \\ matching~\cite{brakatsoulas2005map}} & very high & medium & \checkmark & $\times$ & \checkmark \\
\hline
\end{tabular}
\caption{\tiny Comparison of Alternatives for Evolving Density Estimators}
\end{table}
\end{tiny}

\end{frame}

%------------------------------------------------

\begin{frame}
\frametitle{Comparision of Estimators}

\begin{block}{Accuracy Measure of Estimation}
\fsize{
  Measuring the accuracy of a given estimator is very difficult since actual distribution are unobservable, meaning that there are no ground truths to compare.\\
  \emph{density distance}~\cite{sathe2011creating} is suggested as a solid mathematical means for measuring the distance between the probability density obtained from an evolving density estimator and its corresponding real (ideal) density.
}
\end{block}

\end{frame}

%------------------------------------------------

\begin{frame}
\frametitle{Probabilistic Range Query on Evolving-density Trajectories}

\begin{itemize}
  \setlength{\itemsep}{2em}
  \item \conceptbf{Probabilistic range queries} are perhaps the most common query type on uncertain trajectories
  \begin{fitemize}
    \item effectively retrieve uncertain objects or trajectories using solid mathematical theory
  \end{fitemize}
  \item Extend the definition of probabilistic range queries on evolving-density trajectory database
  \item Present access methods to index evolving-density trajectories
  \begin{fitemize}
    \item teh algorithm for evaluating the queries based on the indexes
  \end{fitemize}
\end{itemize}

\end{frame}

%------------------------------------------------

\begin{frame}
\frametitle{Definitions}

\begin{definition}[Presence Probability]
  Given an uncertain object $u$, a circular query range $\odot(q, r_q)$ centered at $q$ with radius $r_q$, the presence probability of $u$ in the range $\odot(q, r_q)$ is defined as:
  \begin{equation*}
    Pr(u, \odot(q, r_q)) = \int_{u \cap \odot(q, r_q)} pdf(u, p) dp
  \end{equation*}
  where $pdf(u, p)$ denotes the probability density of object $u$ at point $p$.
\end{definition}

\begin{definition}[Snapshot Object]
  Given an uncertain trajectory $o$ and a timestamp $t_q$, we define $o(t_q)$ as the uncertain object of $o$ at time $t_q$.
\end{definition}

\end{frame}

%------------------------------------------------

\begin{frame}
\frametitle{Definitions}

\begin{definition}[Probabilistic Range Query]
  Given a query range $\odot(q, r_q)$, a timestamp $t_q$, and a probability threshold $\rho$, a probabilistic range query $\mathcal{R}$ on an evolving-density trajectory database $\mathcal{D}$ returns all trajectories that have presence probabilities in $\odot(q, r_q)$ above $\rho$.
  \begin{equation*}
    \mathcal{R}_{\mathcal{D}}(\odot(q,r_q), t_q) = \{ o \in \mathcal{D}: Pr(o(t_q), \odot(q, r_q)) > \rho \}
  \end{equation*}
\end{definition}

\end{frame}

%------------------------------------------------

\begin{frame}
\frametitle{Compute the Uncertain Object}

\ssize{
\begin{problem}[compute the uncertain object $o(t_q)$ for the trajectory $o$ at time $t_q$]
  Let $o.t_1, o.t_2, \ldots, o.t_m$ be an increasing sequence of sampling timestamps for the trajectory $o$, i.e., we have $o.t_1 \leq o.t_2 \leq \ldots \leq o.t_m$, in addition, let $o.\mu_i$ and $o.\sigma_i$ be the mean and standard deviation of $o$ at time $o.t_i$.
\end{problem}
}

\vspace{-10pt}

\begin{columns}

  \column{0.2\textwidth}
  \begin{figure}[tb]
    \includegraphics[width=\columnwidth]{figures/5-1/5-1-4.pdf}
  \end{figure}

  \column{0.8\textwidth}
  \begin{block}{CASE 1 \quad $t_q = o.t_i$ for some $i \in [1, m]$}
    In the case, $o.t_q$ is an uncertain object with the mean $o.\mu_i$ and deviation $o.\sigma_i$
    \begin{equation*}
      o(t_q).\mu = o.\mu_i, o(t_q).\sigma = o.\sigma_i
    \end{equation*}
  \end{block}

\end{columns}

\vspace{-5pt}

\begin{columns}

  \column{0.2\textwidth}
  \begin{figure}[tb]
    \includegraphics[width=\columnwidth]{figures/5-1/5-1-5.pdf}
  \end{figure}

  \column{0.8\textwidth}
  \begin{block}{CASE 2 \quad $o.t_i < t_q < o.t_{i+1}$ for some $i \in [1, m)$}
    In the case, $o.t_q$ is an uncertain object with the mean $o.\mu_i$ and deviation $o.\sigma_i$
    \vspace{-10pt}
    \begin{equation*}
      \begin{split}
      & o(t_q).\mu = o.\mu_i + (o.\mu_{i+1} - o.\mu_i) \cdot \frac{t.q-o.t_i}{o.t_{i+1}-o.t_i} \\
      & o(t_q).\sigma = o.\sigma_i + (o.\sigma_{i+1} - o.\sigma_i) \cdot \frac{t.q-o.t_i}{o.t_{i+1}-o.t_i}
      \end{split}
    \end{equation*}
  \end{block}

\end{columns}

\end{frame}

%------------------------------------------------

\begin{frame}
\frametitle{Indexing Evolving-Density Trajectories}

Taking into account the temporal information, to index evolving-density trajectories with two complementary components:


\end{frame}

%------------------------------------------------

\begin{frame}[allowframebreaks]
\frametitle{References}

\begin{thebibliography}{99} % Beamer does not support BibTeX so references must be inserted manually as below
\bibliographystyle{abbrv}
\fsize{

\bibitem{jeung2014managing}
H.~Jeung, H.~Lu, T.B.~Pedersen, S.~Sathe, M L.~Yiu.
\newblock Managing evolving uncertainty in trajectory databases.
\newblock In {\em TKDE}, pp. 1692--1705, 2014.

\bibitem{pfoser1999capturing}
D.~Pfoser, C.S.~Jensen.
\newblock Capturing the uncertainty of moving-object representations.
\newblock In {\em SSD}, pp. 111--131, 1991.

\bibitem{hornsby2002modeling}
K.~Hornsby, M.J.~Egenhofer.
\newblock Modeling moving objects over multiple granularities.
\newblock In {\em Annals of Mathematics and Artificial Intelligence}, pp. 177--194, 2002.

\bibitem{kuijpers2007trajectory}
B.~Kuijpers, W.~Othman.
\newblock Trajectory databases: Data models, uncertainty and complete query languages.
\newblock In {\em ICDT}, pp. 224--238, 2007.

\bibitem{frentzos2009effect}
E.~Frentzos, K.~Gratsias, Y.~Theodoridis.
\newblock On the effect of location uncertainty in spatial querying.
\newblock In {\em TKDE}, pp. 366--383, 2009.

\bibitem{trajcevski2004managing}
G.~Trajcevski, O.~Wolfson, K.~Hinrichs, S.~Chamberlain.
\newblock Managing uncertainty in moving objects databases.
\newblock In {\em TODS}, pp. 463--507, 2004.

\bibitem{trajcevski2002geometry}
G.~Trajcevski, O.~Wolfson, F.~Zhang, S.~Chamberlain.
\newblock The geometry of uncertainty in moving objects databases.
\newblock In {\em EDBT}, pp. 233--250, 2001.

\bibitem{pelekis2009clustering}
N.~Pelekis, I.~Kopanakis, E.~Kotsifakos, E.~Frentzos, Y.~Theodoridis.
\newblock Clustering Trajectories of Moving Objects in an Uncertain World.
\newblock In {\em ICDM}, pp. 417--427, 2009.

\bibitem{zhang2009effectively}
M.~Zhang, S.~Chen, C.S.~Jensen, B.C.~Ooi, Z.~Zhang.
\newblock Effectively indexing uncertain moving objects for predictive queries.
\newblock In {\em VLDB}, pp. 1198--1209, 2009.

\bibitem{ding2008utr}
Z.~Ding.
\newblock UTR-tree: An index structure for the full uncertain trajectories of network-constrained moving objects.
\newblock In {\em MDM}, pp. 33--40, 2008.

\bibitem{ding2004uncertainty}
Z.~Ding, R.H.~G{\"u}ting.
\newblock Uncertainty management for network constrained moving objects.
\newblock In {\em DEXA}, pp. 411-421, 2004.

\bibitem{lamarca2008location}
A.~LaMarca, E.~De Lara.
\newblock Location systems: An introduction to the technology behind location awareness.
\newblock In {\em Synthesis Lectures on Mobile and Pervasive Computing}, pp. 1--122, 2008.

\bibitem{tran2009probabilistic}
T.~Tran, C.~Sutton, R.~Cocci, Y.~Nie, Y.~Diao, P.~Shenoy.
\newblock Probabilistic inference over RFID streams in mobile environments.
\newblock In {\em ICDE}, pp. 1096--1107, 2009.

\bibitem{brakatsoulas2005map}
S.~Brakatsoulas, D.~Pfoser, R.~Salas, C.~Wenk.
\newblock On map-matching vehicle tracking data.
\newblock In {\em VLDB}, pp. 853--864, 2005.

\bibitem{bessho2009location}
M.~Bessho, N.~Koshizuka, S.~Kobayashi, K.~Sakamura.
\newblock Location systems for ubiquitous computing.
\newblock In {\em Journal of IEICE}, pp. 249--255, 2009.

\bibitem{shumway2010time}
R.H.~Shumway, D.S.~Stoffer.
\newblock Time Series Analysis and Its Applications.
\newblock In {\em Springer Science \& Business Media}, 2010.

\bibitem{bauwens2006multivariate}
L.~Bauwens, S.~Laurent, VK.~Jeroen.
\newblock Multivariate GARCH models: a survey.
\newblock In {\em Journal of applied econometrics}, pp. 79--109, 2006.

}
\end{thebibliography}

\end{frame}


%----------------------------------------------------------------------------------------

\begin{frame}
\Huge{\centerline{The End. Thanks :)}}
\end{frame}

%----------------------------------------------------------------------------------------

\end{document}
